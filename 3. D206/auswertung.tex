\newpage
\begin{table}[H]
    \centering
    \caption{Die Messwerte}
    \label{tab:data}
    \begin{tabular}{ S [table-format=2.0] S [table-format=2.1] S[table-format=2.2] S S[table-format=2.1] S [table-format=3.0] }
        \toprule
        {$t \mathbin{/} \si{\minute}$} & {$T_1 \mathbin{/} \si{\celsius}$} & {${p^{*}}_1 \mathbin{/} \si{\bar}$} & 
        {$T_2 \mathbin{/} \si{\celsius}$} & {${p^{*}}_2 \mathbin{/} \si{\bar}$} & {$\symup{N} \mathbin{/} \si{\watt}$}\\
        \midrule
        0	& 21.7&	4.0 &	21.7  &  4.1 &   120\\
        0	& 21.7&	4.0 &	21.7  &  4.1&    120\\
        0	& 21.7&	4.0 &	21.7  &  4.1 &   120\\
        3 &	 25.3&	6.0 &	21.5&	3.5&	   120\\
        4 &	 26.4&	6.0 &	20.8&	3.5	&   120\\
        5 &	 27.5&	6.0 &	20.1&	3.4	&   120\\
        6 &	 28.8&	6.5 &	19.2&	3.3	 &  120\\
        7 &	 29.7&	6.5 &	18.5&	3.2	 &  120\\
        8 &	 30.9&	7.0 &	17.7&	3.2	 &  120\\
        9 	& 31.9&	7.0 &	16.9&	3.0	 &  120\\
        10	 &32.9&	7.0 &	16.2&	3.0	 &  120\\
        11	& 33.9&	7.5 &	15.5&	2.9  &   120\\
        12	& 34.8&	7.5 &	14.9&	2.8  &  120\\
        13	& 35.7&	8.0 &	14.2&	2.8 &	120\\
        14	 &36.7&	8.0 &	13.6&	2.7	  &  120\\
        15&	 37.6&	8.0 &	13.0&	2.6	 &   120\\
        16&	 38.4&	8.5 &	12.4&	2.6 &	120\\
        17&	 39.2& 8.5 &	11.7&	2.6	&    120\\
        18&	 40.0&	9.0 &	11.3&	2.5  &   120\\
        19&	 40.7&	9.0 &	10.9&	2.5	 &   120\\
        20&	 41.4&	9.0 &	10.4&	2.4	 &   120\\
        21&	 42.2&	9.0 &	9.9	 &   2.4	 &   120\\
        22&	 42.9&	9.5 &	9.5	&    2.4	 &   120\\
        23&	 43.6&	9.5 &	9.1	&    2.4	 &   120\\
        24&	 44.3&	10.0&	8.7&    2.4	&    120\\
        25&	 44.9&	10.0&	8.3	&    2.4	 &   120\\
        26&	 45.5&	10.0&	8.0&	    2.3	 &   120\\
        27&	 46.1&	10.0&	7.7	&    2.2	 &   122\\
        28&	 46.7&	10.5&	7.4&	    2.2	 &   122\\
        29&	 47.3&	10.5&	7.1&	    2.2	 &   122\\
        30&	 47.8	&10.75&	6.8	&    2.2	 &   122\\
        31&	 48.4&	11.0&	5.6	 &   2.2	&    122\\
        32&	 48.9&	11.0&	4.3	&    2.2	 &   122\\
        33&	 49.4&	11.0&	3.4	&    2.2	 &   122\\
        34&	 49.9&	11.0&	3.0	&    2.2	&    122\\
        35&	 50.3&	11.0&	2.9	&    2.2	 &   122\\
        \bottomrule
    \end{tabular}
\end{table}
\newpage

%5a)
\begin{figure}
    \centering
    \includegraphics[width=\textwidth]{build/plot.pdf}
    \caption{Ein Plot der Messwerte inklusive der gefiteten Funktionen.}
\end{figure}


%5b)
Dies sind die Fit-Funktionen auf die Messwerte. Dabei ist der Fit auf $T_1$ die Funktion $F(t)$ und der auf $T_2$ ist die Funktion $G(t)$.
\begin{align}
    F(t) &=A \cdot t^2+B \cdot t+ C \\
    G(t) &= \frac{D}{1+E \cdot t^{F}}\\
    &\begin{aligned}
        A &=0.00000322 & B&=0.02027984 & C&=21.82008061 \\
        D &=21.58061064 & E&= 0.00000095& F&=1.97775836 
    \end{aligned}\\
    F(t) &=0.00000322t^2+0.02027984t+21.82008061 \\
    G(t) &= \frac{21.58061064}{1+0.00000095t^{1.97775836}}
\end{align}
\\

%5c)
Im folgenden werden die Differentialquotienten der Temperaturen in verschiedenen Stellen berechnet.
Da die Fit Funktionen eine angemessene Abbildung der Messwerte darstellen berechnen wir die Quotienten über das bilden der Ableitung der Fits $F(t)$ und $G(t)$ 
und ihrer anschließenden Punktweisen Auswertung.
Als Punkte wurden hierbei 7er Schritte gewählt, da man daurch regelmäßige Abstände über den Definitonsbereich hat und einen breiten Bereich des selben abdeckt.

\begin{align}
    \frac{\partial F(t)}{\partial t} &=2At*B & \frac{\partial G(t)}{\partial t} &= \frac{-D \cdot F\cdot E \cdot t^{F-1}}{(1+E\cdot t^{F})^{2}}\\
    \frac{\partial F(7)}{\partial t} &=0.0202347600000000   &     \frac{\partial G(7)}{\partial t} &= -0.000271783769486808  \\
    \frac{\partial F(14)}{\partial t} &=0.0201896800000000   &     \frac{\partial G(14)}{\partial t} &=  -0.000535111561070977 \\
    \frac{\partial F(21)}{\partial t} &=0.0201446000000000   &     \frac{\partial G(21)}{\partial t} &= -0.000795117897688127  \\
    \frac{\partial F(28)}{\partial t} &= 0.0200995200000000  &     \frac{\partial G(28)}{\partial t} &= -0.00105276374853994 
\end{align}

%5d)

Anschließend werden die idealen und realen Güteziffern der Wärmepumpe aus den zuvor berechneten Werten bestimmt.

\begin{equation}
    v_\text{ideal}= \frac{Q_1}{A} = \frac{T_1}{T_1-T_2}
\end{equation}





