\section{Durchführung}


\footnotetext[2]{TU Dortmund. D206 Die Wärmepumpe/Anleitung Wärmepumpe.pdf (2020)}

\begin{figure}
    \centering
    \includegraphics[width=0.7\textwidth]{images/Wärmepumpe.png}
    \caption{Schematischer Aufbau des Versuchs \protect \footnotemark[2].}
    \label{img:pump2}
\end{figure}

Für diesen Versuch sind die Kenngrößen Güteziffer, Massendurchsatz und Wirkungsgrad von Interesse.
Um diese zu bestimmen müssen die Temperaturen $T_1$ und $T_2$,die Drücke $p_a$ und $p_b$ und die Leistung $\symup{N}$ gemessen werden.\\
Dafür stehen die befestigten Thermometer, Manometer und Wattmeter zur Verfügung.
Des Weiteren werden die Reservoire von den befestigten Rührmotoren
immer umgerührt, damit sich Termperatur in den Reservoiren gleichmäßig verteilen kann.\\\\

Die Abbildung \ref{img:pump2} zeigt den Versuchsaufbau.\\
In die isolierten Behälter der beiden Reservoire werden $\SI{4}{\kilo\gram}$ Wasser gefüllt.
Zu Beginn werden alle Drücke, Temperaturen und die Kompressorleistung einmal abgelesen.
Dabei ist der tatsächliche Druck immer der, auf den man noch $\SI{1}{\bar}$ addiert.
Anschließend startet man den Kompressor.\\
Nun ließt man in $\SI{60}{\second}$ Abständen alle Messdaten $\SI{35}{\minute}$ lang ab.