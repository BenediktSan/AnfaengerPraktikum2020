\section{Theoretische Grundlagen}

Eine Wärmepumpe ist eine Vorrichtung, die in der Lage ist Wärmefluss von einem kälteren in ein
wärmeres Thermisches-Reservoir herzustellen. Dies geschieht entgegen dem standardmäßigen Prozess,
das der Wärmeflussvon einem wärmeren in ein 
kälteres Reservoir vorliegt. Damit die Wärmpepumpe diese Richtung umkehrt muss also in ihr,
nach dem zweiten Hauptsatz der Thermodynmik noch zusätzliche
mechanische Arbeit verrichtet werden.
Die vom wärmeren Reservoir aufgenommene Wärmemenge $Q_1$ ist also gleich der
Summe des aus dem vom kälteren Reservoir entnommenen Wärmemenge $Q_2$ und der verrichteten Arbeit $A$:
\begin{equation}
  Q_1 = Q_2 + A  
\end{equation}

\subsection{Güteziffer}

Das Verhältnis zwischen der transportierten Wärmemenge und der 
aufgewendeten Arbeit $v=\frac{Q_1}{A}$ nennt man Güteziffer.
Die Güteziffer ist ein Maß für die Effektivität der Wärmepumpe.




