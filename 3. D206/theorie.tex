\section{Theoretische Grundlagen}

Eine Wärmepumpe ist eine Vorrichtung, die in der Lage ist Wärmefluss von einem kälteren in ein
wärmeres Thermisches-Reservoir herzustellen. Dies geschieht entgegen dem standardmäßigen Prozess,
das der Wärmeflussvon einem wärmeren in ein 
kälteres Reservoir vorliegt. Damit die Wärmpepumpe diese Richtung umkehrt muss also in ihr,
nach dem zweiten Hauptsatz der Thermodynamik noch zusätzliche
mechanische Arbeit verrichtet werden.
Die vom wärmeren Reservoir aufgenommene Wärmemenge $Q_1$ ist also gleich der
Summe des aus dem vom kälteren Reservoir entnommenen Wärmemenge $Q_2$ und der verrichteten Arbeit $A$:
\begin{equation}
  Q_1 = Q_2 + A  
  \label{eqn:transp}
\end{equation}

\subsection{Güteziffer}

Das Verhältnis zwischen der transportierten Wärmemenge und der 
aufgewendeten Arbeit $v=\frac{Q_1}{A}$ nennt man Güteziffer.
Die Güteziffer ist ein Maß für die Effektivität der Wärmepumpe.
Aus dem zweiten Hauptsatz der Thermodynamik lässt sich zudätzlich noch herleiten,
dass die Summe der reduzierten Wärmemenge 
\begin{equation}
    \frac{Q_1}{t_1}-\frac{Q_2}{T_2}=0
\end{equation}
Durch einsetzen der letzten beiden Formeln und \refeq{eqn:transp} ineinander
ergibt sich eine Aussage für die ideale und damit maximal mögliche Güteziffer:
\begin{equation}
    v_\text{ideal}= \frac{T_1}{T_1-T_2}
    \label{eqn:videal}
\end{equation}
Hier zeigt sich, dass die Effektivität einer Wärmepumpe
umso größer ist, desto kleiner die Temperatur der Reservoire ist. \\
Die reale Güteziffer $v_\text{real}$ muss offensichtlicherwiese
kleiner als $v_\text{ideal}$ sein, da nicht alle Arbeit direkt für den zuvor 
beschriebenen Prozess genutzt wird, wodurch insgesamt mehr Arbeit nötig wird.\\
$v_\text{real}$ berechnet sich mit einem über die Differenzenquotienten
einer, über ein Zeitintervall gewonnenen, Wärmemenge und Temperatur.
\begin{equation}
    \frac{\increment Q_1}{\increment t} = \left(m_1 c_w + m_k c_k \right)\frac{\increment T_1}{\increment t}
    \label{eqn:delQ}
\end{equation}
Mit Hilfe der über das Zeitintervall gemittelten Leistungsaufnahme $\symup{N}$
lässt sich dann $v_\text{real}$ bestimmen.
\begin{equation}
    v_\text{real}= \frac{\increment{Q_1}}{\increment t \cdot \symup{N} }
    \label{eqn:vreal1}
\end{equation}\\
Für unsere Rechnungen verwenden wir aber keinen Differenzenquotienten sonderen rechnen mit einem Differential,
einer auf die Messwerte gefiteten Funktion.

\subsection{Massendurchsatz}

Der Massendurchsatz ist der Durchsatz des Transportmediums, in diesem Fall
Dichloridfluormethan $\ce{Cl2F2C}$, von einem Abschnitt des Kreislaufs in einen anderen.\\
Da die Wärmeentnahme aus dem kälteren Reservoir durch Verdampfung des Medium geschieht, 
lässt sich mit Hilfe der Verdampfungswärme des Mediums und dem Differenzenquotienten 
der entnommenen Wärmemenge des Reservoirs, der Durchsatz bestimmen.
Dies geschieht über eine zu \refeq{eqn:vreal1} ähnliche Formel, nur mit korrespondierenden
Werten für $T_2$.
\begin{equation}
    \frac{\increment Q_2}{\increment t} = L\cdot \frac{\increment m}{\increment t}
    = \left(m_2 c_w + m_k c_k \right)\frac{\increment T_2}{\increment t}
\end{equation}

\subsection{Die mechanische Kompressorleistung} \label{kompressor}
\begin{align*}
    \intertext{Die Kompressorleistung ergibt sich als die Kompressionsarbeit pro Zeit, also als:}
    N_\text{mech}&=\frac{\increment A_m}{\increment t}
    \intertext{Desweiteren ist die Kompressionsarbeit die Arbeit, welche verrichtet wird, wenn ein Stoff eines Volumens zu einem anderen Volumen komprimiert wird.}
    A_m&= - \int_{V_a}^{V_b} p \, \symup{d}V
    \intertext{Für die Berechnung der Kompressorleistung  $N_\text{mech}$ darf eine adiabatische Kompression angenommen werden, 
    weswegen die Poissonsche Gelichung genutzt werden kann.}
    p_a V_a^{\kappa} &= p_a V_a^{\kappa} =p V^{\kappa}
    \intertext{Wobei $\kappa$ das Verhältinis der molwärmen der beiden [Dunno] ist.}
    \intertext{So ergibt sich dann zur Berechnung der Leistung die Formel:}
    N_\text{mech} &= \frac{q}{\kappa-1}\left( p_b \sqrt{\kappa}{\frac{p_a}{p_b}}-p_a \right) \frac{1}{\rho} \frac{\increment m}{\increment t}
\end{align*}









