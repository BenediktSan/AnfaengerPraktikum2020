\section{Auswertung}

\subsection{Messungen bis 1 Bar}



Die gefittete Funktion auf die Messwerte wurde mithilfe einer linearen Regression berechnet.
Die Rechnung führt zu der Formel:
\begin{align}
    y&=m\cdot x+n\\
    y&= \cdot x+
\end{align}
Mit Hilfe der Gaskonstante $\symup{R}$   \footnote{Quelle: \url{https://www.chemie.de/lexikon/Universelle_Gaskonstante.html}(Besucht am 09.11.2020)}
und der Steigung der Ausgleichsgerade lässt sich nun eine gute Näherung für $L$, für Werte kleiner als $\SI{1}{\bar}$ berechenen.
\begin{align}
    L&=(-1) \cdot m \cdot \symup{R}\\
    \implies L&\approx\SI{1(1)}{\joule\per\mol}
\end{align}
 












%nervige formel
\subsection{Messung bis 15 Bar}
\begin{align*}
   \symup{R}\cdot T &= \left( p + \frac{A}{V_D^2}\right)\cdot V_D  &
    &\text{mit} &
    A&=\SI{0.9}{\joule\cubic\metre\per\mol\squared}\\
    V_D&=\frac{\symup{R}\cdot T}{2p} \pm \sqrt{\frac{\symup{R}^2\cdot T^2}{4p^2}-\frac{A}{p}}
    \intertext{lalalalalalallalaallalalala}
    L&=T\left(\frac{\symup{R}\cdot T}{2p} \pm\sqrt{\frac{\symup{R}^2\cdot T^2}{4p^2}-\frac{A}{p}}\right)\frac{\symup{d}p}{\symup{d}t}\\
    L&=\frac{T}{p}\left(\frac{\symup{R}\cdot T}{2p} \pm\sqrt{\frac{\symup{R}^2\cdot T^2}{4p^2}-\frac{A}{p}}\right)\frac{\symup{d}p}{\symup{d}t}
    \intertext{Um $\frac{\symup{d}p}{\symup{d}t}$ zu approximieren bestimmen wir eine Fit-Funktion 3.Grades
    für die Temperatur $T$ und den Druck $p$ und differenzieren sie anschließend.}  
\end{align*}
Der Fit auf $p$ und sein Derivat sind: 
\begin{align}
    p(T)&=a\cdot T^3+b\cdot T^2 + c\cdot T+d\\
    \frac{\symup{d}p}{\symup{d}t}&=3a\cdot T^2+2b\cdot T + c
\end{align}
Mit den Werten 
\begin{table}[H]
    \centering
    \sisetup{table-format=1.3}
    \begin{tabular}{ S S [table-format=3.5] @{$ \pm{}$} S [table-format=3.5] S }
        \toprule
        {Parameter} & \multicolumn{3}{c}{ Bestimmte Werte} \\
        \midrule
        \text{a}	&\num{9.29297e-06}  & \num{1.97123e-06} & \; \si{\pascal\per\cubic\kelvin}\\
        \text{b}	&\num{-0.01029}  & \num{0.00256} & \; \si{\pascal\per\kelvin\squared}\\
        \text{c}	&\num{3.84663}  & \num{1.10409} & \; \si{\pascal\per\kelvin}\\
        \text{d}	&\num{-485.74494}  & \num{158.58876} & \; \si{\pascal}\\
        \bottomrule
        \\
    \end{tabular}
\caption {Berechnete Werte für die Polynome der Fit-Funktion gerundet auf die fünfte Nachkommastelle.}
\label{tab:params}
\end{table}
