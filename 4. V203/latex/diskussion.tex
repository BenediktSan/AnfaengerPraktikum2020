\section{Diskussion}
\subsection{Messungen bis 1 Bar}

<<<<<<< HEAD
Für diesen Teil des Versuches gab es bei der Durchführung keine Probleme. Der Startdruck fiel am Anfang auf einen 
zufriedenstellenden Wert und bis auf minimale Druckschwankungen beim erhitzen gab es keine Besonderheiten.
Der aus der Messreihe bestimmte Wert von $L$
||||||| 671da25
Für diesen Teil des Versuches gab es bei der Durchführung keine Probleme. Der Startdruck fiel am Anfang auf einen 
zufriedenstellenden Wert und bis auf minimale Druckschwankungen beim erhitzen gab es keine Besonderheiten.
Der aus der Messreihe bestimmte Wert von $L$ 
=======
Für diesen Teil des Versuches gab es bei der Durchführung keine größeren Probleme. Der Startdruck fiel am Anfang auf einen 
zufriedenstellenden Wert und bis auf minimale Druckschwankungen beim erhitzen gab es keine Besonderheiten. Nur bei den ersten
3 Messpaaren hatte das Wasser noch nicht gesiedet, somit ist nicht sicher ob der Sättigungsdampfdruckbereits vollständig erreicht war,
und müssen mit Vorsicht betrachtet werden. Jedoch scheinen auch die ersten 3 Messwerte gut in die Reihe der anderen Messwerte zu passen.
Der aus der Messreihe bestimmte Wert von $L$ 
>>>>>>> 9eb4e9fc2d2ee0354d9a72073f99131f3fb134f1




\subsection{Messungen zwischen 1 und 15 Bar}

Die Durchführung des Versuches lief ohne irgendwelche direkt erkennbaren Probleme ab.\\
Bei der Auswertung wurde die in \ref{img:plus} zusehende Funktion der in \ref{img:minus} vorgezogen.
Dies liegt, wie zuvor schon erläutert, daran, dass die geuchte Funktion eine negative Steigung besitzen sollte und Werte in der Größenordnung der Theoriewerte besitzen sollte.\\
Beides trifft auf die Erste zu, aber nicht auf die Zweite.
Allerdings gibt es, wie man in \ref{img:plus} sieht, für die ersten Messwerte der gewählten Funktion eine recht hohe Diskrepanz zwischen den Theoriewerten und unserer Funktion für $L$.
Wir vermuten das dies eventuell daran liegen könnte, dass sich der Gleichgewichtszustand, welcher für den Sättigungsdampfdruck benötigt wird, sich erst später einstellt.
Es also ein Problem bei der Durchführung gab, das nicht direkt ersichtlich war.\\
alternativ könnte für die starke Abweichung auch der große Fehler unserer Fit-Funktion mit verantwortlich sein, obwohl sie eigentlich sehr gut auf 
die Messwerte passt. Eine zusätzlicher Faktor wären auch noch die vielen Näherungen die getroffen werden mussten um die Clausius-Clapeyronsche Gleichunglösen zu können
und um die Funktionen für $L$ bilden zu können.\\
Allerdings nähert sich die Funktion für die späteren Messwerte gegen die Theoriewerte an, was ein Indikator dafür seien kann, dass kein grundlegender Fehler vorliegt.\\

