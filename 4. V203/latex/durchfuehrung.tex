\section{Durchführung}

\subsection{kleiner 1 bar}
Für die Messung der Dampfdruckkurve wird die Apperatur aus der Abbildung Nr.3 verwendet. Um das erste Vakuum zu erzeugen wird zunächst 
der Absperrhahn und das Drosselventil geöffnet und das Belülftungsventil geschlossen. Die Wasserstrahlpumpe wird nun solange angestellt, 
bis sich ein konstanter Druck einstellt, der zu erreichende Druck ist hier abhängig von der Wassertemperatur. Bevor nun die 
Wasserstrahlpumpe abgeschaltet wird, muss der Absperrhahn und das Drosselventil geschlossen werden. Als nächstes wird nun die Heizhaube 
eingeschalten und gleichzeitig dafür gesorgt, dass die Kühlflüssigkeit durch die Apperatur fließt. Falls die Substanz nach ein wenig Zeit 
nicht anfängt zu sieden, kann man noch einmal das Drosselventil öffnen um den Druck innerhalb des Gefäßes weiter zu senken. Jetzt wird 
bei laufender Heizung regelmäßig die Siedetemperatur und der Dampfdruck gemessen. Damit höhere Temperaturen erreicht werden können,
muss nach einer Weile der Durchfluß der Kühlflüssigkeit reduziert werden, bis die Kühlflüssigkeit nur noch tropft.


\subsection{1 bis 15 bar}

Die Dampfdruckkurve für Drücke von mehr als einem bar, wwird mit Hilfe  der Apperatur aus Abbildung 4 vermessen.
Um den Versuch zu starten, muss zunächst der Holraum vollständig mit destilliertem und entgastem Wasser befüllt werden.
Während das Wasser nun langsam erhiztzt wird, liest man die Siedetemperatur und den Sättigungsdampfdruck ab, dabei ist daruf zu 
achten, dass der Vollauschlag des Manometers nicht überschritten wird. Das Manometer wird durch ein mit kaltem Wasser gefülltes 
Kühlbecken vor Erhitzung geschützt.