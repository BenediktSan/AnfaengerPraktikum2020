\section{Zielsetzung}

In diesem Versuch soll durch Beobachtung des Phasenübergangs eines Stoffes mit Hilfe der Dampfdruckkurve die 
Verdampfungswärme des Stoffes bestimmt werden. Dabei soll die Temperaturabhängigkeit nicht vernachlässigt werden.

\section{Theoretische Grundlagen}

%\subsection{Phasendiagramm}

Die \glqq Phase\grqq{} eines Stoffes beschreibt in diesem Fall den Aggregatzustand, wir unterscheiden zwischen fest, flüssig und gasförmig.
Zur Veranschaulichung wird hier ein Phasendiagramm von Wasser geutzt in dem die Temperatur gegen den Druck aufgetragen ist und die 3 "Phasen" 
durch 3 Kurven voneinander abgegrenzt sind. Die beiden wichtigen Punkte in dem Diagramm sind der TripelPunkt (TP.) indem die 
3 Aggregatzustände und der kritische Punkt (K.P) in dem der flüssige und gasförmige Zustand in einem 
thermodynamischem Gleichgewicht stehen.
In diesem Experiment untersuchen wir jedoch nur den Übergang: flüssig $\Leftrightarrow$ gasförmig, dieser Übergang wird von der 
Gasdruckkurve beschrieben bzw. von der Siedepunktskurve zwischen dem Triplepunkt und dem kritischem Punkt, charakterisiert wird diese 
Kurve durch die temperaturabhängige Verdampfungswärme $L$. Innerhalb jedes Stoffes werden die Geschwindigkeiten der einzelnen Teilchen
durch die Maxwellsche Geschwindigkeitsverteilung vorgegben. Sobald die Geschwindigkeit eines Teilchen einen gewissen Grenzwert 
überschreitet, kann es aus der flüssig in die gasförmige Phase übergehen. Bei diesem Übergang muessen molekulare Bindungskaefte ueberwunden
werden, dies ist die Verdampfungswärme $L$, diese Energie wird beim Umkehrprozess, der Kondensation wieder frei. Zwischen 
Kondensation und Verdampfung entsteht nach einiger Zeit ein Gleichgewicht mit einem konstantem Sättigungsdruck.
$L$ ist eine temperaturabhängige Groeße, jedoch gibt es Temperaturbereiche in denen die Verdampfungswärme naeerungsweise konstant ist, 
diesen Bereich nutzen wir in diesem Experiment um $L$ zu bestimmen.
\\$\Rightarrow$ Somit beschreibt die molare Verdampfungswärme $L$ eine stoffspezifische Größe , die angibt, wie viel Energie nötig ist,
um ein Mol einer Flüssigkeit isotherm und isobar zu verdampfen.\\
Der Sättigungsdruck lässt sich nicht durch die allgemeine Gasgleichung:
\begin{equation}
    pV=RV\text{,mit }R=\text{allgemeine Gaskonstante }= \SI{8.314}{\joule\per\mole\kelvin}\\
    \label{eqn:Gasgl}
\end{equation}
\\
beschreiben, da der Sättigungsdruck nicht von dem Volumen des Gases, jediglich von der Temperaturen der Flüssigkeit 
und des Gases abhängen.

Dampfdruck Kurve wird durch reversiblen Kreislauf berechnet.\\
isotherme und isobare vedampfung $\Leftrightarrow$ isotherme und isobare kondensation.\\

Ausgangspunkt A, dp und dT wird hinzugefügt $\rightarrow$ Zustand B\\
Die Flüssigkeit vedampft isobar und isotherm unter aufnahme der Verdampfungswärme $\rightarrow$ Zustand C\\
Durch Wärmeentzug wird der Dampf wieder auf die Temperatur T abgekühlt $\rightarrow$ Zustand D\\


\begin{equation}
    (C_\text{F}-C_\text{D})dT + dL = (V_\text{D}-V_\text{F})dP\\
    \label{eqn:long}
\end{equation}
\\
\begin{equation}
    \sum_{i}\frac{Q_\text{i}}{T_\text{i}}=0\\
    \label{eqn:sum}
\end{equation}
\\
\begin{equation}
    (V_\text{D}-V_\text{F})dp =\frac{L}{T}dT\\
    \label{eqn:sum}
\end{equation}
\\
\begin{align}
p &= \text{exp}(C)\cdot \text{exp}\left(-\frac{L}{RT}\right)\\
\text{ln}(p)&= -\frac{L}{RT} + C
\end{align}

