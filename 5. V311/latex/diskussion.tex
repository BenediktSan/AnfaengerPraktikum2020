\section{Diskussion}
Dieser Versuch wurde nur Aufgrund von erhaltenen Messwerten ausgeführt. Die Diskussion der Durchführung entfällt deswegen größtenteils.
\subsection{Widerstandsmessung}
Bei diesem Teil des Versuches wurde aus den Abmessungen der Platten ihr Widerstand bestimmt. Die Literaturwerte für die spezifischen Widerstände \cite{Widerstand}
sind $\rho_\text{Kupfer}=\SI{0.018}{\micro\ohm\metre}$ und $\rho_\text{Kupfer}=\SI{0.06}{\micro\ohm\metre}$.\\
Verglichen mit den berechneten Werten $\rho_\text{berechnet$_\text{Kupfer}$} = \SI{0.0388+-0.0008}{\nano\ohm\meter}$ und 
$\rho_\text{berechnet$_\text{Zink}$} = \SI{0.429+-0.011}{\nano\ohm\meter}$. Damit erhalten wir eine Abweichung von $\approx \SI{46291.75}{\percent}$ für Kupfer und 
$\approx \SI{13886.01}{\percent}$ für Zink. Für Kupfer entspricht dies einer Abweichung um die Größenordung $\approx 10^3$ und für Zink $\approx 10^2$.\\
Wo her genau diese starke Abweichung stammt, lässt sich nicht ermitteln.\\
Allerdings liegt der, aus der abfallenden Spannung und dem Strom ermittelte, Widerstand für beide Werte in der selben Größenordnung neben dem Idealwert.
Es war aber nicht möglich aus den erhaltenen Werten, mittels des Ohm'schen Gesetzes, andere Werte zu erhalten.  
Bei der Interpretation der Werte für die mittlere Flugzeit, die Beweglichkeit und die mittlere freie Weglänge muss also mit beachtet werden, dass sie sich aus diesen Werten berechnen und damit 
eine sehr starke Abweichung besitzen müssen.

\subsection{Hall-Effekt}

\begin{table}[H]
    \centering
    %\caption{Die Messwerte}
    \begin{tabular}{ S [table-format=5.0] S [table-format=5.5] S [table-format=5.5] S [table-format=5.2] }
        \toprule
        {Messwert} & {Theoriewert Kupfer} & {Messwert Kupfer}\\
        \midrule
        \raggedleft$\text{Mittlere Flugzeit} \mathbin{\scalebox{1.5} /} \SI{1e-14}{\per\second}$ & 2.7 & 111960     \\
        \raggedleft$\text{Mittlere Weglänge} \mathbin{\scalebox{1.5} /} \SI{1e-7}{\milli\metre}$ &430 &25520000     \\
        \raggedleft$\text{Beweglichkeit}\mathbin{\scalebox{1.5} /} \SI{1e-6}{\cubic\metre\per\volt\per\second}$&  & 393.809 \\
        %\raggedleft$\text{Driftgeschwindigkeit}$ &  & &  \\
        \raggedleft$\text{Elektronenzahldichte} \mathbin{\scalebox{1.5} /} \SI{e22}{\per\cubic\centi\metre}$ & 8.45 &48910.300   \\
        \raggedleft$\text{Totalgeschwindigkeit} \mathbin{\scalebox{1.5} /} \SI{1e5}{\metre\per\second}$ & 1.57  &6.354   \\
        \raggedleft$\text{Fermienergie} \mathbin{\scalebox{1.5} /} \si{\electronvolt}$ & 7.00 &1.148   \\
        \bottomrule
    \end{tabular}
\caption{Vergleich der Theoriewerte für Kuper mit den Messwerten. Die Messwerte wurden dabei nur exemplarisch gewählt.}
\label{tab:theo1bar}
\end{table}

\begin{table}[H]
    \centering
    %\caption{Die Messwerte}
    \begin{tabular}{ S [table-format=5.0] S [table-format=5.5] S [table-format=5.5] S [table-format=5.2]}
        \toprule
        {Messwert}& {Theoriewert Zink}&  {Messwert Zink} \\
        \midrule
        \raggedleft$\text{Mittlere Flugzeit} \mathbin{\scalebox{1.5} /} \SI{1e-14}{\per\second}$ &  & 169 28    \\
        \raggedleft$\text{Mittlere Weglänge} \mathbin{\scalebox{1.5} /} \SI{1e-7}{\milli\metre}$ &  &3250000    \\
        \raggedleft$\text{Beweglichkeit}\mathbin{\scalebox{1.5} /} \SI{1e-6}{\cubic\metre\per\volt\per\second}$&  &59.541    \\
        %\raggedleft$\text{Driftgeschwindigkeit}$&  &   & \\
        \raggedleft$\text{Elektronenzahldichte} \mathbin{\scalebox{1.5} /} \SI{e22}{\per\cubic\centi\metre}$ &13.10  &24166.7    \\
        \raggedleft$\text{Totalgeschwindigkeit} \mathbin{\scalebox{1.5} /} \SI{1e5}{\metre\per\second}$&18.174  &3.772     \\
        \raggedleft$\text{Fermienergie} \mathbin{\scalebox{1.5} /} \si{\electronvolt}$ & 9.39 &0.543  \\
        \bottomrule
    \end{tabular}
\caption{Vergleich der Werte}
\label{tab:theo1bar}
\end{table}