\section{Zielsetzung}
In diesem Versuch sollen die mikroskopischen Parameter untersucht werden, 
die die Bewegung von Leitungselektronen in Metallen beschreiben.\\
Dafür werden der elektrische Widerstand und die Hall-Spannung bei unterschiedlischen Materialien 
untersucht und dann ein Zusammenhang zwischen diesen Größen und Parametern wie der mittleren Driftgeschwindigkeit in Stromrichtung.



\section{Theoretische Grundlagen}

\subsection{Bandstruktur und elektrische Leitfähigkeit von Kristallen}

In einem kristallinen Festkörper lassen sich die Energieniveaus seiner Elektronen und seine Leitfähigkeit durch
ein Modell mit Energiebändern beschreiben.\\
Die Valenzelektronen des Materials, also die Elektronen der äußersten Schale, 
bilden in einem kristallinen Festkörper ein zusammenhängendes System. Nach dem Pauli-Prinzip dürfen in einem System nur Elektronen mit entgegengesetztem Spin
den gleichen Zustand und damit gleiche Energie besitzen.\\
Die möglichen energetischen Zustände der Elektronen lassen sich dann, wie in Abb.\refeq{img:band} sichtbar,, 
durch quasikontinuierliche Energiebänder darstellen.\\
Die Lücken zwischen ihnen heißen "verbotene Zone" und bilden die nicht möglichen Zustände ab. Sie grenzen also einzelne Energiebänder ab, obwohl es auchmöglich ist, dass sich diese überlappen.\\
Mit diesem Modell der Energiebänder lassen sich nun sehr gut Vorgänge, die die Elektronen des Festkörpers betreffen, beschreiben.\\
In einem komplett mit Elektronen gefülltem Band, wie es oft bei den Bändern der inneren 
Schalen der Fall ist, kann, da jeder mögliche Zustand voll besetzt ist,
kein Elektron Energie aufnehmen oder abgeben. \\
Durch einen dieser Prozesse würde das Elektron nämlich einen Zustand einnehmen müssen, der schon besetzt ist, was das Pauli-Prinzip untersagt.
Daraus folgt dann auch dass die Elektronen dieser Bänder nicht, zum Beispiel durch äußere elektrische Felder, beschleunigt werden können 
und damit nicht zur elektrischen Leitfähigkeit beitragen können.\\
Die Elektronen die beschleunigt werden können sind die, die in dem äußersten, nicht gefülltem Band sitzen.
In diesen Bändern gibt es Quantenzustände welche noch unbestzt sind, zum Beispiel die von Elektronen mit umgekehrtem Spin.
Die Elektronen in diesen Bändern haben also die Möglichkeit beschleunigt zu werden, wodruch sich in dem kristallinen Festkörper
ein makroskopischer Strom entstehen kann.\\
Da dieser Strom durch Elektronen des nur teilweise besetzten Bandes hervorgerufen wird, werden sie auch Leitungselektronen und das Band das Leitfähigkeitsband genannt.\\\\

%HIER ABB MIT DEN VALENZBÄNDERN VON NATRIUM


\noindent
Ein Beispiel dafür wären natürlich in erster Linie Metalle, die mit ihrer hervorragenden Leitfähigkeit, die sich nach diesem Prinzip beschreiben lässt, glänzen.
Ein anderes Beispiel ist in Abb.\refeq{img:band} in der die Bänder eines Natrium-Atoms aufgeführt sind.\\
Die einzelnen sichtbaren Bänder sind dabei die einzelnen Schalen des Atoms.
Die Lücken zwischen ihnen sind dabei, die zuvor genannten verbotenen Zonen.
Die 1s-, 2s-, 2p-, Bänder sind komplett gefüllt und tragen nichts zur elektrischen Leitfähigkeit bei.
Das 3s-Band, welches mit der 3s-Schale korrespondiert, ist allerdings nicht komplett gefüllt.
In der Schale befindet sich ein ungepaartes Elektron. Es ist also noch Platz für eins mit entgegengesetztem Spin.
Dadurch trägt dieses zur Leitfähigkeit bei.\\\\
\noindent
Bei nichtleitenden Festkörpern, oder auch Isolatoren, ist die verbotene Zone so breit, dass die Energie, 
die zum überspringen von ihr nötig wäre, nicht aufgebracht werden kann. Des Weiteren besitzt die äußerste Schale kein Elektron, welches
überhaupt erst beschleunigt werden könnte.\\
Nach der Quantenmechanik sollten ideale Metallkristalle eine unendlich hohe Leitähigkeit besitzen, da
die Elektronen als Materiewellen betrachtet werden können, welche nicht in Wechselwirkung untereinander oder mit den Atomrümpfen treten.\\
Die endliche reale Leitfähigkeit lässt sich also auf Abweichungen vom Ideal und damit Kristallbaufehler begründen.\\


\subsection{Berechnung der Leitfähigkeit}

Die zuvor beschriebene Bewegung der Leitungselektronen durch das Kristallgitter lässt 
mit der Bewegung der Teilchen eines idealen Gases vergleichen.\\
Die Elektronen kollidieren mit Fehlstellungen im Gitter und auch mit Ionenrümpfen, die sich aus dem Gitterverband entfernt haben.
Sie werden, wenn ein elektrisches Feld anliegt, also nicht kontinuirlich beschleunigt sondern auch immer wieder in zufällige Richtungen gestreut.
Die Zeit zwischen solchen Zusammenstößen lässt sich über die mittlere Flugzeit $\overline{\tau}$ ausdrücken.\\
Während $\overline{\tau}$ wird, bei angelegtem Feld, ein Elektron also in Richtung $\vec{E}$ gleichmäßig beschleunigt.
Diese Beschleunigung lässt bei der Ladunng $\symup{e_0}$ und der Ruhemasse $\symup{m_0}$ durch $\vec{b}$ ausdrücken.

\begin{equation}
  \vec{b} = -\frac{\symup{e_0}}{\symup{m_0}} \cdot \vec{E} \nonumber
\end{equation}
\noindent
Die Geschwindigkeitsänderung $\increment\vec{\overline{v}}$ während $\overline{\tau}$ in Richtung $\vec{E}$ ergibt sich dann zu:

\begin{equation}
  \increment\vec{\overline{v}}= -\frac{\symup{e_0}}{\symup{m_0}}\cdot \vec{E} \cdot \overline{\tau} \nonumber
\end{equation}
\noindent
Da die Elektronen, wie zuvor bereits erwähnt, immer zufällig gestreut werden, ändert sich immer ihre Geschwindigkeitskomponenten in Richtung
des elektrischen Feldes. Diese ist nach der Streuung, im Durchschnitt, immer null.
Nach jedem Stoß müssen die Elektronen also immer komplett neu in Richtung des elektrischen Feldes beschleunigt werden.\\
Dieser Sachverhalt lässt sich dann beschreiben in dem eine mittlere Driftgeschwindigkeit eingeführt wird, welche das Mittel aus $\increment\vec{\overline{v}}$ 
und 0 ist. Die Forme dafür ergibt sich dann zu:

\begin{equation}
  \vec{\overline{v_d}}=\frac{1}{2}\increment\vec{\overline{v}} \nonumber
\end{equation}
\noindent
Mit dieser Gleichung lässt sich sich dann eine Formel für die Stromdichte $j$ herleiten. Diese ist nämlich die Anzahl $n$ der Ladungen, hier ausgedrückt
über die Elektronenladung $-\symup{e_0}$, die sich mit dem Betrag der Geschwindigkeit $\overline{v_d}$ bewegen, also:

\begin{equation}
  j=-n\overline{v_d}\cdot \symup{e_0} \nonumber
\end{equation}
\noindent
Oder anders gesagt; Der Strom $I$, der durch die Querschnittsfläche des Drahtes $Q$ fließt, also $\frac{I}{Q}$.
Einsetzen von $\overline{v_d}$  führt dann zu:

\begin{equation}
  j=n\frac{1}{2}\frac{\symup{e_0}^2}{\symup{m_0}}\cdot \vec{E} \cdot \overline{\tau} \nonumber
\end{equation}
\noindent
Unter der Annahme, dass wir uns in einem homogenen Leiter befinden, lässt sich $j=\frac{I}{Q}$ und für das elektrische Feld $E=\frac{U}{L}$, wie im homogenen Feld eines Plattenkondensators, annehmen.
Erneutes einsetzen führt dann zu einer dem Ohm'schen Gesetz sehr ähnlichen Gleichung:

\begin{equation}
  I=n\frac{1}{2}\frac{\symup{e_0}^2}{\symup{m_0}}\cdot \overline{\tau} \frac{Q}{L} \cdot U
  \label{eqn:ohm1}
\end{equation}
Wenn nun \refeq{eqn:ohm1} mit dem Ohm'schen Gesetz $U=R\cdot I$ verglichen wird, wird sichtbar, dass 
\begin{equation}
  \frac{1}{R}=n\frac{1}{2}\frac{\symup{e_0}^2}{\symup{m_0}}\cdot \overline{\tau} \frac{Q}{L} = S\nonumber
\end{equation}
gelten muss.\\
Dieser Term, das Reziprok des Widerstands, wird als elektrische Leitfähigkeit $S$ bezeichnet. Wie aus \refeq{eqn:ohm1} ersichtlich bildet
sie einen Proportionalitätsfaktor zwischen Spannung und Strom.\\
Um allgemeinere Größen zu erhalten werden nun die geometrieabhängigen Größen $Q$ und $L$ weggelassen.\\
Dadurch wird $S$ zur spezifischen Leitfähigkeit $\sigma$ und $R$ wieder zu dem dazu reziproken spezifischen Widerstand $\rho$.
\begin{align}
  \sigma&=n\frac{1}{2}\frac{\symup{e_0}^2}{\symup{m_0}} \cdot \overline{\tau} \label{eqn:sigma}\\
  \rho &=\frac{2}{n}\frac{\symup{m_0}}{\symup{e_0}^2 \cdot \overline{\tau}}  \label{eqn:rho}
\end{align}



\subsection{Hall-Effekt}

Für die weitere Bestimmung von $I$ in \refeq{eqn:ohm1} fehlen noch die Variablen $n$ und $\overline{\tau}$.\\
Die Anzahl der Teilchen $n$ lässt sich einfach mit Hilfe des Hall-Effektes bestimmen.












\subsection{Berechnung der mikroskopischen Leitfähigkeitsparameter R und $\text{U}_\text{H}$} 

\subsection{Elektrizitätsleitung mit positiven Ladungsträgern}





