\begin{table}[H]
    \centering
    \begin{tabular}{S [table-format=2.1] S [table-format=3.1]}
        \toprule
        {$U_\text{Amp} \mathbin{\scalebox{1.5} / }\si{\volt}$} & {$t_\text{Amp} \mathbin{\scalebox{1.5} / }\si{\micro\second}$}\\
        \midrule
        14.0 & 0.0 \\
        -11.5 & 14.0\\
        10.5 & 27.6\\
        -8.3 & 42.0\\
        7.8 & 54.00\\
        -6.0 & 68.0\\
        5.8 & 82.0\\
        -4.2 & 95.0\\
        4.1 & 108.0\\
        -3.0 & 122.0\\
        3.1 & 136.0\\
        -2.2 & 150.0\\
        2.5 & 163.0\\
        \bottomrule
    \end{tabular}
\caption{Die Messwerte der Amplitudenspannung mit ihren korrespondierenden Zeiten.}
\label{tab:Uamp}
\end{table}


\begin{table}[H]
    \centering
    \begin{tabular}{S [table-format=1.2] S [table-format=2.0]}
        \toprule
        {$\frac{U}{\symup{U_0}}$} & {$f \mathbin{\scalebox{1.5} / } \si{\kilo\hertz}$}\\
        \midrule
        1.08 & 5\\
        1.24 & 10\\
        1.44 & 15\\
        2.16 & 20\\
        2.40 & 21\\
        3.00 & 23\\
        3.40 & 24\\
        3.72 & 25\\
        3.80 & 26\\
        3.64 & 27\\
        3.12 & 28\\
        2.44 & 30\\
        1.60 & 33\\
        1.24 & 35\\
        0.84 & 40\\
        0.60 & 45\\
        0.44 & 50\\
        \bottomrule
    \end{tabular}
\caption{Die Messwerte Spannung im Verhältnis zur angelegten Spannung und die dazugehörigen Frequenzen.}
\label{tab:Uu0}
\end{table}

\begin{table}[H]
    \centering
    \begin{tabular}{S [table-format=2.1] S [table-format=2.1] S [table-format=1.3]}
        \toprule
        {$f \mathbin{\scalebox{1.5} / } \si{\kilo\hertz}$} & {$\increment t \mathbin{\scalebox{1.5} / } \si{\micro\second}$} & {$\phi \mathbin{\scalebox{1.5} / } \si{\radian}$}\\
        \midrule
        5.0& 0.0 & 0.000  \\
        10.0 & 1.0 & 0.063 \\
        12.0 & 1.5 & 0.113 \\
        14.0 & 1.5 & 0.132 \\
        20.0 & 0.0 & 0.000 \\
        25.0 & 0.5 & 0.079 \\
        30.0 & 1.0 & 0.188 \\
        35.0 & 1.5 & 0.330 \\
        35.7 & 3.0 & 0.674 \\
        36.2 & 4.0 & 0.911 \\
        36.7 & 6.0 & 1.385 \\
        37.5 & 8.0 & 1.885 \\
        38.0 & 10.0 & 2.388\\
        38.2 & 11.0 & 2.644\\
        38.7 & 11.0 & 2.678\\
        40.0 & 11.0 & 2.765\\
        42.5 & 11.5 & 3.071\\
        45.0 & 11.0 & 3.110\\
        \bottomrule
    \end{tabular}
\caption{Die Messwerte der Phasenverschiebung zwischen Kondensator- und Erregerspannung bei unterschiedlichen Frequenzen. $\increment t$ ist dabei die Zeitdifferenz zwishen zwei Amplituden und $\phi$ diese umgewandelt in einen Winkel.}
\label{tab:phi}
\end{table}

\begin{table}[H]
    \centering
    \sisetup{table-format=1.3}
    \begin{tabular}{ S | S [table-format=2.5] @{$ \pm{}$} S [table-format=4.5]  }
        \toprule
        {Parameter} & \multicolumn{2}{c}{ Bestimmte Werte} \\
        \midrule
        \text{A}	& \num{-8.3}\SI{1e-8}{}& \num{0.5}\SI{1e-8}{} \\
        \text{B}	&\num{0.00432}  & \num{0.00028}  \\
        \text{C}	&\num{-52}  & \num{4}  \\
        \bottomrule
    \end{tabular}
\caption {Berechnete Werte für die quadratische Fit-Funktion gerundet auf die fünfte Nachkommastelle.}
\label{tab:signum}
\end{table}



\begin{table}[H]
    \centering
    \sisetup{table-format=1.3}
    \begin{tabular}{ S | S [table-format=5.5] @{$ \pm{}$} S [table-format=2.5] S }
        \toprule
        {Parameter} & \multicolumn{3}{c}{ Bestimmte Werte} \\
        \midrule
        \text{A}	&\num{2.94167}  & \num{0.08634} & \; \si{\radian}\\
        \text{B}	&\num{36988.06355}  & \num{84.81132} & \\
        \text{C}	&\num{0.08872}  & \num{0.05578} & \; \si{\radian}\\
        \text{D}	&\num{838.78531}  & \num{67.35911} & \\
        \bottomrule
    \end{tabular}
\caption {Berechnete Werte für die Signums-Funktion gerundet auf die fünfte Nachkommastelle.}
\label{tab:signum}
\end{table}




\begin{table}[H]
    \centering
    \sisetup{table-format=1.3}
    \begin{tabular}{ S | S [table-format=2.2] @{$ \quad \pm{}$} S [table-format=1.2] S [table-format=2.2]  S [table-format=2.2] @{$ \quad \pm{}$} S [table-format=1.2] }
        \toprule
        \multicolumn{1}{c|}{Frequenz} & \multicolumn{2}{p{4cm}}{\centering Theoretischer Wert \\$ \si{\kilo\hertz}$} & 
        \multicolumn{1}{p{4cm}}{\centering Experimenteller Wert\\ $ \si{\kilo\hertz}$} & 
        \multicolumn{2}{p{4cm}}{\centering Abweichung von der Theorie \\ $\si{\percent}$} \\
        \midrule \cmidrule(lr){2-3}\cmidrule(lr){5-6}
        $w_\text{res}$   &\num{16.98} & \num{0.28}       &\num{3.70}       &\num{78.20} &\num{0.35}\\
        $w_\text{1}	$    &\num{17.34} & \num{0.28}       &\num{3.60}       &\num{79.23}&\num{0.33}\\
        $w_\text{2}$	 &\num{16.64} & \num{0.27}       &\num{3.80}       &\num{77.2}&\num{0.40}\\
        \bottomrule
    \end{tabular}
\caption {Vergleich der charakteristischen Frequenzen. \newline Dabei ist $w_\text{res}$ die für den Wert $\frac{\pi}{2}$, 
$w_\text{1} $ für $\frac{\pi}{4}$ und $w_\text{2} $ für $\frac{3 \pi}{4}$.}
\label{tab:omegas}
\end{table}