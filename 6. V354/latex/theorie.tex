\section{Zielsetzung}
    In diesem Versuch sollen verschieden Eigenschaften eines LRC-Kreises untersucht werden. Diese sind der Dämpfungswiderstand aus der 
    Zeitabhängigen Amplitude und der Widerstand bei dem der Aperiodische Grenzfall eintritt. Zusätzlich wird an einem 
    Serienresonanzkeisnoch die Frequenzabhängigkeit der Kondensatorspannung und die Frequenzabhängigkeit der Phase zwischen Erreger- und 
    Kondensatorspannung untersucht.

\section{Theoretische Grundlagen}

    \subsection{Herleitung gedämpfter Schwingkreis}

    \begin{equation}
    U_{\symup{R}} + U_{\symup{C}} + U_{\symup{L}} +  = 0     
    \end{equation}

    \begin{align}
        U_{\symup{R}} & = RI \\
        U_{\symup{C}} & = \frac{Q}{C}\\
        U_{\symup{L}} & = L \cdot \symup{\frac{d}{dt}}I
    \end{align}

    \begin{align}
        L  \symup{\frac{d}{dt}}I + RI + \frac{Q}{C} & = 0\\
        \symup{\frac{d^2}{dt^2}}I + \frac{R}{L} \symup{\frac{d}{dt}}I + \frac{1}{LC}I & = 0
    \end{align}
    
    \begin{equation}
        I(t) = Z \cdot \symup{e}^{j \tilde{w} t}
    \end{equation}

    \begin{equation}
        \tilde{w}^2 - j \frac{R}{L}\tilde{w} - \frac{1}{LC} = 0
    \end{equation}

    \begin{equation}
        \tilde{w}_{1,2} = j \frac{R}{2L} \pm \sqrt{\frac{1}{LC}-\frac{R^2}{4L}}
    \end{equation}
    
    \begin{equation}
        I(t) = Z_1 \cdot \symup{e}^{j \tilde{w}_1t} + Z_2 \cdot \symup{e}^{j\tilde{w}_2 t}
    \end{equation}
    mit 
    \begin{align}
        2 \pi \mu & \coloneq \frac{R}{2L} \\ 
        2 \pi \tilde{\nu} & \coloneq \sqrt{\frac{1}{LC} - \frac{R^2}{4L^2}}
    \end{align}

    \begin{equation}
        I(t) = \symup{e}^{- 2 \pi \mu t} \cdot  (Z_1 \symup{e}^{j 2\pi \tilde{v}t} + Z_2 \symup{e}^{-j 2\pi \tilde{v} t})
    \end{equation}

    \subsection{Fallunterscheidung}

        \subsubsection{1.Fall}

        \begin{equation}
            \frac{1}{LC} > \frac{R^2}{4L^2} 
        \end{equation}
        
        $v \in \mathds{R}$
       
        für $Z_1 = \overline{Z_2}$ ist $I(t) \in \mathds{R}$

        \begin{align}
            Z_1 & = \frac{1}{2} A_0 \symup{e}^{j \eta }\\
            Z_2 & = \frac{1}{2} A_0 \symup{e}^{-j \eta}
        \end{align}

        aus

        \begin{equation}
            \frac{\symup{e}^{j \phi} + \symup{e}^{-j \phi}}{2} = cos(\phi)
        \end{equation}

        ergibt sich:

        \begin{equation}
            I(t) = A_0 \symup{e}^{- 2 \pi \mu t} \cdot cos(2\pi \nu t + \eta)
        \end{equation}

        \begin{equation}
            T = \frac{1}{\nu} = \frac{2 \pi}{\sqrt{\frac{1}{LC}-\frac{R^2}{4L^2}}}
        \end{equation}

        mit 
        \begin{equation}
            \frac{1}{LC} > \frac{R^2}{4L^2}
        \end{equation}

        ist
        \begin{equation}
            T_0 = \frac{2\pi}{\omega_0} = 2\pi \sqrt{LC}
        \end{equation}

        \begin{equation}
            T_{\text{exp}} = \frac{1}{2 \pi \mu } = \frac{2L}{R}
        \end{equation}

        \subsubsection{2. Fall}

        \begin{equation}
            \frac{1}{LC} < \frac{R^2}{4L^2}
        \end{equation}
        $\tilde{v} \in \mathds{C}$

        \begin{equation}
            I \sim e^{- \left( \frac{R}{2L} - \sqrt{\frac{R^2}{4L^2} - \frac{1}{LC}}\right) t}
        \end{equation}

        Spefzialfall für:
        \begin{equation}
            \frac{1}{LC} = \frac{R^2_{\text{ap}}}{4L^2}
        \end{equation}

        $\nu = 0$

        \begin{equation}
            I(t) = A \symup{e}^{-\frac{R}{2L}t} = A \symup{e}^{-\frac{t}{LC}}
        \end{equation}

        \subsection{vergleich Mechanik}

        \begin{equation}
            m \ddot{x} + s \dot{x} + Dx = 0
        \end{equation}

        \begin{align}
            m \Leftrightarrow L \\
            s \Leftrightarrow R \\
            D \Leftrightarrow \frac{1}{C}
        \end{align}

    \subsection{erzwungene Schwingungen}

    \begin{equation}
        u(t)= \symup{U_0} \symup{e}^{j \omega t}
    \end{equation}

    \begin{equation}
        L \dot{I} + R I + \frac{Q(t)}{C} = \symup{U_0} \symup{e}^{j \omega t}
    \end{equation}

    oder mit:
    \begin{equation}
        V_{\symup{C}} (t) = \frac{Q(t)}{C}
    \end{equation}

    ergibt sich:
    \begin{equation}
        LC \ddot{V}_{\symup{C}} + RC \dot{V}_{\symup{C}} + V_{\symup{C}} = \symup{U_0} \symup{e}^{j \omega t}
    \end{equation}

    mit $A \in  \mathds{C}$ der Amplitude der Kodensatorspannung:
    \begin{equation}
        u_{\symup{C}} (t) = A(\omega) \symup{e}^{j \omega t}
    \end{equation}

    \begin{equation}
        -LC \omega^2 A + j\omega RC A +A = \symup{U_0}
    \end{equation}

    \begin{equation}
        A= \frac{\symup{U_0}}{1 - LC \omega^2 + j \omega RC} = \frac{ \symup{U_0}(1 -LC \omega^2 - j\omega RC)}{\left( 1- LC \omega^2 \right)^2}
    \end{equation}
    