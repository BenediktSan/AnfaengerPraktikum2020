\section{Zielsetzung}
    In diesem Versuch sollen verschieden Eigenschaften eines LRC-Kreises untersucht werden. Diese sind der Dämpfungswiderstand aus der 
    Zeitabhängigen Amplitude und der Widerstand bei dem der Aperiodische Grenzfall eintritt. Zusätzlich wird an einem 
    Serienresonanzkeisnoch die Frequenzabhängigkeit der Kondensatorspannung und die Frequenzabhängigkeit der Phase zwischen Erreger- und 
    Kondensatorspannung untersucht.

\section{Theoretische Grundlagen}

    \subsection{Herleitung gedämpfter Schwingkreis}

    Das 2. Kirchhoffsche Gesetz führt direkt zu seiner Beziehung der abfallenden Spannungen:
    \begin{equation}
    U_{\symup{R}} + U_{\symup{C}} + U_{\symup{L}} +  = 0     
    \label{eq:Uabfall}
    \end{equation}

    \noindent In die Formel \ref{eq:Uabfall} lassen sich nun die Schaltelement spezifischen Formeln  einsetzten
    \begin{align}
        U_{\symup{R}} & = RI \nonumber\\
        U_{\symup{C}} & = \frac{Q}{C}\nonumber\\
        U_{\symup{L}} & = L \cdot \symup{\frac{d}{dt}}I \nonumber
    \end{align}

    \noindent Dies führt zu einer Differentialgleichung der Form:
    \begin{align}
        L  \symup{\frac{d}{dt}}I + RI + \frac{Q}{C} & = 0 \nonumber\\
        \symup{\frac{d^2}{dt^2}}I + \frac{R}{L} \symup{\frac{d}{dt}}I + \frac{1}{LC}I & = 0 
        \label{eq:DGL1}
    \end{align}
    
    \noindent Diese Differetialgleichung wird über den Ansatz einer komplexen e Funktion gelöst.
    \begin{equation}
        I(t) = Z \cdot \symup{e}^{j \tilde{w} t}
        \label{eq:eAn}
    \end{equation}

    \noindent Einsetzen von \ref{eq:eAn} in die DGL\ref{eq:DGL1} ergibt:
    \begin{equation}
        \tilde{\omega}^2 - j \frac{R}{L}\tilde{\omega} - \frac{1}{LC} = 0 \nonumber
    \end{equation}

    \noindent Umstellen nach $\tilde{\omega}$ ergibt:
    \begin{equation}
        \tilde{\omega}_{1,2} = j \frac{R}{2L} \pm \sqrt{\frac{1}{LC}-\frac{R^2}{4L}} \nonumber
    \end{equation}
    
    \noindent $I(t)$ lässt sich nun also durch folgende Gleichung ausdrücken:
    \begin{equation}
        I(t) = Z_1 \cdot \symup{e}^{j \tilde{w}_1t} + Z_2 \cdot \symup{e}^{j\tilde{w}_2 t} \nonumber
    \end{equation}
    
    \noindent Die Definitionen für $\mu$ und $\nu$ bieten sich an um die Formel übersichtlicher zu gestalten.
    \begin{align}
        2 \pi \mu & \coloneq \frac{R}{2L} \nonumber\\ 
        2 \pi \tilde{\nu} & \coloneq \sqrt{\frac{1}{LC} - \frac{R^2}{4L^2}} \nonumber
    \end{align}

    \noindent $I(t)$ lässt sich dann als:
    \begin{equation}
        I(t) = \symup{e}^{- 2 \pi \mu t} \cdot  (Z_1 \symup{e}^{j 2\pi \tilde{v}t} + Z_2 \symup{e}^{-j 2\pi \tilde{v} t}) \nonumber
    \end{equation}
    \noindent schreiben. Um $I(t)$ weiter zu unterschen muss eine Fallunterscheidung gemacht werden. Hier wird unterschieden ob der 
    Inhalt der Wurzel positiv oder negativ definiert ist und somit der Exponent reel oder imaginär ist. 
    \subsection{Fallunterscheidung}

        \subsubsection{1.Fall}
        Zu erst wird der Fall
        \begin{equation}
            \frac{1}{LC} > \frac{R^2}{4L^2}  \nonumber
        \end{equation}
        untersucht. Hier ist $v \in \mathds{R}$. Damit die Lösungsfunktion $I(t) \in \mathds{R}$ ist muss
        $Z_1 = \overline{Z_2}$ sein. Dies lässt sich durh den Ansatz
        \begin{align}
            Z_1 & = \frac{1}{2} A_0 \symup{e}^{j \eta } \nonumber\\
            Z_2 & = \frac{1}{2} A_0 \symup{e}^{-j \eta} \nonumber
        \end{align}
        erreichen.
        
        Die folgende Eigendschaft komplexer e-Funktionen;
        \begin{equation}
            \frac{\symup{e}^{j \phi} + \symup{e}^{-j \phi}}{2} = cos(\phi) \nonumber
        \end{equation}
        wird nun auf das Problem angewandt.

        Es ergibt sich:
        \begin{equation}
            I(t) = A_0 \symup{e}^{- 2 \pi \mu t} \cdot cos(2\pi \nu t + \eta) \nonumber
        \end{equation}

        Aus diesem Term kann nun eine Beziehung für die Periodendauer bestimmt werden:
        \begin{equation}
            T = \frac{1}{\nu} = \frac{2 \pi}{\sqrt{\frac{1}{LC}-\frac{R^2}{4L^2}}} \nonumber
        \end{equation}

        mit der Näherung
        \begin{equation}
            \frac{1}{LC} > \frac{R^2}{4L^2} \nonumber
        \end{equation}

        entsteht die Thomsonsche Schwingungsformel
        \begin{equation}
            T_0 = \frac{2\pi}{\omega_0} = 2\pi \sqrt{LC} \nonumber
        \end{equation}

        die Zeit $T_{\text{exp}}$, beschreibt den Zeitraum in dem die Amplitude auf den e-ten Teil abfällt
        \begin{equation}
            T_{\text{exp}} = \frac{1}{2 \pi \mu } = \frac{2L}{R} \nonumber
        \end{equation}

        \subsubsection{2. Fall}

        \begin{equation}
            \frac{1}{LC} < \frac{R^2}{4L^2} \nonumber
        \end{equation}
        $\tilde{v} \in \mathds{C}$

        \begin{equation}
            I \sim \symup{e}^{- \left( \frac{R}{2L} - \sqrt{\frac{R^2}{4L^2} - \frac{1}{LC}}\right) t} \nonumber
        \end{equation}

        Spefzialfall für:
        \begin{equation}
            \frac{1}{LC} = \frac{R^2_{\text{ap}}}{4L^2} \nonumber
        \end{equation}

        $\nu = 0$

        \begin{equation}
            I(t) = A \symup{e}^{-\frac{R}{2L}t} = A \symup{e}^{-\frac{t}{LC}} \nonumber
        \end{equation}

        \subsection{vergleich Mechanik}

        \begin{equation}
            m \ddot{x} + s \dot{x} + Dx = 0 \nonumber
        \end{equation}

        \begin{align}
            m \Leftrightarrow L  \nonumber\\
            s \Leftrightarrow R  \nonumber\\
            D \Leftrightarrow \frac{1}{C} \nonumber
        \end{align}

    \subsection{erzwungene Schwingungen}

    \begin{equation}
        u(t)= \symup{U_0} \symup{e}^{j \omega t} \nonumber
    \end{equation}

    \begin{equation}
        L \dot{I} + R I + \frac{Q(t)}{C} = \symup{U_0} \symup{e}^{j \omega t} \nonumber
    \end{equation}

    oder mit:
    \begin{equation}
        V_{\symup{C}} (t) = \frac{Q(t)}{C} \nonumber
    \end{equation}

    ergibt sich:
    \begin{equation}
        LC \ddot{V}_{\symup{C}} + RC \dot{V}_{\symup{C}} + V_{\symup{C}} = \symup{U_0} \symup{e}^{j \omega t} \nonumber
    \end{equation}

    mit $A \in  \mathds{C}$ der Amplitude der Kodensatorspannung:
    \begin{equation}
        u_{\symup{C}} (t) = A(\omega) \symup{e}^{j \omega t} \nonumber
    \end{equation}

    \begin{equation}
        -LC \omega^2 A + j\omega RC A +A = \symup{U_0} \nonumber
    \end{equation}

    \begin{equation}
        A= \frac{\symup{U_0}}{1 - LC \omega^2 + j \omega RC} = \frac{ \symup{U_0}(1 -LC \omega^2 - j\omega RC)}{\left( 1- LC \omega^2 \right)^2} \nonumber
    \end{equation}
    
    \begin{equation}
        |A|= \symup{U_0} \sqrt{\frac{\left(1-LC\omega^2\right)^2 + \omega^2 R^2C^2}{\left( \left(1 -LC \omega^2\right)^2 +w^2R^2C^2 \right)^2}} \nonumber
    \end{equation}

    \begin{equation}
        \text{tan}(\phi (\omega)) = \frac{\text{Im}(A)}{\text{Re}(A)} = \frac{\omega RC}{1 - LC\omega^2} \nonumber
    \end{equation}

    \begin{equation}
        \phi (\omega) = \text{arctan}\left(\frac{\omega RC}{1 - LC\omega^2} \right) \nonumber
    \end{equation}

    \begin{equation}
        U_{\symup{C}}(\omega) = \frac{U_0}{\sqrt{\left(1-LC\omega^2)\right)^2 + \omega^2 R^2 C^2}} \nonumber
    \end{equation}

    \begin{equation}
        \omega_{\text{res}} = \sqrt{\frac{1}{LC}-\frac{R^2}{2L^2}} \nonumber
    \end{equation}

    für: 
    \begin{equation}
        \frac{R^2}{2L^2} \ll \frac{1}{LC} \nonumber
    \end{equation}

    \begin{equation}
        U_{\text{C,max}} = \frac{1}{\omega_0 RC} = \frac{1}{R} \sqrt{\frac{L}{C}} U_0 \nonumber
    \end{equation}

    \begin{equation}
        \frac{U_0}{\sqrt{2}} \frac{1}{\omega_0 RC} = \frac{U_0}{C \sqrt{\omega^2_{\pm} R^2 + \left( \omega^2_{\pm}L - \frac{1}{C} \right) }} \nonumber
    \end{equation}

    \begin{equation}
        \omega_+ - \omega_- \approx \frac{R}{L} \nonumber
    \end{equation}

    \begin{equation}
        q = \frac{\omega_0}{\omega_+ - \omega_-} \nonumber
    \end{equation}

    \begin{equation}
        \frac{R^2}{2L^2} \gg \frac{1}{LC} \nonumber
    \end{equation}

    \begin{equation}
        \omega^2_0 = \frac{1}{LC} \nonumber
    \end{equation}

    \begin{equation}
        \omega_{1,2} = \pm  \frac{R}{2L} + \sqrt{\frac{R^2}{4L^2}+ \frac{1}{LC}} \nonumber
    \end{equation}

    \begin{equation}
        \omega_1 - \omega_2 = \frac{R}{L} \nonumber
    \end{equation}

    \begin{equation}
        r = X + jY \nonumber
    \end{equation}

    \begin{equation}
        |r| = \sqrt{X^2 + Y^2} \nonumber
    \end{equation}

    \begin{align}
        r_{\symup{C}} = -j\frac{1}{\omega C} \nonumber\\
        r_{\symup{L}} = j\omega L  \nonumber\\
        r_{\symup{R}} = R_{\symup{s}} \nonumber
    \end{align}

    \begin{equation}
        r_{\symup{S}} = R_{\symup{S}} + j \left(    \omega L - \frac{1}{\omega C} \right) \nonumber
    \end{equation}

    \begin{align}
        X_{\symup{S}} = R_{\symup{S}}  \nonumber\\ 
        Y_{\symup{S}} = j \omega L - \frac{1}{\omega C} \nonumber
    \end{align}

    \begin{equation}
        |r_{\symup{S}}| = \sqrt{R^2_{\symup{S}} + \left( \omega L - \frac{1}{\omega C} \right)^2} \nonumber
    \end{equation}

    \begin{equation}
        r_{\symup{p}} = \frac{\frac{1}{R_{\symup{p}}}+j \left( \frac{1}{\omega L} - \omega C \right)}{\frac{1}{R^2_{\symup{p}}}+ \left( \frac{1}{\omega L} - \omega C \right)^2} \nonumber
    \end{equation}

    \begin{equation}
        |r_{\symup{p}}|=  \frac{1}{\sqrt{\frac{1}{R^2_{\symup{p}}} \left( \frac{1}{\omega L} - \omega C \right)^2}} \nonumber
    \end{equation}
