\section{Zielsetzung}
    In diesem Versuch sollen verschieden Eigenschaften eines LRC-Kreises untersucht werden. Diese sind der Dämpfungswiderstand aus der 
    Zeitabhängigen Amplitude und der Widerstand bei dem der Aperiodische Grenzfall eintritt. Zusätzlich wird an einem 
    Serienresonanzkeisnoch die Frequenzabhängigkeit der Kondensatorspannung und die Frequenzabhängigkeit der Phase zwischen Erreger- und 
    Kondensatorspannung untersucht.

\section{Theoretische Grundlagen}

    \subsection{Herleitung gedämpfter Schwingkreis}

    Das 2. Kirchhoffsche Gesetz führt direkt zu seiner Beziehung der abfallenden Spannungen:
    \begin{equation}
    U_{\symup{R}} + U_{\symup{C}} + U_{\symup{L}} +  = 0     
    \label{eq:Uabfall}
    \end{equation}

    \noindent In die Formel \ref{eq:Uabfall} lassen sich nun die Schaltelement spezifischen Formeln  einsetzten
    \begin{align}
        U_{\symup{R}} & = RI \nonumber\\
        U_{\symup{C}} & = \frac{Q}{C}\nonumber\\
        U_{\symup{L}} & = L \cdot \symup{\frac{d}{dt}}I \nonumber
    \end{align}

    \noindent Dies führt zu einer Differentialgleichung der Form:
    \begin{align}
        L  \symup{\frac{d}{dt}}I + RI + \frac{Q}{C} & = 0 \nonumber\\
        \symup{\frac{d^2}{dt^2}}I + \frac{R}{L} \symup{\frac{d}{dt}}I + \frac{1}{LC}I & = 0 
        \label{eq:DGL1}
    \end{align}
    
    \noindent Diese Differetialgleichung wird über den Ansatz einer komplexen e Funktion gelöst.
    \begin{equation}
        I(t) = Z \cdot \symup{e}^{j \tilde{w} t}
        \label{eq:eAn}
    \end{equation}

    \noindent Einsetzen von \ref{eq:eAn} in die DGL\ref{eq:DGL1} ergibt:
    \begin{equation}
        \tilde{\omega}^2 - j \frac{R}{L}\tilde{\omega} - \frac{1}{LC} = 0 \nonumber
    \end{equation}

    \noindent Umstellen nach $\tilde{\omega}$ ergibt:
    \begin{equation}
        \tilde{\omega}_{1,2} = j \frac{R}{2L} \pm \sqrt{\frac{1}{LC}-\frac{R^2}{4L}} \nonumber
    \end{equation}
    
    \noindent $I(t)$ lässt sich nun also durch folgende Gleichung ausdrücken:
    \begin{equation}
        I(t) = Z_1 \cdot \symup{e}^{j \tilde{w}_1t} + Z_2 \cdot \symup{e}^{j\tilde{w}_2 t} \nonumber
    \end{equation}
    
    \noindent Die Definitionen für $\mu$ und $\nu$ bieten sich an um die Formel übersichtlicher zu gestalten.
    \begin{align}
        2 \pi \mu & \coloneq \frac{R}{2L} \nonumber\\ 
        2 \pi \tilde{\nu} & \coloneq \sqrt{\frac{1}{LC} - \frac{R^2}{4L^2}} \nonumber
    \end{align}

    \noindent $I(t)$ lässt sich dann als:
    \begin{equation}
        I(t) = \symup{e}^{- 2 \pi \mu t} \cdot  (Z_1 \symup{e}^{j 2\pi \tilde{v}t} + Z_2 \symup{e}^{-j 2\pi \tilde{v} t}) \nonumber
    \end{equation}
    \noindent schreiben. Um $I(t)$ weiter zu unterschen muss eine Fallunterscheidung gemacht werden. Hier wird unterschieden ob der 
    Inhalt der Wurzel positiv oder negativ definiert ist und somit der Exponent reel oder imaginär ist. 
    \subsection{Fallunterscheidung}

        \subsubsection{1.Fall}
        Zu erst wird der Fall
        \begin{equation}
            \frac{1}{LC} > \frac{R^2}{4L^2}  \nonumber
        \end{equation}
        untersucht. Hier ist $v \in \mathds{R}$. Damit die Lösungsfunktion $I(t) \in \mathds{R}$ ist muss
        $Z_1 = \overline{Z_2}$ sein. Dies lässt sich durch den Ansatz
        \begin{align}
            Z_1 & = \frac{1}{2} A_0 \symup{e}^{j \eta } \nonumber\\
            Z_2 & = \frac{1}{2} A_0 \symup{e}^{-j \eta} \nonumber
        \end{align}
        erreichen.
        
        Die folgende Eigendschaft komplexer e-Funktionen;
        \begin{equation}
            \frac{\symup{e}^{j \phi} + \symup{e}^{-j \phi}}{2} = cos(\phi) \nonumber
        \end{equation}
        wird nun auf das Problem angewandt.

        Es ergibt sich:
        \begin{equation}
            I(t) = A_0 \symup{e}^{- 2 \pi \mu t} \cdot cos(2\pi \nu t + \eta) \nonumber
        \end{equation}

        Aus diesem Term kann nun eine Beziehung für die Periodendauer bestimmt werden:
        \begin{equation}
            T = \frac{1}{\nu} = \frac{2 \pi}{\sqrt{\frac{1}{LC}-\frac{R^2}{4L^2}}} \nonumber
        \end{equation}

        mit der Näherung
        \begin{equation}
            \frac{1}{LC} > \frac{R^2}{4L^2} \nonumber
        \end{equation}

        entsteht die Thomsonsche Schwingungsformel
        \begin{equation}
            T_0 = \frac{2\pi}{\omega_0} = 2\pi \sqrt{LC} \nonumber
        \end{equation}

        die Zeit $T_{\text{exp}}$, beschreibt den Zeitraum in dem die Amplitude auf den e-ten Teil abfällt
        \begin{equation}
            T_{\text{exp}} = \frac{1}{2 \pi \mu } = \frac{2L}{R} \nonumber
        \end{equation}

        \subsubsection{2. Fall}

        Der zweite Fall beschreibt:
        \begin{equation}
            \frac{1}{LC} < \frac{R^2}{4L^2} \nonumber
        \end{equation}
        Hier ist dann $\tilde{v} \in \mathds{C}$.
        Für große Zeiten ergibt sich die Proportionalität:
        \begin{equation}   
            I \sim \symup{e}^{- \left( \frac{R}{2L} - \sqrt{\frac{R^2}{4L^2} - \frac{1}{LC}}\right) t} \nonumber
        \end{equation}

        Ein Spefzialfall für diesen Fall ist:
        \begin{equation}
            \frac{1}{LC} = \frac{R^2_{\text{ap}}}{4L^2} \nonumber
        \end{equation}
        
        Dadurch ist dann $\nu = 0$ und somit
        \begin{equation}
            I(t) = A \symup{e}^{-\frac{R}{2L}t} = A \symup{e}^{-\frac{t}{LC}} \nonumber
        \end{equation}

        \subsection{Vergleich Mechanik}

       % Hier bietet sich nun ein Vergleich mit einer rein mechanischen Schwingung an.
       % 
       % \begin{equation}
       %     m \ddot{x} + s \dot{x} + Dx = 0 \nonumber
       % \end{equation}
       %
       % \begin{align}
       %     m \Leftrightarrow L  \nonumber\\
       %     s \Leftrightarrow R  \nonumber\\
       %     D \Leftrightarrow \frac{1}{C} \nonumber
       % \end{align}

    \subsection{erzwungene Schwingungen}
    
    In diesem Abschnitt werden LRC-Schwingkreise untersucht die von außen angeregt sind. Dazu wird eine Spannungsquelle an den Stromkreis 
    angeschlossen die eine Sinusfoermige Spannung liefert.
    \begin{equation}
        u(t)= \symup{U_0} \symup{e}^{j \omega t} \nonumber
    \end{equation}
    $u(t)$ ist hier die anregende Spannung.

    Die Differentialgleichung erhält nun also die folgende Form:
    \begin{equation}
        L \dot{I} + R I + \frac{Q(t)}{C} = \symup{U_0} \symup{e}^{j \omega t} \nonumber
    \end{equation}

    oder mit:
    \begin{equation}
        V_{\symup{C}} (t) = \frac{Q(t)}{C} \nonumber
    \end{equation}

    ergibt sich:
    \begin{equation}
        LC \ddot{V}_{\symup{C}} + RC \dot{V}_{\symup{C}} + V_{\symup{C}} = \symup{U_0} \symup{e}^{j \omega t} 
        \label{eq:angDGL}
    \end{equation}

    mit $A \in  \mathds{C}$ der Amplitude der Kodensatorspannung:
    \begin{equation}
        u_{\symup{C}} (t) = A(\omega) \symup{e}^{j \omega t} 
        \label{eq:uteAn}
    \end{equation}

    Einsetzen von \ref{eq:uteAn} in \ref{eq:angDGL} liefert:
    \begin{equation}
        -LC \omega^2 A + j\omega RC A +A = \symup{U_0} \nonumber
    \end{equation}

    Die Formal nach $A$ umgestellt ergibt:
    \begin{equation}
        A= \frac{\symup{U_0}}{1 - LC \omega^2 + j \omega RC} = \frac{ \symup{U_0}(1 -LC \omega^2 - j\omega RC)}{\left( 1- LC \omega^2 \right)^2} \nonumber
    \end{equation}
    
    oder
    \begin{equation}
        |A|= \symup{U_0} \sqrt{\frac{\left(1-LC\omega^2\right)^2 + \omega^2 R^2C^2}{\left( \left(1 -LC \omega^2\right)^2 +w^2R^2C^2 \right)^2}}
        \label{eq:betA}
    \end{equation}

    Die Phase berechnet sich zu:
    \begin{equation}
        \text{tan}(\phi (\omega)) = \frac{\text{Im}(A)}{\text{Re}(A)} = \frac{\omega RC}{1 - LC\omega^2} \nonumber
    \end{equation}

    bzw.
    \begin{equation}
        \phi (\omega) = \text{arctan}\left(\frac{\omega RC}{1 - LC\omega^2} \right)
        \label{eq:phi}
    \end{equation}

    Da der Betrag der gesuchten Lösungsfunktion $u_{\symup{c}}(t)$ nach \ref{eq:uteAn} gleich dem Betrag von A ist, entsteht aus 
    \ref{eq:betA} das Ergebnis:
    \begin{equation}
        U_{\symup{C}}(\omega) = \frac{U_0}{\sqrt{\left(1-LC\omega^2)\right)^2 + \omega^2 R^2 C^2}} 
        \label{eq:Ucw}
    \end{equation}

    Somit ist die gesucht Funktion für $U_{\symup{C}}$ in Abhängigkeit von $\omega$, der Frequenz der Erregerspannung gefunden. Durch 
    weiters Untersuchen von \ref{eq:Ucw} ergibt sich, dass $U_{\symup{C}}$ für $\omega \to \infty $ gegen 0 und für $\omega \to 0$ gegen 
    das $U_0$ der Erregerspanung geht. Jedoch gibt es auch ein $\omega$ bei dem das $U_{\symup{C}}$ innerhalb eines Maximums die 
    Erregerspannung übertrifft. Dies Phänomen wird als Resonanz bezeichnet und tritt bei der Resonanzfrequenz $\omega_{\text{res}}$. 
    Eine Rechnung ergibt eine Formel für $\omega_{\text{res}}$:
    \begin{equation}
        \omega_{\text{res}} = \sqrt{\frac{1}{LC}-\frac{R^2}{2L^2}} \nonumber
    \end{equation}

    für: 
    \begin{equation}
        \frac{R^2}{2L^2} \ll \frac{1}{LC} \nonumber
    \end{equation}

    ergibt sich ein Fall von besonderer Bedeutung in der sich $\omega_{\text{res}}$ an $\omega_0$ der Frequenz einer ungedämpften 
    Schwingung annähert. In diesem Fall übertrifft $U_{\symup{C}}$ die Erregerspannung.
    \begin{equation}
        U_{\text{C,max}} = \frac{1}{\omega_0 RC} = \frac{1}{R} \sqrt{\frac{L}{C}} U_0 \nonumber
    \end{equation}

    Somit kann $U_{\symup{C},\text{max}} \to \infty$ für $R \to 0$ gehen, dies wird als "Resonanzkatastrophe" bezeichnet. Der Faktor
    $\frac{1}{\omega_0 RC}$ wird auch als als Resonanzüberhöhung oder Güte q des Schwingkreises bezeichnet. Die Schärfe der Resonanz 
    der Resonanzkurve wird durch die Schnittpunkte $\omega_{\pm}$ die jeweils den Ort an dem $U_{\symup{C}}$ auf einen Faktor $\frac{1}{\sqrt{2}}$ 
    absinken. Somit werden $\omega_+$ und $\omega_-$ durch 
    \begin{equation}
        \frac{U_0}{\sqrt{2}} \frac{1}{\omega_0 RC} = \frac{U_0}{C \sqrt{\omega^2_{\pm} R^2 + \left( \omega^2_{\pm}L - \frac{1}{C} \right) }} \nonumber
    \end{equation}
    beschrieben.

    Für $\frac{R^2}{L^2} \ll {\omega}^2_0$ folgt ebenfalls für die Breite der Resonanzkurve
    \begin{equation}
        \omega_+ - \omega_- \approx \frac{R}{L} 
        \label{eq:v1}
    \end{equation}

    Somit besteht auch folgende Beziehung zwischen der Güte und der Breite:
    \begin{equation}
        q = \frac{\omega_0}{\omega_+ - \omega_-} \nonumber
    \end{equation}

    Im Gegensatz dazu ist im Falle der Starken Dämpfung:
    \begin{equation}
        \frac{R^2}{2L^2} \gg \frac{1}{LC} \nonumber
    \end{equation}

    Hier fällt jetzt die Spannung $U_{\symup{C}}$ ausgehend von $U_0$ mit steigender Frequez $\omega$ gegen 0. Bei hohen Frequenzen fällt
    die Spannung proportional zu $\frac{1}{\omega^2}$. Unter diesen Umständen kann der RLC-Kreis auch als Tiefpass benutzt werden.
    \begin{equation}
        \omega^2_0 = \frac{1}{LC} \nonumber
    \end{equation}

    Als nächstes wird die Frequenzabhängigkeit der Phase untersucht. Aus \ref{eq:phi} ist zu erkennen, dass für kleine $\omega$ die 
    Kondensatorspannung und sie Erregerspannung fast in Phase sind jedoch bei hohen Frequenzen $U_{\symup{C}}$ etwa eine um $\pi$ 
    kleinere Phase als $U$ hat. $\phi$ ist = $\frac{-\pi}{2}$ an der Stelle $\omega^2_0 = \frac{1}{LC}$.
    Aus \ref{eq:phi} folgt ebenfalls, dass $\omega_{1,2}$ bei denen $\phi= \frac{\pi}{4} \text{oder} \frac{3\pi}{4}$ ist die Beziehung:
    \begin{equation}
        \omega_{1,2} = \pm  \frac{R}{2L} + \sqrt{\frac{R^2}{4L^2}+ \frac{1}{LC}} \nonumber
    \end{equation}

    Und daraus auch:
    \begin{equation}
        \omega_1 - \omega_2 = \frac{R}{L} 
        \label{eq:v2}
    \end{equation}

    Somit ist im Vergleich von \ref{eq:v1} und \ref{eq:v2} zu sehen, dass im Falle schwacher Dämpfung $\omega_1$ und $\omega_2$ 
    $\omega_+ - \omega_-$ zusammenfällt.

    \section{Impedanz des Schwingkreises}

    Ein Schwingkreis lässt sich auch wie in Abbildung ?ref? als Zweipol beschreiben. An seinen Enden kann nun ein frequenzabhängiger 
    Widerstand r, dieser wird als Impedanz bezeichnet. Aufgrund der Phasenverschiebung zwischen der Spannung und dem Strom muss r als 
    komplexe Zehl definiert werden :
    \begin{equation}
        r = X + jY \nonumber
    \end{equation}

    Hier sind $X$ und $Y$ reele Widerstände, $X$ ist  der Wirkwiderstand und $Y$ ist der Blindwiderstand oder die Reaktanz. Der Betrag
    \begin{equation}
        |r| = \sqrt{X^2 + Y^2} \nonumber
    \end{equation}
    wird als Scheinwiderstand bezeichnet. In der komplexen Zahlenebene wird der Verlauf der Größe r($\omega$) als Ortskurve bezeichnet. 
    r ist dort nun ein Pfeil vom Ursprung dessen Länge der Scheinwiderstand und der Winkel zur reelen Achse die Phasenverschiebung 
    beschreibt.

    \begin{align}
        r_{\symup{C}} = -j\frac{1}{\omega C} \nonumber\\
        r_{\symup{L}} = j\omega L  \nonumber\\
        r_{\symup{R}} = R_{\symup{s}} \nonumber
    \end{align}

    \begin{equation}
        r_{\symup{S}} = R_{\symup{S}} + j \left(    \omega L - \frac{1}{\omega C} \right) \nonumber
    \end{equation}

    \begin{align}
        X_{\symup{S}} = R_{\symup{S}}  \nonumber\\ 
        Y_{\symup{S}} = j \omega L - \frac{1}{\omega C} \nonumber
    \end{align}

    \begin{equation}
        |r_{\symup{S}}| = \sqrt{R^2_{\symup{S}} + \left( \omega L - \frac{1}{\omega C} \right)^2} \nonumber
    \end{equation}

    \begin{equation}
        r_{\symup{p}} = \frac{\frac{1}{R_{\symup{p}}}+j \left( \frac{1}{\omega L} - \omega C \right)}{\frac{1}{R^2_{\symup{p}}}+ \left( \frac{1}{\omega L} - \omega C \right)^2} \nonumber
    \end{equation}

    \begin{equation}
        |r_{\symup{p}}|=  \frac{1}{\sqrt{\frac{1}{R^2_{\symup{p}}} \left( \frac{1}{\omega L} - \omega C \right)^2}} \nonumber
    \end{equation}
