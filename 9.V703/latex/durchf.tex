\newpage
\section{Durchführung}

\subsection{a}
Beta quelle vor Zählrohr so das die Impulsrate nicht über 100/s geht damit die Totzeit nicht für zu große führt
Im Bereich des Plateaus muss die Anzahl sehr genau bestimmt werden damit die relativ kleine Plateau steigung bestimmt werden kann.
Die Messintervall Länge wird auf 120 Sekunden gesetzt damit der Fehler jedes Messpunktes $\ll$ 1$\percent$ ist.
Die Spannung darf 700V nicht überschreiten damit das Geiger-Müller-Zählrohr nicht durch selbstständige Gasentladung zerstoert wird.

\subsection{b}

Nachentladung, 
absolut keine Ahnung

\subsection{c}

Zur Messung der Totzeit mittels des Oszilloskops wird der Oszillosgraph auf den Anstieg des Impulses getriggert.
Mittels der Kathodenstrahlgeschwindigkeit lässt sich mit dem enntstandenen Bild eine Totzeit ablesen und eine Erholungszeit abschätzen.

\subsection{d}
Zur Bestimmung der Totzeit mit der 2 Quellen Methode wird erst die Quelle Nummer 1 auf das Zählrohr gerichtet und die Impulszahl gemessen.
Dann wird die Quelle 1 und 2 auf das Zählrohr gerichtet und die Impulszahl der beiden Quellen kombiniert gemessen. Zuletzt wird die Impulszahl gemessen 
während nur Quelle 2 auf das Zählrohr zeigt. Mit diesen 3 Werten lässt sich die Totzeit berechnen.

