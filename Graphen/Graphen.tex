\input{header/header.tex}


\begin{document}

\subject{Praktikum}
\title{Graphen Übungsaufgabe}

\maketitle
\thispagestyle{empty}
%\tableofcontents
\newpage

\section{Federkonstante}
    Durch mathematisches Fitten der der Werte für die Masse und die korrelierenden Auslenkungen können wir eine Ausgleichsgerade
    der form 
    \begin{equation}
        x = \symup{a} \cdot m + \symup{b}
    \end{equation} 
    bestimmen. Das a beschreibt in diesem Fall a=$\frac{g}{k}$ mit der Schwerenbeschleunigung g=$\SI{9,81}{meter\per\second\squared}$ 
    und der zu bestimmenden Federkonstente k.
    Durch Auswerten der Ausgleichsgerade erhalten wir eine Federkonstante k=$\SI{17,518}{\kilogram\per\second\squared}$
    \begin{figure}
            \centering
            \includegraphics{build/Graph1.pdf}
            \caption{G/B Diagramm}
            \label{fig:plt1}
        \end{figure}
    
    
\newpage

\section{Brennweite}

    \subsection{a}

    \subsection{b}
        

    \subsection{c}
\newpage

\section{Absorptionsgesetz}
    

\printbibliography{}

\end{document}
