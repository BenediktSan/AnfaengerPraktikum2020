\input{header/header.tex}


\begin{document}

\subject{Praktikum}
\title{Graphen Übungsaufgabe}

\maketitle
\thispagestyle{empty}
%\tableofcontents
\newpage

\section{Federkonstante}
    Durch mathematisches Fitten der der Werte für die Masse und die korrelierenden Auslenkungen können wir eine Ausgleichsgerade
    der Form 
    \begin{equation}
        x = \symup{a} \cdot m + \symup{b}
    \end{equation} 
    bestimmen. Das a beschreibt in diesem Fall a=$\frac{g}{k}$ mit der Schwerenbeschleunigung g=$\SI{9,81}{meter\per\second\squared}$ 
    und der zu bestimmenden Federkonstente k.
    Durch Auswerten der Ausgleichsgerade erhalten wir eine Federkonstante k=$\SI{17,518}{\kilogram\per\second\squared}$
    \begin{figure}
            \centering
            \includegraphics{build/Graph1.pdf}
            \caption{ Diagramm}
            \label{fig:plt1}
    \end{figure}
    
    
\newpage

\section{Brennweite}
In dieser Aufgabe wird die Brennweite einer Linse ausgerechnet, hier Vergleichen wir das direkte Ausrechnen über eine Formel und
das approximieren über eine Ausgleichsgerade. Folgende Werte werden zum Berechnen der Werte genutzt:
\begin{table}
    \centering
    \begin{tabular}{S[table-format=3.0] S [table-format=3.0]}
        \toprule
        {Gegenstandsweite g [mm]} & {Bildweite b [mm]}  \\
        \midrule
        60  & 285\\
        80  & 142\\
        100 & 117\\
        110 & 85\\ 
        120 & 86\\
        125 & 82\\
        \bottomrule      
    \end{tabular}
\end{table}

\subsection{a}
In Teilaufgabe a) wird die Brennweite direkt f über die Formel 
\begin{equation}
\frac{1}{f} = \frac{1}{b} + \frac{1}{g}
\end{equation}
berechnet. Die berechneten Brennweiten und weitere Statistische Untersuchungen ergeben dann:\\
\begin{table}
    \centering
    \begin{tabular}{S[table-format=1] S [table-format=2.3]}
        \toprule
        {Messpar} & {Brennweite f [mm]}  \\
        \midrule
        1 & 49.565\\
        2 & 51.171\\
        3 & 53.917\\
        4 & 47.949\\ 
        5 & 50.097\\
        6 & 49.516\\
        \bottomrule  
    \end{tabular}   
\end{table}
\begin{align}
\text{Mittelwert}&= \num{50.369}\\
\text{Standardabweichung}&= \num{2.027}\\
\text{Fehler des Mittelwerts}&= \num{0.827}
\end{align}

\subsection{b}  
In Teilaufgabe b) wird mittels einer durch Linearen Regression ausgerechten Ausgleichsgerade der Form
\begin{equation}
    \frac{1}{g} = \frac{1}{b} \cdot m + \symup{a},
   % \eqref{eqn:ausg}
\end{equation} 
die Brenweite der Linse ausgerechnet. Es folgt $f=\frac{1}{a}$.

Mit den selben Werten für $b$ und $g$ aus a) erhalten wir folgende Werte für die Ausgleichsgerade:
\begin{figure}  
    \centering
    \includegraphics{build/Graph2.pdf}
    \caption{ Diagramm}
    \label{fig:plt2}
\end{figure}
Aus a=$\SI{0,0193 +- 0,0011}{\per\milli\meter}$ ergibt sich f= $\SI{51.7+-2.8}{\milli\meter}$
\subsection{c}
Wir erhalten in Teilaufgabe a und b leicht unterschiedliche Werte für die Brennweite f, jedoch würde ich die beiden Auswertungsmethoden
gleichwertig behandeln, denn die Methode mittels der Linearen Regression behinhaltet bereits die einzelnen Messunsicherheiten, mit der 
Methode aus Teilaufgabe a) müssen die einzelnen Ergebnisse noch auf Statistische Unsicherheiten untersucht werden.
\newpage

 

\section{Absorptionsgesetz}
    

\printbibliography{}

\end{document}
