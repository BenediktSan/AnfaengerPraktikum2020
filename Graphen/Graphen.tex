\input{header/header.tex}


\begin{document}

\subject{Praktikum}
\title{Graphen Übungsaufgabe}

\maketitle
\thispagestyle{empty}
%\tableofcontents
\newpage

\section{Federkonstante}
    Durch mathematisches Fitten der der Werte für die Masse und die korrelierenden Auslenkungen können wir eine Ausgleichsgerade
    der Form 
    \begin{equation}
        x = \symup{a} \cdot m + \symup{b}
    \end{equation} 
    bestimmen. Das a beschreibt in diesem Fall a=$\frac{g}{k}$ mit der Schwerenbeschleunigung g=$\SI{9,81}{meter\per\second\squared}$ 
    und der zu bestimmenden Federkonstente k.
    Durch Auswerten der Ausgleichsgerade erhalten wir eine Federkonstante k=$\SI{17,518}{\kilogram\per\second\squared}$
    \begin{figure}
            \centering
            \includegraphics{build/Graph1.pdf}
            \caption{ Diagramm}
            \label{fig:plt1}
        \end{figure}
    
    
\newpage

\section{Brennweite}
In dieser Aufgabe wird die Brennweite einer Linse ausgerechnet, hier Vergleichen wir das direkte Ausrechnen über eine Formel und
das approximieren über eine Ausgleichsgerade. Folgende Werte werden zum Berechnen der Werte genutzt:
\begin{table}
    \centering
    \begin{tabular}{S[table-format=3.0] S [table-format=3.0]}
        \toprule
        {Gegenstandsweite g [mm]} & {Bildweite b [mm]}  \\
        \midrule
        60  & 285\\
        80 & 142\\
        100 & 117\\
        110 & 85\\ 
        120 & 86\\
        125 & 82\\
        \bottomrule      
    \end{tabular}
\end{table}

\subsection{a}
In Teilaufgabe a) wird die Brennweite direkt f über die Formel 
\begin{equation}
\frac{1}{f} = \frac{1}{b} + \frac{1}{g}
\end{equation}
berechnet. Die berechneten Brennweiten ergeben dann:
\begin{table}
    \centering
    \begin{tabular}{S[table-format=1] S [table-format=2.3]}
        \toprule
        {Messpar} & {Brennweite f [mm]}  \\
        \midrule
        1 & 49.565\\
        2 & 51.171\\
        3 & 53.917\\
        4 & 47.949\\ 
        5 & 50.097\\
        6 & 49.5169\\
        \bottomrule      
    \end{tabular}
\end{table}

\subsection{b}  
In Teilaufgabe b) wird durch eine Ausgleichsgerade der Form
\begin{equation}
    \frac{1}{g} = \frac{1}{b} \cdot m + \symup{a}
\end{equation} 

\subsection{c}


\newpage

\section{Absorptionsgesetz}
    

\printbibliography{}

\end{document}
