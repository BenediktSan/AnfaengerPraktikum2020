\documentclass[
  bibliography=totoc,     % Literatur im Inhaltsverzeichnis
  captions=tableheading,  % Tabellenüberschriften
  titlepage=firstiscover, % Titelseite ist Deckblatt
]{scrartcl}

% Paket float verbessern
\usepackage{scrhack}

% Warnung, falls nochmal kompiliert werden muss
\usepackage[aux]{rerunfilecheck}

% unverzichtbare Mathe-Befehle
\usepackage{amsmath}
% viele Mathe-Symbole
\usepackage{amssymb}
% Erweiterungen für amsmath
\usepackage{mathtools}

% Fonteinstellungen
\usepackage{fontspec}
% Latin Modern Fonts werden automatisch geladen
% Alternativ zum Beispiel:
%\setromanfont{Libertinus Serif}
%\setsansfont{Libertinus Sans}
%\setmonofont{Libertinus Mono}

% Wenn man andere Schriftarten gesetzt hat,
% sollte man das Seiten-Layout neu berechnen lassen
\recalctypearea{}

% deutsche Spracheinstellungen
\usepackage[ngerman]{babel}


\usepackage[
  math-style=ISO,    % ┐
  bold-style=ISO,    % │
  sans-style=italic, % │ ISO-Standard folgen
  nabla=upright,     % │
  partial=upright,   % ┘
  warnings-off={           % ┐
    mathtools-colon,       % │ unnötige Warnungen ausschalten
    mathtools-overbracket, % │
  },                       % ┘
]{unicode-math}

% traditionelle Fonts für Mathematik
\setmathfont{Latin Modern Math}
% Alternativ zum Beispiel:
%\setmathfont{Libertinus Math}

\setmathfont{XITS Math}[range={scr, bfscr}]
\setmathfont{XITS Math}[range={cal, bfcal}, StylisticSet=1]

% Zahlen und Einheiten
\usepackage[
  locale=DE,                   % deutsche Einstellungen
  separate-uncertainty=true,   % immer Unsicherheit mit \pm
  per-mode=symbol-or-fraction, % / in inline math, fraction in display math
]{siunitx}

% chemische Formeln
\usepackage[
  version=4,
  math-greek=default, % ┐ mit unicode-math zusammenarbeiten
  text-greek=default, % ┘
]{mhchem}

% richtige Anführungszeichen
\usepackage[autostyle]{csquotes}

% schöne Brüche im Text
\usepackage{xfrac}

% Standardplatzierung für Floats einstellen
\usepackage{float}
\floatplacement{figure}{htbp}
\floatplacement{table}{htbp}

% Floats innerhalb einer Section halten
\usepackage[
  section, % Floats innerhalb der Section halten
  below,   % unterhalb der Section aber auf der selben Seite ist ok
]{placeins}

% Seite drehen für breite Tabellen: landscape Umgebung
\usepackage{pdflscape}

% Captions schöner machen.
\usepackage[
  labelfont=bf,        % Tabelle x: Abbildung y: ist jetzt fett
  font=small,          % Schrift etwas kleiner als Dokument
  width=0.9\textwidth, % maximale Breite einer Caption schmaler
]{caption}
% subfigure, subtable, subref
\usepackage{subcaption}

% Grafiken können eingebunden werden
\usepackage{graphicx}

% schöne Tabellen
\usepackage{booktabs}

% Verbesserungen am Schriftbild
\usepackage{microtype}

% Literaturverzeichnis
\usepackage[
  backend=biber,
]{biblatex}
% Quellendatenbank
\addbibresource{lit.bib}
\addbibresource{programme.bib}

% Hyperlinks im Dokument
\usepackage[
  german,
  unicode,        % Unicode in PDF-Attributen erlauben
  pdfusetitle,    % Titel, Autoren und Datum als PDF-Attribute
  pdfcreator={},  % ┐ PDF-Attribute säubern
  pdfproducer={}, % ┘
]{hyperref}
% erweiterte Bookmarks im PDF
\usepackage{bookmark}

% Trennung von Wörtern mit Strichen
\usepackage[shortcuts]{extdash}

\author{%
  Niklas Düser\\%
  \href{mailto:niklasdueser@tu-dortmund.de}{niklas.dueser@tu-dortmund.de}%
  \and%
  Benedikt Sander\\%
  \href{mailto:benedikt.Sander@tu-dortmund.de}{benedikt.sander@tu-dortmund.de}%
}
\publishers{TU Dortmund – Fakultät Physik}

\begin{document}
    \section{Aufgabe }
        \subsection{}
        Der Mitelwert ist eine Art "Durchschnittswert" von Messwerten.

        \subsection{}
        Die Standardabweichung ist ein Maß für die Abweichung

        \subsection{}


    \section{Aufgabe}
        \begin{equation}
            \sigma = \sqrt{ \frac{1}{(n-1)} \sum_{k=1}^n (x_k -\overline{x})^2 }
        \end{equation}
        
    Um den Fehler zu Berechnen wird $\symup{C}$ so beliebig gewählt, so dass man für den Fehler 10 $\symup{n_0}$ als 2 (kleinstmögliches $n$) setzen kann. 
    Dies ist $\symup{C}=100$. Am Ende werden dann $\symup{n_0}-n$ zusätliche Schritte benötigt.
        \begin{equation}
           \symup{C}=\sum_{k=1}^n (x_k -\overline{x})^2 \\
        \end{equation}

        \begin{align}  
            \symup{C}&=100\si{\metre} & \sigma_u &= 10 \si{\metre\per\second}\\
            \implies \sigma_u &= \sqrt{ \frac{1}{(\symup{n_0}-1)} \cdot \symup{C}}\\
            \iff n_0 &= \frac{\symup{C}}{( \sigma_u)^2}+1\\
            \implies n_0 &= 2
        \end{align}
        \\
        \\  
        Für $ \sigma =\SI{3}{\metre\per\second} $
        \begin{align}
            \sigma &= \SI{3}{\metre\per\second}\\
            \implies  n &= 12.11111111 
        \end{align}
        Es werden für eine Unsicherheit von  $\pm 3 \si{\metre\per\second}$ also $\symup{n_0}-n $ und damit $\approx 11$ weitere Messungen benötigt.
         \\
         \\
        Für  $\sigma = \SI{0.5}{\metre\per\second}$ 
        \begin{align}
            \sigma &= \SI{0.5}{\metre\per\second}\\
            \implies n &= 401 \quad \quad \; \; \; \;
        \end{align}
        Es werden für eine Unsicherheit von $\pm 0.5\si{\metre\per\second}$ also $\symup{n_0}-n$ und damit $\approx 309$ weitere Messungen benötigt.

        \section{Aufgabe}

        \begin{align}
            R_\text{außen}&= \SI{15(1)}{\centi\metre} & R_\text{innen}&=\SI{10(1)}{\centi\metre} & h&=\SI{20(1)}{\centi\metre} 
        \end{align}

        \begin{align}
            V &=\pi \cdot ((R_\text{außen})^2 - (R_\text{innen})^2) \cdot h \\
            \implies V &= \SI{7853.98163397448}{\cubic\centi\metre}
        \end{align}
        \\
        Fehlerformel und Fehlerwert:
        \begin{align}
            \increment V &= \sqrt{
                \left(\frac{\partial V}{\partial R_\text{außen} }\right)^2 \cdot \left(\increment R_\text{außen} \right)^2 +
                \left(\frac{\partial V}{\partial R_\text{innen} }\right)^2 \cdot \left(\increment R_\text{innen} \right)^2 +
                \left(\frac{\partial V}{\partial h }\right)^2 \cdot \left(\increment h \right)^2
                }\\
        \increment V&=
        \sqrt{
        \begin{aligned}
                & \left(
                2 \pi \cdot R_\text{außen} \cdot h \right) ^2 \cdot (\increment R_\text{außen} )^2
                + \left( -2 \pi \cdot R_\text{innen} \cdot h \right) ^2 \cdot (\increment R_\text{innen} )^2 \\
                &+ \left( \pi \cdot ((R_\text{außen})^2 - (R_\text{innen})^2) \right) ^2 \cdot (\increment h )^2
        \end{aligned}
        }\\
        \implies \increment V &= \SI{229.921874934367}{\cubic\centi\metre}
    \end{align}
            
Das Volumen des Hohlzylinders beträgt also $\approx7853.9816\pm 229.9219\; \si{\cubic\centi\metre}$.
\end{document}