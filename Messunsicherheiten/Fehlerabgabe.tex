\input{../Vorlagen//header.tex}
\begin{document}
    \section{Aufgabe }
        \subsection{}
        Der Mitelwert ist eine Art "Durchschnittswert" von Messwerten.

        \subsection{}
        Die Standardabweichung ist ein Maß für die Abweichung

        \subsection{}


    \section{Aufgabe}
        \begin{equation}
            \sigma = \sqrt{ \frac{1}{(n-1)} \sum_{k=1}^n (x_k -\overline{x})^2 }
        \end{equation}
        
    Um den Fehler zu Berechnen wird $\symup{C}$ so beliebig gewählt, so dass man für den Fehler 10 $\symup{n_0}$ als 2 (kleinstmögliches $n$) setzen kann. 
    Dies ist $\symup{C}=100$. Am Ende werden dann $\symup{n_0}-n$ zusätliche Schritte benötigt.
        \begin{equation}
           \symup{C}=\sum_{k=1}^n (x_k -\overline{x})^2 \\
        \end{equation}

        \begin{align}  
            \symup{C}&=100\si{\metre} & \sigma_u &= 10 \si{\metre\per\second}\\
            \implies \sigma_u &= \sqrt{ \frac{1}{(\symup{n_0}-1)} \cdot \symup{C}}\\
            \iff n_0 &= \frac{\symup{C}}{( \sigma_u)^2}+1\\
            \implies n_0 &= 2
        \end{align}
        \\
        \\  
        Für $ \sigma =\SI{3}{\metre\per\second} $
        \begin{align}
            \sigma &= \SI{3}{\metre\per\second}\\
            \implies  n &= 12.11111111 
        \end{align}
        Es werden für eine Unsicherheit von  $\pm 3 \si{\metre\per\second}$ also $\symup{n_0}-n $ und damit $\approx 11$ weitere Messungen benötigt.
         \\
         \\
        Für  $\sigma = \SI{0.5}{\metre\per\second}$ 
        \begin{align}
            \sigma &= \SI{0.5}{\metre\per\second}\\
            \implies n &= 401 \quad \quad \; \; \; \;
        \end{align}
        Es werden für eine Unsicherheit von $\pm 0.5\si{\metre\per\second}$ also $\symup{n_0}-n$ und damit $\approx 309$ weitere Messungen benötigt.

        \section{Aufgabe}

        \begin{align}
            R_\text{außen}&= \SI{15(1)}{\centi\metre} & R_\text{innen}&=\SI{10(1)}{\centi\metre} & h&=\SI{20(1)}{\centi\metre} 
        \end{align}

        \begin{align}
            V &=\pi \cdot ((R_\text{außen})^2 - (R_\text{innen})^2) \cdot h \\
            \implies V &= \SI{7853.98163397448}{\cubic\centi\metre}
        \end{align}
        \\
        Fehlerformel und Fehlerwert:
        \begin{align}
            \increment V &= \sqrt{
                \left(\frac{\partial V}{\partial R_\text{außen} }\right)^2 \cdot \left(\increment R_\text{außen} \right)^2 +
                \left(\frac{\partial V}{\partial R_\text{innen} }\right)^2 \cdot \left(\increment R_\text{innen} \right)^2 +
                \left(\frac{\partial V}{\partial h }\right)^2 \cdot \left(\increment h \right)^2
                }\\
        \increment V&=
        \sqrt{
        \begin{aligned}
                & \left(
                2 \pi \cdot R_\text{außen} \cdot h \right) ^2 \cdot (\increment R_\text{außen} )^2
                + \left( -2 \pi \cdot R_\text{innen} \cdot h \right) ^2 \cdot (\increment R_\text{innen} )^2 \\
                &+ \left( \pi \cdot ((R_\text{außen})^2 - (R_\text{innen})^2) \right) ^2 \cdot (\increment h )^2
        \end{aligned}
        }\\
        \implies \increment V &= \SI{229.921874934367}{\cubic\centi\metre}
    \end{align}
            
Das Volumen des Hohlzylinders beträgt also $\approx7853.9816\pm 229.9219\; \si{\cubic\centi\metre}$.
\end{document}