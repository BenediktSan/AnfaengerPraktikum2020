\input{header/header.tex}


\begin{document}

\subject{Praktikum}
\title{Graphen Übungsaufgabe}

\maketitle
\thispagestyle{empty}
%\tableofcontents
\newpage

\section{Federkonstante}
\noindent Durch mathematisches Fitten der der Werte für die Masse und die korrelierenden Auslenkungen können wir eine Ausgleichsgerade
    der Form 
    \begin{equation}
        x = \symup{a} \cdot m + \symup{b} \nonumber
    \end{equation} 
    \noindent bestimmen. Das a beschreibt in diesem Fall a=$\frac{g}{k}$ mit der Schwerenbeschleunigung g=$\SI{9,81}{meter\per\second\squared}$ 
    und der zu bestimmenden Federkonstente k.
    Durch Auswerten der Ausgleichsgerade erhalten wir eine Federkonstante k=$\SI{17,518}{\kilogram\per\second\squared}$
    \begin{figure}[H] 
            \centering
            \includegraphics{build/Graph1.pdf}
            \caption{ Diagramm}
            \label{fig:plt1}
    \end{figure}
    
    
\newpage

\section{Brennweite}
\noindent In dieser Aufgabe wird die Brennweite einer Linse ausgerechnet, hier Vergleichen wir das direkte Ausrechnen über eine Formel und
das approximieren über eine Ausgleichsgerade. Folgende Werte werden zum Berechnen der Werte genutzt:
\begin{table}[H] 
    \centering
    \begin{tabular}{S[table-format=3.0] S [table-format=3.0]}
        \toprule
        {Gegenstandsweite g [mm]} & {Bildweite b [mm]}  \\
        \midrule
        60  & 285\\
        80  & 142\\
        100 & 117\\
        110 & 85\\ 
        120 & 86\\
        125 & 82\\
        \bottomrule      
    \end{tabular}
\end{table}

\subsection{a}
\noindent In Teilaufgabe a) wird die Brennweite direkt f über die Formel 
\begin{equation}
\frac{1}{f} = \frac{1}{b} + \frac{1}{g} \nonumber
\end{equation}
\noindent berechnet. Die berechneten Brennweiten und weitere Statistische Untersuchungen ergeben dann:\\
\begin{table}[H] 
    \centering
    \begin{tabular}{S[table-format=1] S [table-format=2.3]}
        \toprule
        {Messpar} & {Brennweite f [mm]}  \\
        \midrule
        1 & 49.565\\
        2 & 51.171\\
        3 & 53.917\\
        4 & 47.949\\ 
        5 & 50.097\\
        6 & 49.516\\
        \bottomrule  
    \end{tabular}   
\end{table}
\begin{align}
\text{Mittelwert}&= \num{50.369}  \nonumber\\
\text{Standardabweichung}&= \num{2.027} \nonumber  \\
\text{Fehler des Mittelwerts}&= \num{0.827} \nonumber
\end{align}

\subsection{b}  
\noindent In Teilaufgabe b) wird mittels einer durch Linearen Regression ausgerechten Ausgleichsgerade der Form
\begin{equation}
    \frac{1}{g} = \frac{1}{b} \cdot m + \symup{a} \nonumber
   % \eqref{eqn:ausg}
\end{equation} 
\noindent die Brenweite der Linse ausgerechnet.
Mit den selben Werten für $b$ und $g$ aus a) erhalten wir folgende Werte für die Ausgleichsgerade:
\begin{figure}[H]  
    \centering
    \includegraphics{build/Graph2.pdf}
    \caption{Brennweite}
    \label{fig:plt2}
\end{figure}
\noindent Hier ist $m = \num{-0.94+-0.11}$ und $a = \num{0.0193+-0.0011}$, da $m \approx -1$ ist, folgt $f=\frac{1}{a}$.
Aus a=$\SI{0,0193 +- 0,0011}{\per\milli\meter}$ ergibt sich somit f= $\SI{51.7+-2.8}{\milli\meter}$
\subsection{c}
\noindent Wir erhalten in Teilaufgabe a und b leicht unterschiedliche Werte für die Brennweite f, jedoch würde ich die beiden Auswertungsmethoden
gleichwertig behandeln, denn die Methode mittels der Linearen Regression behinhaltet bereits die einzelnen Messunsicherheiten, mit der 
Methode aus Teilaufgabe a) müssen die einzelnen Ergebnisse noch auf Statistische Unsicherheiten untersucht werden.
\newpage

 

\section{Absorptionsgesetz}
    In dieser Aufgabe wird das Absortsionsegetz mittels einer einer e - Funktion untersucht. Dazu wird ein Fit für die Messwerte:
    \begin{table}[H] 
        \centering
        \begin{tabular}{S[table-format=1.1] S [table-format=4]}
            \toprule
            {d[cm]} & {N [1/60s]}  \\
            \midrule
            0.1 &7565   \\
            0.2 &6907   \\
            0.3 &6214   \\
            0.4 &5531   \\
            0.5 &4942   \\
            1.0 &2652   \\
            1.2 &2166   \\
            1.5 &1466   \\
            2.0 &970    \\
            3.0 &333    \\
            4.0 &127    \\
            5.0 &48     \\
            \bottomrule  
        \end{tabular}   
    \end{table}
    \noindent berechnet. Der Fit wird durch $N_0 * \symup{e}^{- \mu d}$ beschrieben. Hier beschreibt $N_0$ die Anzahl der Gamma Quanten ohne 
    Abschirmung, $d$ die Dicke der Platte und $\mu$ ist der Absorptionskoeffizient. Mit den angegebenen Messweten berechnet sich  
    $\mu$ zu : $\num{0.94+-0.06}$ und $N_0$ zu $\num{8640+-80}$.

    \begin{figure}[H] 
        \centering
        \includegraphics{build/Graph3.pdf}
        \caption{Absorption}
        \label{fig:plt3}
    \end{figure}
    \noindent Hier sind jeweils die Messdaten und der ausgerechnete Fit eingezeichnet. Die Messfehler sind bereits an die Messwerte eingezeichnet     
    und mit einem Faktor 5 vergoessert. Dies dient der besseren Anschaulichkeit, zusätzlich sind die Messfehler auch noch einmal einzeln 
    eingefügt.


    \printbibliography{}
    
    \end{document}
    
