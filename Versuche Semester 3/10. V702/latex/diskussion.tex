\newpage
\section{Diskussion}
Da für diesen Versuch nur Messwerte zur Verfügung gestellt wurden fällt die Diskussion der Versuchsdurchführung natürlich an dieser Stelle weg.\\\\
Für $\ce{^{52}_{}V}$ wurde eine Halbwertszeit von $\SI{202.61626(723460)}{\second}$ berechnet, was einer relativen Abweichung vom Literaturwert von 
$\SI{9.8(32)}{\percent}$ entspricht. Dieses Ergebnis ist also wirklich gut gelungen.\\
Für diesen Zerfallsprozess wurde sogar noch eine zweite Fit-Funktion erstellt, welche diesmal aber nur Werte genutzt hat, die bis zur doppelten Halbwertszeit reichen.
Damit wurde eine noch genauere Halbwertszeit von $\SI{208.92148(1674884)}{\second}$ mit einer Abweichung von $\SI{7(7)}{\percent}$ berechnet.\\
Allerdings ist hier der Fehler wesentlich größer, was vermutlich an der kleineren Menge an genutzten Messwerten liegt, weswegen der Fehler des Fits größer geworden ist.\\
Alles in allem ist diese Messreihe aber sehr gelungen. Abweichung lassen sich unter anderem durch den wachsenden Einfluss der Nullrate für kleine Zerfalllsraten erklären.
Außerdem hätte noch eine höhere Genaugkeit erzielt werden können, wenn noch mehr Nullraten Messungen vorgenommen wären.\\
Die letzten beiden Punkte gelten auch für die Auswertung des Zerfalls von $\ce{^{104}_{}Rh}$.\\\\
Der Untersuchung des Zerfalls von $\ce{^{104}_{}Rh}$ hat auch sehr gute Ergebnisse geliefert.\\
Zuerst wurde die Halbwertszeit von $\ce{^{104i}_{}Rh}$ untersucht. Dabei wurde eine Halbwertszeit von $\SI{262.38900(4969076)}{\second}$ und eine Abweichung von $\SI{-01(19)}{\percent}$ errechnet. 
Diese Werte sind zwar sehr gut, allerdings gibt es bei dieser Bestimmung dieses Wertes zwei Probleme.\\
Erstens wurde das Intervall, aus dem die Messwerte genutzt werden etwas willkürlich gewählt, was den Einfluss des kurzlebigeren Isotops nicht ganz ausschließen lässt.\\
Zum anderen wird der Fehler bei spät aufgenommenen Messwerten, unter anderem wegen des wachsenden Einflusses der Nullrate, größer, was genaue Rechnungen erschwert.
Die Werte sind nichtsdestotrotz sehr nah am Theroriewert.\\
Für die Untersuchung des Zerfalls von $\ce{^{104}_{}Rh}$, also das kurzlebigere Isotop, musste wieder halbwegs willkürlich ein Intervall bestimmt werden in dem die Untersuchung vorgenommen wurden.
Außerdem sind die bestimmten Werte zusätzlich noch von den Werten des anderen Zerfalls abhängig. Da diese aber sehr gut geworden sind, ist dieser Fakt wohl eher zu vernachlässigen.\\
Die Rechnungen haben zu einer Halbwertszeit von $\SI{39.68759(243149)}{\second}$ und einer Abweichung von $\SI{6(6)}{\percent}$ geführt, welche auch sehr gut sind.\\\\
Alles in allem hat der Versuch sehr gute Ergebnise für die Halbwertszeiten der Isoptope erzielt.
\newpage


