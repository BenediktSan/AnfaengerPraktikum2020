\section{Zielsetzung}
In diesem Versuch wird eine Methode zur Bestimmung der Halbwertszeit $T$ unterschiedlicher Atome durch Aktivierung mittels Neutronen untersucht.


\section{Theorie}

\subsection{Kernreaktionen mit Neutronen}

\noindent Kernreaktionen sind im allgemeinen Reaktionen eines Teilchens mit dem Kern eines Atoms. In diesem Verusch werden die Wechselwirkungen 
von Neutronen mit einem Kern genauer untersucht. Absorbiert ein Kern A ein Neutronen bei solch einer Wechselwirkungen entsteht ein neuer Kern
$\text{A}^{\ast}$, dieser wird Zwischenkern oder Compound Kern genannt. Der kern $\text{A}^{\ast}$ nimmt die kinetische und die Bindungsenergie 
des Neutrons auf und liegt somit energetisch über dem Kern A. Diese Energie wird dann innerhalb des Kerns auf die einzelnen Nukleonen verteilt, 
somit reicht die Energie nicht mehr aus um ein Nukleon des Kerns wieder auszustoßen. Um nun wieder in den Grundzustand zurück zu kommen gibt der 
angere Kern Energie in Form eines $\upgamma$-Quants nach etwa $\SI{10e-16}{\second}$ ab.\\
Es läuft folgende Reaktion ab:

\begin{equation}
   \ce{^m_z\symup{A} + ^1_0\symup{n} -> ^{m+1}_z\symup{A}^{\ast} -> ^{m+1}_z \symup{A} + \upgamma }. \nonumber
\end{equation}
Mit der m der Massenzahl des Kern. Der hier entstandene Kern $\ce{^{m+1}_z \text{A}}$ ist meistens instabil, da er ein Neutron mehr hat als ein 
stabiler Kern gleicher Ordnungszahl. Aufgrund des emissierten $\upamma$-Quants ist der Kern jedoch nicht angeregt und somit relativ langlebig.
Durch Emission eines Elektrons wird der Kern wieder stabil.
\begin{equation}
   \ce{^{m+1}_z \symup{A} -> ^{m+1}_{z+1} \symup{C} + \upbeta^- + \symup{E_{kin}} + \bar{\symup{\nu}}_{\symup{e}}} \nonumber
\end{equation}

\noindent $\bar{\symup{\nu}}_{\symup{e}}$ ist hier ein Antineutrino. Dieses Antineutrino ensteht da die Masse des Kerns $\ce{^{m+1}_z \text{A}}$ 
groeßer als Massen der rechten Seite ist. Die restliche Masse ist nach der Beziehung

\begin{equation}
   \Delta E = \Delta m c^2 \nonumber
\end{equation}

\noindent in kinetische Energie von Elektron und Antineutrino umgewandelt.\\
Die Wahrscheinlichkeit, dass ein Kern ein Neutron einfängt wird als Wirkungsquerschnitt $\sigma$ bezeichnet. Dieser gibt ein Verhältnis der 
Einfangwahrscheinlichkeit in Abhängigkeit von gewissen Material Eigenschaften an und hat somit die Einheit einer Fläche.
\begin{equation}
   \sigma = \frac{u}{nKd} \nonumber
\end{equation}

\noindent Mit $u$ der Einfänge, $n$ der Neutronen pro Sekunde, $d$ der Dicke $K$ der Atomdichte der Folie. $\sigma$ wird angepasst an den 
Kernquerschnitt in $\si{\barn} \coloneqq \SI{10e-24}{\centi\metre\squared}$  angegeben. Da der Wirkungsquerschnitt stark von der kinetischen Energie 
der einfallenden Teilchen abhängt, wird hier zwischen schnellen und langsamen Neutronen unterschieden. Die Wellenlänge eines Neutrons 
berechnet sich mittels der De-Broglie-Wellenlänge:

\begin{equation}
   \lambda = \frac{h}{m_{\symup{n}}v}. \nonumber
\end{equation}

\noindent Hier ist $h$ das Planksche Wirkungsquantum, $m_n$ die Neutronenmasse und $v$ die Geschwindigkeit des Neutrons. Für große Geschwindigkeiten 
und somit kleine Wellenlänge im Verhältnis zum Kernradius R($\approx \SI{10e-12}{\centi\meter}$), lassen sich geometrische Überlegungen auf die 
wechselwirkungs Wahrscheinlichkeit anwenden. Es besteht eine Analogie zur optischen Streuung von Wellen and Verhältnismäßig großen Objekten. ­Für 
große Wellenlängen entstehn wieder wie in der Analogie zur Optik, Interferenzeffekte, somit sind geometrische Überlegungen unbrauchbar. Das 
Experiment zeigt eine Neutronengeschwindigkeit bei der, der Wirkungsquerschnitt deutliche groeßer als der geometrische Querschnitt. Dieser Effekt
entsteht, wenn die Neutronenenergie gleich der Differenz zweier Energienevaus im Zwischenkern. Die durch Breit und Wigner angegebene Formel:

\begin{equation}
   \sigma(E) = \sigma_0 \sqrt{ \frac{E_{\symup{r_i}}}{E} } \frac{\tilde{c}}{\left( E - E_{\symup{r_i}} \right)^2 + \tilde{c}} \nonumber
\end{equation}

\noindent beschreibt den Wirkungsquerschnitt in Abhängigkeit der Neutroenenergie $E$. Hier sind $\tilde{c}$ und $\sigma_0$ charakterische 
Konstanten und $E_{\symup{r_i}}$ die Energienevaus. Aus der Formel ist abzulesen, dass $E_{\symup{r_i}}$ für $E$=$ E_{\symup{r_i}}$ maximal wird.
Ist $E$ jedoch deutlich kleiner als $E_{\symup{r_i}$, kann $\left( E - E_{\symup{r_i}} \right)^2$ als Konstante angenommen werden. Dadurch ist:

\begin{equation}
   \sigma \sim \frac{1}{\sqrt{E}} \sim \frac{1}{v} . \nonumber
\end{equation}

\noindent Diese Antiproportionalität von $\sigma$ und $v$ deckt sich mit der Vorstellung, dass bei kleineren Geschwindigkeiten, das Neutron mehr 
Zeit hat um mit dem Kern zu wechselwirken.

\subsection{Erzeugung niederenergetischer Neutronen}

\begin{equation}
   \ce{ ^9_4 \symup{Be} + ^4_2$\alpha$ -> ^{12}_6\symup{C} + ^1_0\symup{n}} \nonumber
\end{equation}

\begin{equation}
   E_{symup{ü}}  = E_0 \frac{4Mn}{\left( M + m \right)^2} \nonumber
\end{equation}

\begin{equation}
   \ce{^{51} \symup{V}, ^{55}\text{Mn}, ^{79}\text{Br}, ^{115}\text{In}, ^{127}\text{J}, ^{164}\text{Dy}, ^{107}{Ag}, ^{109}\text{Ag}, ^{103}\text{Rh}} \nonumber
\end{equation}

\begin{equation}
   \ce{^{107}_{47}\text{Ag} + \symup{n} -> ^{108}_{47}\text{Ag} -> ^{108}_{48}\text{48} + $\upbeta$^- + \bar{\symup{\nu}}_{\symup{e}}} \nonumber
\end{equation}

\begin{equation}
   \ce{^{109}_{47}\text{Ag} + \symup{n} -> ^{110}_{47}\text{Ag} -> ^{110}_{48}\text{48} + $\upbeta$^- + \bar{\symup{\nu}}_{\symup{e}}} \nonumber
\end{equation}

\begin{equation}
   \begin{split}
      \ce{^{103}_{45}\text{Rh} + \symup{n}} \mathbin{\scalebox{2.5} \textbraceleft } \quad \nonumber
   \end{split}
   \begin{split}
      &\overset{10\%}{\longrightarrow} \ce{^{104\symup{i}}_{45}\text{Rh} -> ^{104}_{45}\text{Rh} + $\upgamma$ -> ^{104}_{46}\text{Pd} + $\upbeta$^- + \bar{\symup{\nu}}_{\symup{e}} } \nonumber\\
      &\overset{90\%}{\longrightarrow} \ce{^{104}_{45}\text{Rh} -> ^{104}_{46}\text{Pd} + $\upbeta$^- +\bar{\symup{\nu}}_{\symup{e}} } \nonumber\\
   \end{split}
\end{equation}

\begin{equation}
   N(t) = N_0 \symup{e}^{- \lambda t} \nonumber
\end{equation}

\begin{equation}
   \frac{N_0}{2} = N_0 \symup{e}^{- \lambda T}  \nonumber
\end{equation}

\begin{equation}
   T = \text{ln}\left(\frac{2}{\lambda}\right) \nonumber
\end{equation}

\begin{equation}
   N_{\Delta t}(t) = N(t) - N(t + \Delta t) \nonumber
\end{equation}

\begin{equation}
   N_{\Delta t}(t) = N_0 \symup{e}^{ - \lambda t} - N_0 \symup{e}^{- \lambda \left( t + \Delta t \right) } = N_0 \left( 1 - \symup{e}^{-\lambda \Delta t} \right) \symup{e}^{-\lambda t} \nonumber
\end{equation}

\begin{equation}
   \text{ln} \left( N_{\Delta t}(t)\right) = \text{ln} \left( N_0 \left( 1- \symup{e}^{-\lambda \Delta t}\right) -\lambda t \right) \nonumber
\end{equation}

\begin{equation}
   \text{ln}\left(N_0\left(1- \symup{e}^{-\lambda\Delta t}\right)  \right) \nonumber
\end{equation}

\begin{equation}
   N_{\Delta t_{l}} \coloneqq N_{0_l}\left( 1- \symup{e}^{-lambda_{l} \Delta t}\right) \symup{e}^{\lambda_l t} \nonumber
\end{equation}

\begin{equation}
   N_{\Delta t}(t_{\symup{i}}) - N_{\Delta t_l}(t_{\symup{i}}) \leq 0 \nonumber
\end{equation}

\begin{equation}
   N_{\Delta t} (t_{\symup{i}}) = N_{\Delta t , \text{gem}}(t_{\symup{i}}) - N_{\Delta t, \symup{u}} \nonumber
\end{equation}