\newpage
\section{Durchführung}

\begin{figure}[H]
    \centering
    \includegraphics[width=0.6\textwidth]{images/schaltplan.PNG}
    \caption{Schaltpskizze der Messapparatur \protect \cite{V703}.}
    \label{img:schalt}
\end{figure}
Die obige Abbildung \ref{img:schalt} zeigt den genauen Aufbau des Geiger-Müller-Zählrohrs. Dabei fließt die Ladung aus dem Zählrohr über den Widerstand $R$ ab, 
wird anschließend vom Kondensator $C$ entkoppelt um dann vom vor dem Zählen durch den Zähler noch verstärkt zu werden. Alternativ kann das ganze auch auf einem Oszilloskop dargestellt werden.\\\\

\subsection{Aufnahme der Charakteristik des Zählrohres}


\noindent
In diesem Versuchsabschnitt wird eine $\beta$-Strahlungsquelle vor dem Fenster des Zählrohrs platziert.\\
Nun wird für verschiedene Spannungen die Zählrate gemessen.\\
Dabei müssen mehrere Sachen beachtet werden.\\
Zum einen sollte nicht wesentlich mehr Impulse als $\SI{100}{\per\second}$ gemessen werden, da sonst aufgrund von zu hohen Totzeiten Korrekturen vorgenommen werden müssten.
Besonders sollte hier auch auf die Genauigkeit der Messungen geachtet werden, da das Plateau, dessen Steigung bestimmt werden soll, nur einen schwachen Anstieg besitzt.
Des Weiteren sollte das Messintervall $\approx\SI{120}{\second}$ betragen, damit der Fehler der Messpunkte, aufgrund des $\sqrt{n}$-Gesetzes, $<\SI{1}{\percent}$ ist.\\
Zuletzt sollte die angelegte Spannung $<\SI{700}{\volt}$ ist, damit Selbstentladung vermieden wird.\\\\

\subsection{Sichtbarmachung von Nachentladungen}


\noindent
Anschließend sollen, mit Hilfe des Oszilloskops, die Nachentladungen qualitativ nachgewiesen werden.\\
Dafür wird die Intensität der $\beta$-Strahlungsquelle so weit heruntergeregelt, dass auf dem Oszillioskop, bei einer Ablenkgeschwindigkeit von $\SI{50}{\micro\second\per\centi\metre}$, 
nur noch ein einziger Impuls gemessen wird. \\
Die angelegte Spannung sollte dabei etwa $\SI{350}{\volt}$ betragen, da für diese Spannung Nachentladungen vernachlässigbar sind.\\
Nun wird die Spannung schlagartig auf $\SI{700}{\volt}$ hochgeregelt und die Beobachtung notiert, so wie der zeitliche Abstand zwischen Primär- und Nachentladung gemessen.

\subsection{Oszillographische Messung der Totzeit}


\noindent
Für das folgende wird die Intensität der Quelle wieder erhöht.
Nun wird zur Messung der Totzeit mit Hilfe des Oszilloskops auf den Anstieg des Impulses getriggert.
Anschließend lässt sich die Totzeit durch den Vergleich mit Abb. \ref{img:tot} einfach ablesen. Die Erholungszeit lässt sich hingegen nur abschätzen.

\subsection{Bestimmung der Totzeit mit der Zwei-Quellen-Methode}


\noindent
Die Messmethode der Totzeit mittels des Oszillioskops ist ungenau, weswegen diese Methode eine Alternative bietet.\\\\
Die Totzeit führt dazu, dass die regristrierte Impulsrate $N_r$ immer kleiner ist als die der tatsächlich absorbierten Teilchen $N_w$.
Da das Zählrohr für den Bruchteil $T\cdot N_r $ unempfindlich ist lässt sich die tatsächliche Impulsrate der tatsächlich absorbierten Teilchen mit der folgenden Gleichung bestimmen:
\begin{equation}
    N_w=\frac{\text{Impulsrate}}{\text{Messzeit}}=\frac{N_r t}{(1-T \cdot N_r)t}=\frac{N_r }{(1-T \cdot N_r)}
    \label{eqn:imp}
\end{equation}
Wenn nun erst die Zählrate $N_1$ eines Präparats, dann die Zählrate $N_1$ und die eines zusätzlichen anderen zweiten Präparat $N_2$ und zuletzt nur $N_2$, gemessen werden, lässt sich damit die Totzeit berechnen.\\
Dabei wird genau diese Reihenfolge gewählt um unnötiges Bewegen  der Präparate zu vermeiden.\\\\
Aufgrund der Totzeit gilt die Formel
\begin{equation*}
    N_\text{1+2}<N_1 +N_2
\end{equation*}
und nicht
\begin{equation*}
    N_\text{1+2}=N_1 +N_2
\end{equation*}
Nun ergibt sich aus der Formel \refeq{eqn:imp} für die drei unterschiedlichen Messaufbauten die drei folgenden Zusammenhänge:
\begin{align*}
    N_{w_1} &=\frac{N_1 }{(1-T \cdot N_1)}\\
    N_{w_2} &=\frac{N_2 }{(1-T \cdot N_2)}\\
    N_{w_{1+2}} &=\frac{N_{1+2} }{(1-T \cdot N_{1+2})}
\end{align*}
Hieraus lässt sich nun für $T^2\cdot N^2 \ll 1$ mit $i=(1,2,1+2)$ und mit $N_\text{1+2}=N_1 +N_2$ die folgende Gleichung für $T$ aufstellen:
\begin{equation*}
    T\approx \frac{N_1 +N_2 -N_{{1+2}} }{2 N_1 N_2}
\end{equation*}

\subsection{Messung der pro Teilchen vom Zählrohr freigesetzten Ladungsmenge}


\noindent
Der mittlere Zählstrom kann, wie in Abb. \ref{img:schalt}, mit einem empfindlichen Amperemeter, bei hohen Zählraten, gemessen werden.\\
Die dazugehörige Gleichung ist folgende:
\begin{equation*}
    \overline{\symup{I}} \coloneq \frac{1}{\tau} \int_0^\tau \frac{U(t)}{R} dt
\end{equation*}
Dabei gilt $\tau \gg T$.\\
mit HIlfe von $ \overline{\symup{I}} $ lässt sich bei bekannter Impulszahl pro Zeit und bei bekannter eindringender Teilchenmenge die freigesetzte Ladung bestimmen.
Daraus lässt sich widerum die folgende Formel aufstellen, wobei $Z$ die Anzahl der Teilchen pro $\increment t$ ist:
\begin{equation*}
    \overline{\symup{I}} =\frac{\increment Q}{\increment t} \cdot Z
\end{equation*}
Dabei hängt $\increment Q$ auch noch von der Spannung ab, weswegen es in der Abhängigkeit von ihr untersucht werden sollte.