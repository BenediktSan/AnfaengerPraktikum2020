\newpage
\section{Durchführung}

\begin{figure}[H]
    \centering
    \includegraphics[width=0.6\textwidth]{images/schaltplan.PNG}
    \caption{Schaltpskizze der Messapparatur \protect \cite{V703}.}
    \label{img:schalt}
\end{figure}
Die obige Abbildung \ref{img:schalt} zeigt den genauen Aufbau des Geiger-Müller-Zählrohrs. Dabei fließt die Ladung aus dem Zählrohr über den Widerstand $R$ ab, 
wird anschließend vom Kondensator $C$ entkoppelt umd dann vom vorm zählen durch den Zähler noch verstärkt zu werden. Alternativ kann das ganze auch auf einem Oszilloskop dargestellt werden.\\\\

\subsection{Aufnahme der Charakteristik des Zählrohres}

\noindent
In diesem Versuchsabschnitt wird eine $\beta$-Strahlungsquelle vor dem Fenster des Zählrohrs platziert.\\
Nun wird für verschiedene Spannungen die Zählrate gemessen.\\
Dabei müssen mehrere Sachen beachtet werden.\\
Zum einen sollte nicht wesentlich mehr Impulse als $\SI{100}{\per\second}$ gemessen werden, da sonst aufgrund von zu hohen Totzeiten Korrekturen vorgenommen werden müssten.
Besonders sollte hier auch auf die Genauigkeit der Messungen geachtet werden, da das Plateau, dessen Steigung gemessen wird, nur einen schwachen Anstieg besitzt.
Des Weiteren sollte das Messintervall $\approx\SI{120}{\second}$ betragen, damit der Fehler der Messpunkte, aufgrund des $\sqrt{n}$-Gesetzes, $<\SI{1}{\percent}$ beträgt.\\
Zuletzt sollte die angelegte Spannung $<\SI{700}{\volt}$ ist, damit die Selbstentladung vermieden wird.\\\\

\subsection{Sichtbarmachung von Nachentladungen}

\noindent
Anschließend sollen, mit Hilfe des Oszilloskops, die Nachentladungen qualitativ nachgewiesen werden.\\
Dafür wird die Intensität der $\beta$-Strahlungsquelle so weit heruntergeregelt, dass man auf dem Oszillioskop, bei einer Ablenkgeschwindigkeit von $\SI{50}{\micro\second\per\centi\metre}$, 
nur noch einen einzigen Impuls misst. \\
Die angelegte Spannung sollte dabei etwa $\SI{350}{\volt}$ betragen, da dort Nachentladungen vernachlässigbar sind.\\
Nun wird die Spannung schlagartig auf $\SI{700}{\volt}$ hochgeregelt und die Beobachtung notiert, so wie der zeitliche Abstand zwischen Primär- und Nachentladung gemessen.

\subsection{Oszillographische Messung der Totzeit}

Für das folgende wird die Intensität der Quelle wieder erhöht.
Nun wird zur Messung der Totzeit mit Hilfe des Oszilloskops auf den Anstieg des Impulses getriggert.
Nun lässt sich die Totzeit durch den Vergleich mit Abb. \ref{img:tot} einfach die Zeit ablesen. Die Erholungszeit lässt sich hingegen nur abschätzen.

\subsection{Bestimmung der Totzeit mit der Zwei-Quellen-Methode}

