\section{Diskussion}
\subsection{Messungen bis 1 Bar}

Für diesen Teil des Versuches gab es bei der Durchführung keine größeren Probleme. Der Startdruck fiel am Anfang auf einen 
zufriedenstellenden Wert und bis auf minimale Druckschwankungen beim Erhitzen gab es keine Besonderheiten. Nur bei den ersten
drei Messpaaren hatte das Wasser noch nicht gesiedet. Somit ist nicht sicher ob der Sättigungsdampfdruck bereits vollständig erreicht war.
Diese Werte müssen mit etwas Vorsicht betrachtet werden.\\
Jedoch scheinen auch die ersten drei Messwerte gut in die Reihe der anderen Messwerte zu passen.
\begin{table}[H]
    \centering
    %\caption{Die Messwerte}
    \begin{tabular}{ S [table-format=3.0] S [table-format=5.0] }
        \toprule
        {$T \mathbin{\scalebox{1.5}/} \si{\celsius}$} & {$L_\text{Theorie} \mathbin{\scalebox{1.5} /} \si{\joule\per\mole}$}\\
        \midrule
        25 & 43990\\
        40 & 43350\\
        60 & 42483\\
        80 & 41585\\
        100& 40657\\
        120& 39684\\
        \bottomrule
    \end{tabular}
\caption{Eine Tabelle der Theoriewerte\protect \cite{Chemie-Schule.de-Verdampfungswärme} für $L$.}
\label{tab:theo1bar}
\end{table}
\noindent
Der Mittelwert der aus den oben aufgeführten Theoriewerten für die Verdampfungswärme ist: $\SI{41958.17}{\joule\per\mol}$\\
Dies führt für unseren Wert von $L=\SI{36.3578843(6667737)e3}{\joule\per\mol}$ zu einer
prozentualen Abweichung von $\SI{1.54(21)e1}{\percent}$.\\
Diese Abweichung und damit der gemittelte Wert von $L$, der aus der Messreihe bestimmt wurde, ist zufriedenstellend.\\
Da sich  die innere Verdampfungswärme pro Molekül, $L_i=\SI{34.47(69)e-2}{\electronvolt}$ ziemlich direkt, mit nur einer kleinen Näherung, aus dem oben genannten Wert berechnet, 
muss ihr Wert auch zufriedenstellend sein.\\ 
Allerdings ist der Wert kleiner als die Theoriewerte.
Er wurde aber auch mithilfe von Näherungen bestimmt und er ist außerdem auch nur ein gemittelter Wert.
Trotzdem könnte sich dies dadurch erklären lassen, dass die Temperatur nicht über dem Wasser, wie es optimal wäre, sondern im Wasser gemessen wurde.
Dies kann zu leicht erhhöhten Messwerten geführt haben, was wiederum zu einer kleineren Verdampfungswärme führen sollte.\\
Zusammenfassend kann aber gesagt werden, dass die Messwerte recht gut sind.



\subsection{Messungen zwischen 1 und 15 Bar}

Die Durchführung des Versuches lief ohne irgendwelche direkt erkennbaren Probleme ab.\\
Bei der Auswertung wurde die in \ref{img:plus} zusehende Funktion der in \ref{img:minus} vorgezogen.
Dies liegt, wie zuvor schon erläutert, daran, dass die gesuchte Funktion eine negative Steigung besitzen sollte und Werte in der Größenordnung der Theoriewerte besitzen sollte.\\
Beides trifft auf die Erste zu, aber nicht auf die Zweite.\\
Allerdings gibt es, wie in \ref{img:plus} und in \ref{tab:Lvergleich} sichtbar ist, für die ersten Messwerte der gewählten Funktion eine recht hohe Diskrepanz zwischen den Theoriewerten und unserer Funktion für $L$.\\
Diese Diskrepanz beträgt am Anfang sogar $80 \si{\percent}$, am Ende aber nur $ 8\si{\percent}$.\\
Dies könnte daran liegen, dass sich der Gleichgewichtszustand, welcher für den Sättigungsdampfdruck benötigt wird, sich später einstellt.
Dies würde bedeuten, dass es also ein Problem bei der Durchführung gab, das nicht direkt ersichtlich war.\\
Alternativ könnte für die starke Abweichung auch der große Fehler der Fit-Funktion mit verantwortlich sein, obwohl sie eigentlich sehr gut auf 
die Messwerte passt. Eine zusätzlicher Faktor wären auch noch die vielen Näherungen die getroffen werden mussten um die Clausius-Clapeyronsche Gleichung lösen zu können
und um die Funktionen für $L$ bilden zu können.\\
Allerdings nähert sich die Funktion für die späteren Messwerte gegen die Theoriewerte an, was ein Indikator dafür seien kann, dass kein grundlegender Fehler vorliegt.\\
Die in \ref{tab:Lvergleich} zusehenden Werte sind aber alles in allem zufriedenstellend.

