\section{Diskussion}
Die Abweichung der theoretischen Werte, von den aus den Messwerten errechneten, ist für fast alle Werte recht groß.
Für die Berechnung des effektiven Widerstandes und des aperiodischen Grenzfall kann
das ganze noch mit dem \enquote{defekten} Nadelimpulsgenerator erklärt werden.\\\\
\noindent
Bei dem effektiven Widerstand wurde, Generatorinnenwiderstand außen vor gelassen, eine Abweichung von $\approx \SI{300}{\ohm}$ errechnet.
Der Wert für den Generatorinnenwiderstand ist so viel größer als der zu Erwartende. Zurückführen lässt sich dies auf das schnelle Abklingen der Schwingung.
Dies könnte eine Folge des Generators sein.\\
Der experimentell bestimmte Wert für den Widerstand bei dem der aperiodische Grenzfall auftritt
zeigt eine Abweichung von $\approx \SI{66}{\percent}$ vom theoretischen Wert.\\
Da dieser allerdings nur von den Bauteildaten des Schwingkreises abhängt und 
dieser Widerstand nicht direkt von der Spannung oder Frequenz abhängt ist es eher unwahrscheinlich
dass der Generator dafür verantwortlich ist.\\
Diese Abweichung ließe sich eher durch die nicht beachteten Widerstände der anderen Bauteile oder durch, durch Alterung erzeugte Abweichung von den angegebenen Bauteildaten, erklären.\\\\
\noindent
Bei dem Faktor der Resonanzüberhöhung gab es auch wieder eine hohe Diskrepanz zwischen dem theoretischen und dem experimentell bestimmten Wert.
Die Abweichung bewegte sich hier in einer Größenordnung von  $\approx \SI{85}{\percent}$.\\
Da die Messreihe, die für diesen Wert benötigt wurde, zur Verfügung gestellt wurde kann hier aus Unkenntnis nicht zu möglichen experimentellen Ursachen gesagt werden.
Hier könnte aber auch wieder mit abweichenden Bauteildaten argumentiert werden.\\
Dagegen spricht allerdings, dass bei der Bestimmung der Halbwertsbreite für diese Messreihe nur eine Abweichung von $\approx \SI{2}{\percent}$ aufgetreten ist.\\
Die Form der Messreihe, wie sie in Abb. \ref{img:quad} sichtbar ist, ist also sehr gut, nur das Maximum der normierten Spannung sollte höher liegen.\\\\
\noindent
Bei der Phasenverschiebung weichen alle experimentell ermittelten Werte in einer Größnordnung von $\approx \SI{78}{\percent}$ ab.\\
Die Form der Messwertverteilung und die Verteilung der charakteristischen Punkte, wie in \ref{img:plot4} dargestellt, entspricht dabei der zu erwartenden Form.
Die ermittelten Werte sind im Vergleich mit den theoretischen fast um einen Faktor $10$ zu klein.\\
Dies könnte, da die generelle Form stimmt, auf einen Fehler beim Messen oder eine nicht beachtete Dämpfung, 
wie einen nicht beachteten Widerstand, zurück führbar sein.\\\\
\noindent
Alles in allem sind die Messwerte aber sehr solide, da sie trotz großer Abweichung von den theoretischen Werten, 
immer sehr gut die untersuchten Phänomene darstellen.
\newpage

