\section{Diskussion}
    In diesem Versuch lässt sich nur der Faktor $\frac{\symup{h}}{\symup{e_0}}$ mit Theorie Werten vergleichen.
    Unser berechneter Wert $\num{20525(6630)}$ weicht von dem Theoriewert von 25812,81 um $\num{21 (2)}$ \% ab.
    Mit dieser Abweichung war zu rechnen da wir nur 3 Messungen der Grenzspannung gemacht hatten und diese bereits sehr ungenau erscheinen.
    Die Ausgleichsgerade beschreibt die Punkte auch nicht wirklich sehr gut, dies liest sich bereits aus der Ungenauigkeit von fast 30\% in der 
    Steigung der Geraden ab. Die Ungenauigkeit der Messung enstand wahrscheinlich dadurch, dass beim Vermessen kleine Änderungen an dem 
    Versuchsaufbau gemacht wurden und somit auch die Relationen der Messreihen verunreinigt wurden. 
    Der Wert der Austrittsarbeit von $\SI{1.2(6)}{\electronvolt}$ hat aus den gleichen Gründen wie $\frac{\symup{h}}{\symup{e_0}}$ einen sehr 
    großen Fehler ist jedoch in der richtigen Größenordnung.