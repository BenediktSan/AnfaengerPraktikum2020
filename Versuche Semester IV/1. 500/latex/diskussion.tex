\section{Diskussion}
    In diesem Versuch lässt sich nur der Faktor $\frac{\symup{h}}{\symup{e_0}}$ mit Theoriewerten vergleichen.
    Der berechnete Wert $\num{20525(6630)}$ weicht von dem Theoriewert 25812,81 um $\num{21 (2)}$ \% ab.\\
    Mit dieser Abweichung war zu rechnen da nur 3 Messungen der Grenzspannung gemacht worden sind und diese bereits sehr ungenau erschienen.
    Die Ausgleichsgerade beschreibt die Punkte auch nicht wirklich gut. Dies lässt sichaus der Ungenauigkeit von fast 30\% in der 
    Steigung der Geraden herauslesen.\\
    Die Ungenauigkeit der Messung entstand wahrscheinlich dadurch, dass beim Vermessen kleine Änderungen an dem 
    Versuchsaufbau gemacht wurden und somit auch die Relationen der Messreihen verunreinigt wurden. \\
    Diese Änderungen wurden vorgenommen, da vorher keine vernünftigen Messungen möglich waren.\\
    Der Wert der Austrittsarbeit von $\SI{1.2(6)}{\electronvolt}$ hat aus den gleichen Gründen wie $\frac{\symup{h}}{\symup{e_0}}$ einen sehr 
    großen Fehler ist jedoch in der richtigen Größenordnung.