\newpage
\section{Diskussion}

\noindent
Dieser Versuch lief alles in allem Recht gut. 
Die Messwerte konnten alle sehr genau aufgenommen werden, auch wenn am Ende immer ein Hintergrundrauschen übrig geblieben ist.\\
Es gab bei der Durchführung zwar ein paar Probleme mit nicht durchführbaren Messungen, diese konnten aber durch rumprobieren mit anderen Bauteilen behoben werden.
Dies ist auch der Grund warum bei der Maxwell-Brücke und der Induktivitätsmessbrücke nicht das selbe Bauteil untersucht werden konnten.\\
Die relative Abweichung von den Theoriewerten ist in Tabelle \ref{tab:rel} zu finden.\\
Die zu den Messwerten gehörigen Theoriewerte wurden schon in den Auswertungen, der einzelnen Brückenschaltungen aufgeführt.



\begin{table}[ht]
    \centering
    \caption{Relative Aweichung von den Theoriewerten für die einzelnen Messungen}
    \label{tab:rel}
        \begin{tabular}{S [table-format=4.0] S [table-format=2.0] S [table-format=1.4]}
        \toprule
            {$\text{Brückenart}$} &{Messgröße}& {$\text{relative Abweichung von der Theorie} $} \\
        \midrule
            $\text{Wheatstone}$  & $R_{1}$ & $\SI{0.0023}{\percent}$\\
                                 & $R_{2}$ & $\SI{0.75}{\percent}$\\
            \hline
            $\text{Kapazität}$  & $R$ & $\SI{-0.0027}{\percent}$\\
                                & $C$ & $\SI{-0.41}{\percent}$\\
            \hline
            $\text{Maxwell}$ & $R$&     $\SI{7.72}{\percent}$\\
                       & $L$&     $\SI{5.07}{\percent}$\\
            \hline
        \bottomrule
    \end{tabular}
\end{table} 

\noindent Die relativen Abweichungen sind alle sehr gering, was ein Indikator dafür ist, dass die Messinstrumente sehr genau gearbeitet haben und die Messungen ohne große Störungen durchgeführt werden konnten.\\
Alles in allem lieferte die Durchführung also sehr gute Messwerte.