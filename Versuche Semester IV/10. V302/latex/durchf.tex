\section{Durchführung}

    \noindent Im folgenden wird die Benutztung der unterschiedlichen Brückenschaltungen zur Vermessung der elektrischen Bauteile erklärt. Die 
    nicht frequenzabhängigen Schaltungen werden mit einer Sinus-Spannung von $\SI{1000}{\hertz}$ betrieben, die Brückenspanungen werden mittels 
    eines Oszilloskops gemessen.

    \subsection{Wheatstonesche Brücke}

        \noindent Da hier nur der Quotient $\frac{R_3}{R_4}$ benötigt wird, wird als Widerstand für $R_3$ und $R_4$ ein Potentiometer mit 
        einem Gesamtwiderstand von $\SI{1}{\kilo\ohm}$ benutzt. An diesem ist der Widerstand von $R_3$ abzulesen und $R_4$ kann einfach 
        aus der Differenz zu 1000 berechnet werden. Das Potentiometer wird auch in den folgenden Schaltungen als veränderlicher Widerstand 
        benutzt. Durch variieren des Quotienten an dem Potentiometer wird nun die Brückenspannung minimiert, aus dem Minimum lässt sich dann 
        $R_x$ berechnen. Dieser Vorgang wird noch einmal mit einem anderem Widerstand $R_2$ wiederholt.

    \subsection{Kapazitätsmessbrücke}

        \noindent Die Kapazitätsmessbrücke wird nach dem Schaltbild mit dem Referenz Kondensator $C_2$ aufgebaut. Der Kondensator mit der 
        Kapazität $C_x$ und dem Innenwidestand $R_x$ werden bestimmt, indem abwechselnd die beiden Potentiometer verändert werden um die 
        Brückenspannung zu minimieren. Diese Messreihe wird dann erneut mit einer anderen Referenz Kapazität $C_2$ durchgeführt.

    \subsection{Induktivitätsmessbrücke}

        \noindent Die Induktivitätsmessbrücke wird auch wieder nach dem Schaltbild aufgebaut, das Referenzbauteil ist nun die Spule mit 
        der Induktivität $L_2$. Die restlichen Messungen werden analog zur Kapazitätsmessbrücke durchgeführt.

    \subsection{Maxwell-Brücke}

        \noindent Die Induktivität wird ebenfalls mit der Maxwell-Brücke vermessen, dazu wird die Schaltung nach dem Schaltbild aufgebaut und 
        eine Messreihe analog zu der der Induktivitätsmessbrücke durchgeführt.

    \subsection{Wien-Robinson-Brücke}

        \noindent Auch die Wien-Robinson-Brücke kann einfach nach dem Schaltbild aufgebaut werden. Nun wird jedoch nicht mehr der Wiederstand variiert, 
        sondern die Frequenz der anliegenden Quellspannung. Die Frequenz wird hier in einem Bereich zwischen von 20-$\SI{10000}{\hertz}$ variiert. Es ist 
        darauf zu achten, dass die die meisten Messwerte um die Resonanzfrequenz $\omega_0$ aufgenommen werden. Diese wird mit den in der 
        Schaltung verwendeten Widerständen und Kapazitäten berechnet.

