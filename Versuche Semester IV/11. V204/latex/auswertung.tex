\section{Auswertung}

    \noindent Die Messung der Wärmeleitung wird in diesem Versuch mit zwei unterschiedlichen Methoden durchgeführt, die folgende Auswertung ist 
    in zwei Teilen eingeteilt in denen dann diese Methoden ausgewertet werden. Sämtliche in diesem Versuch aufgenommene Daten können nicht im 
    Protokoll dargestellt werden da bereits einzelne Messreihen 

        \subsection{Statische Methode}

        \begin{figure}[H]
            \centering
            \includegraphics{build/plots/stat_plot.pdf}
            \caption{Die Temperaturen an den fernen Messpunkte der einzelnen Stäbe bei der statischen Messung. }
            \label{fig:plot_stat}
        \end{figure}

        Im Abbildung(\raf{fig:plot_stat}) sind die Temperaturen $T1, T4, T5$ und $T8$ geplottet, diese Stellen die vom Peltier-Element ferner 
        liegenden Messtellen dar. Edelstahl startet bei der niedrigsten Temperatur mit ca $\SI{26}{\celsius}$, die beiden Messing Stäbe starten 
        bei ca. $\SI{29}{\celsius}$ und Aluminium startet am höchsten bei $\SI{31}{\celsius}$. Die Temperaturverläufe 
        lassen sich nun in unterschiedliche Phasen einteilen. Zunächst ändern sich die Temperaturen kaum, die Energie des 
        Heitzgerätes ist noch nicht an dem hinteren Messgerät angekommen. Dieser Bereich ist bei Messing und 
        Aluminium ungefähr gleich lang, dauert jedoch bei Edelstahl ca. 3 mal so lang. Nun beginnt die Phase der 
        höchsten Steigung, hier hat Aluminium die höchste Steigung, die beiden Messing Stäbe haben zunächst die 
        gleiche Steigung, sie ist nur etwas kleiner als die die von Aluminium. Die Steigung von Aluminium ist 
        deutlich kleiner als der anderen Metalle. Die Phase des hohen Temperatur anstiegs hält bei Aluminium auch am länsten,
        bei den Messingen Stäben fällt die Steigung bei dem schmalen früher wieder ab. Die Phase der hohen Steigung 
        hält bei Edelstahl nur sehr kurz. Zuletzt steigen alle Metalle mit einer relativ kleinen aber ähnlichen Steigung konstant 
        weiter an.

        Die Temperatur der jeweiligen Stäbe zum Zeitpunkt $t=\SI{700}{\second}$ lassen sich in den Messdaten ablesen und ergeben sich zu:
        \begin{align*}
          \text{für den breiten Messingstab:} \: T&= \SI{54.02}{\celsius}\\ 
          \text{für den schmalen Messingstab:} \: T&= \SI{51.07}{\celsius}\\ 
          \text{für den Aluminiumstab:} \: T&= \SI{56.46}{\celsius}\\ 
          \text{für den Edelstahlstab:} \: T&= \SI{41.87}{\celsius}
        \end{align*}

        % Hier fehlt noch eine Formel und der Zusammenhang


        \begin{table}
            \centering
            \caption{Der Wärmestrom der verschiedenen Metallstäben zu 5 verschiedenen Zeitpunkten.}
            \label{tab:Wärmestrom}
            \begin{tabular}{S[table-format=3.0] %Zeit
                            S[table-format=1.3] %Messing breit
                            S[table-format=1.3] %Messing schmal
                            S[table-format=1.3] %ALuminium
                            S[table-format=1.3] %Edelstahl
                            }
            \toprule
            &\multicolumn{4}{c}{$\frac{\Delta Q}{\Delta t} [\si{\joule\per\second}]$}\\
            \cmidrule(lr){2-5}
            {$ t [\si{\second}]$}&{Messing (breit)}&{Messing (schmal)}&{Aluminium}&{Edelstahl}\\
            \midrule
            100 & 0.466 & 0.359 & 0.796 & 0.125 \\
            200 & 0.710 & 0.532 & 1.440 & 0.131 \\
            350 & 1.082 & 0.780 & 2.089 & 0.156 \\
            450 & 1.267 & 0.906 & 2.320 & 0.177 \\
            600 & 1.459 & 1.034 & 2.491 & 0.212 \\
            \bottomrule
            \end{tabular}
        \end{table}

        \noindent Abschließend werden die Temperaturdifferenzen zwischen den Temperaturmessstellen eines Stabes für den breiten Messingstab und den 
        Edelstahlstab in einem $t$-$T$- Diagramm aufgetragen.

        \begin{figure}[H]
          \centering
          \includegraphics[width=\textwidth]{build/plots/plot_diff.pdf}
          \caption{Die Temperaturdifferenzen zwischen den Messstellen auf einem Stab.}
          \label{fig:plot_diff}
        \end{figure}

        In der Abbildung(\ref{fig:plot_diff}) fällt auf, dass bei beiden Stoffen der Temperaturunterschied tief startet und dann stark ansteigt.
        Die Temperaturdifferenz innerhalb des breiten Messingstabes hat bei etwa $ t = \SI{50}{\second}$ seinen Hochpunkt von $\SI{8.3}{\celsius}$ erreicht.
        Dann fällt die Kurve, flacht ab und nähert sich einem Wert um die $\SI{2}{\celsius}$ an.
        Die Temperaturdifferenz des Edelstahlstabes steigt höher als die des breiten Messingstabes bis sie ihren Hochpunkt 
        bei etwa $\SI{160}{\second} $ und $\SI{17}{\celsius}$ erreicht hat.
        Die Kurve fällt daraufhin asymtotisch auf den Wert $\SI{9.5}{\celsius}$.

    \subsection{Dynamische Methode}









