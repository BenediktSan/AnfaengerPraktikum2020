\section{Diskussion}

    \noindent Zunächst folgt die Diskussion der statischen Methode.\\ Hier wurde neben dem Wärmefluss, die Temperatur nach 700s betrachtet.
    Die Daten hierzu sind in den Gleichungen \ref{eqn:M1} bis \ref{eqn:E1} dargestellt. Aus diesen Daten lässt sich bestimmen, dass Aluminium die höchste Wärmeleitfähigkeit besitzt.
    Messing hat die nächst höhere Wärmeleitfähigkeit.\\ Die in den Messwerten zu findende höhere Temperatur des breiteren 
    Messingstabes lässt sich darauf zurück führen, dass der Wärmefluss neben der spezifischen Wärmeleitfähigkeit auch von der 
    Querschnittsfläche abhängt. Zuletzt kommt Edelstahl mit der geringsten Wärmeleitfähigkeit. Diese Beobachtungen stimmen mit den Werten aus 
    der Literatur\cite{leit} überein.\\

    \noindent Der Verlauf der Temperaturdifferenz aus Abbildung(\ref{fig:plot_diff}) ist damit zu erklären, dass die Temperatur des Heizelements 
    als erstes bei dem näherem Messelement ankommt und dieses somit als erstes aufheizt. Sobald die Energie dann bei dem zweiten Messelement 
    angekommen ist, ist das Maximum der Temperaturdifferenz erreicht.\\ Die Differenz nähert sich dann einem Wert an, weil die beiden Messstellen 
    konstant aufgewärmt werden.\\

    \noindent Mittels der dynamischen Methode wurden Werte für die Wärmeleitfähigkeit berechnet, diese können nun mit den Theorie Werten verglichen 
    werden. Dies erfolgt in Tabelle(\ref{tab:diskussion}).

    \begin{table}[H]
        \centering
        \caption{Vergleich der im Experiment ermittelten Größen mit den damit korrespondierenden Literaturwerten.}
        \label{tab:diskussion}
        \begin{tabular}{S[table-format=15]
                        S[table-format=3.3]
                        S[table-format=6]
                        S[table-format=1.3]}
        \toprule
        {Metall}&{$\kappa_{\text{Messung}}\; \;[\si{\watt\per\metre\per\kelvin}]$}&{$\kappa_{\text{Literatur}} \; \;[\si{\watt\per\metre\per\kelvin}]$ \cite{leit}}&{$\text{Abweichung} \; \;[\si{\percent}]$}\\
        \midrule
        \text{Messing (breit)} & 117.264 & 120 & 2.280 \\
        \text{Aluminium}& 231.646 & 237 & 2.259 \\
        \text{Edelstahl}& 16.000 & 15\text{ bis }21\\
        \bottomrule 
        \end{tabular}
    \end{table}

    \noindent Aus den geringen Abweichungen lässt sich schließen, dass das Experiment insgesamt gut geklappt hat.
    

