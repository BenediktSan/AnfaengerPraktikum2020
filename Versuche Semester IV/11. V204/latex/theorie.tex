\section{Zielsetzung}
Das Ziel dieses Versuchs ist es die Wärmeleitfähigkeit diverser Metalle zu untersuchen.\\
Dabei wird eine statische und einde dynamische Methode angewandt.

\section{Theoretische Grundlagen}
Befindet sich ein Körper nicht im Temperaturgleichgewicht entsteht ein Wärmestrom.
Dieser transportiert Wärme entlang des negativen Temperaturgradienten.\\
Dies kann über Konvektion, Wärmestrahlung oder Wärmeleitung geschehen. Letzteres beinhaltet dabei den Wärmetransport über Phononen und über freie Elektronen.\\
In diesem Versuch wird dabei von allem genannten nur der Wärmetransport über die Elektronen betrachtet.\\
Wird nun ein Stab, dessen eines Ende wärmer ist, mit der Länge $L$, der Querschnittsfläche $A$ und der Dichte $\rho$ betrachtet, ergibt sich für die transportierte Wärmemenge  
\begin{equation}
    \symup{d}Q=-\kappa A \frac{\partial T}{\partial x} \symup{d} t \; \; .
\end{equation}
Dabei ist $\kappa$ die materialabhängige Wärmeleitfähigkeit. Aus dieser Gleichung lässt sich auch die Richtung des Wärmeflusses ablesen. 
Dieser findet, wie zuvor genannt, von warm zu kalt statt.\\
Für die Dichte des Wärmestroms $j_W$ gilt damit
\begin{equation*}
    j_W=- \kappa \frac{\partial T}{\partial x} \; \; .
\end{equation*}
Über die Kontinuitätsgleichung lässt sich hiermit dann folgende Gleichung, die eindimensionale Wärmeleitungsgleichung, herleiten
\begin{equation*}
    \frac{\partial T}{\partial t} = \frac{\kappa}{\rho c} \frac{\partial ^2 T}{\partial x ^2} \; \; .
\end{equation*}
Mit dieser Gleichung lässt sich die zeitliche und räumliche Entwicklung der Temperatur beschreiben. Dabei wird $\sigma_T= \frac{\kappa}{\rho c}$ auch Temperaturleitfähigkeit genannt.\\\\

\noindent Die Lösung der Wärmeleitungsgleichung für mit der Periode $T$ wechselnden Stabendentemperaturen ist die einer gedämpften Welle.\\
Diese besitzt die Gleichung
\begin{equation*}
    T(x,t)=T_{max} \symup{e}^{-\sqrt{\frac{\omega \rho c}{2 \kappa}}x} \cos\left( \omega T- \sqrt{\frac{\omega \rho c}{2 \kappa}} x \right) \; \; .
\end{equation*}
Die Phasengeschwindigkeit dieser eindimensionalen Welle $v$ ergibt sich zu
\begin{equation*}
    v= \frac{\omega}{k}= \frac{\omega}{\sqrt{\frac{\omega \rho c}{2 \kappa}}}=\sqrt{\frac{2 \kappa \omega}{\rho c}} \; \; . 
\end{equation*}
Für die Wärmeleitfähigkeit lässt sich dann weiter mit $\omega = \frac{2 \pi}{T}$ und der Phase $\phi=\frac{2 \pi \increment t}{T}$ der Term 
\begin{equation*} 
    \kappa =\frac{\rho c(\increment x)^2}{2 \increment t \ln{\left(\frac{A_{nah}}{A_{fern}}\right)}}
\end{equation*}
finden. Dabei lässt sich über die Amplituden in einem nahen und einem fernen Messpunkt $A_{nah}$ und $A_{fern}$ ein Dämpfungsterm bilden.
Der Abstand zwischen diesen Messpunkten ist $\increment x$.\\
Die Phasendifferenz der Welle zwischen den Messpunkten entspricht $\increment t$.