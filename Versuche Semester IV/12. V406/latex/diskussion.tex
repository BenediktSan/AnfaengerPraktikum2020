\section{Diskussion}

\noindent Die Aufnahme der Messwerte lief alles in allem gut. Die Messwerte für die Photoströme und die damit korrespondierenden Auslenkungen der Photozelle konnten sehr genau vorgenommen werden und decken sich ungefähr mit den Erwartungen.\\
Allerdings konnte der Laser nicht genau auf eine nicht vorhandene Auslenkung der Zelle zentriert werden, weswegen das globale Maximum der Intensitätsverteilung um $\approx \SI{1}{\milli\metre}$ verschoben ist.\\
Dies stellt aber kein Problem dar, da dies durch eine Verschiebung der Ausgleichsfunktionen behoben werden konnte.\\\\
Die Ausgleichsfunktion für den Einzelspalt liefert für die Breite des Spaltes den Wert $ b_1=\SI{0.1734(00004)}{\milli\metre}$.
Mit dem Theoriewert $ b_{theo}=\SI{0.15}{\milli\metre}$ ergibt sich die relative Abweichung von diesem zu 
\begin{equation*}
    \frac{b_{theo}-b_1}{\b_theo}\approx \SI{-15.645}{\percent} \; \;.
\end{equation*}
Dieses Ergebnis ist gut. Allerdings ist in Abbildung \ref{fig:plot1} zu erkennen, dass die Ausgleichsfunktion sich nicht perfekt mit denn Messwerten deckt. 
Dies liegt vermutlich daran, dass die Anpassung auf die Maxima erster Ordunung die Werte der Funktion so ändert das sie höher wird. Dies könnte dann auch zu der Abweichung führen.\\\\
Für den Doppelspalt wurde $s=\SI{0.9954(00171)}{\milli\metre}$ und $ b_2=\SI{0.1740(00251)}{\milli\metre}$ über die Ausgleichsrechnung bestimmt.\\
Mit den Theoriewert $s_{theo}=\SI{1}{\milli\metre}$ ergibt sich eine relative Abweichung von $\approx\SI{0.459}{\percent}$. 
Für $b_{theo}=\SI{0.15}{\milli\metre}$ ergibt sich $\approx\SI{-15.975}{\percent}$.\\
Das Ergebnis für den Abstand der Spalten ist also sehr gut. Für die Breite von ihnen ist der Wert und die Abweichung in der sleben Größenordnung wie beim Einzelspalt.\\
Dies könnte die selben Ursachen haben wie beim ihm haben, könnte aber auch auf eventuelle Probleme bei der Durchführung hindeuten.\\ 
Zum Beispiel könnte das Hauptmaxima der Funktion nicht fein genug aufgespalten worden sein, so dass die gemessenen Ströme dort kleiner ausfallen. \\
Für die gemsessenen Ströme des Doppelspaltes ist es allerdings sonderbar, dass die Messwerte, wie in Abbildung \ref{fig:plot2} zu sehen, um $0$ herum nur auf der Einhüllenden liegen.\\
Eigentlich wäre es zu erwarten das die Werte auch wesentlich kleinere Werte annehmen.\\
Da die Ausgleichsrechnung auf diese Werte aber sehr gute Ergebnisse liefert, scheint dies nicht weiter besorgniserregend zu sein.\\\\
Die in Abbildung \ref{fig:plot3} zusammen dargestellten normierten Funktionen des Einzelspaltes und des Doppelspaltes decken sich sehr gut.\\\\
Alles in allem sind die Ergebnisse also gut.