\section{Zielsetzung}

    \noindent In diesem Versuch wird wird ein Zusammenhang zwischen den Beugungsmuster, dem beugenden Objekt un der Aperturfunktion hergestellt.
    Dazu wird wird das Licht als Welle betrachtet nach dem es auf einen Spalt mit einem relativ zur Wellenlänge des Lichts kleine Abmessungen hat.

\section{Einleitung}

    \noindent Licht kann nicht mehr mittels geometrischen Optik betrachtet werden mehr dem es durch eine Öffnung hindurchtritt welche kleine 
    Abmessungen im Vergleich zur Wellenlänge des Lichts aufweist. Es entsteht Beugung, dieser Effekt lässt sich quantenmechanisch beschreiben, 
    es ist jedoch bei einer großen Anzahl an Lichtquanten, das Licht als Welle zu Nähern. 

\section{Theoretische Grundlagen}

    

