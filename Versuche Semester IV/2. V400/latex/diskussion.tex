\newpage
\section{Diskussion}

Alles in allem lief die Versuchsdurchführung sehr gut.
Die Winkel konnten zwar nur bis auf einen halben Grad genau gemessen werden und Winkel mit nicht fixierten Unterlagen zu messen war nicht optimal, die Ergebnisse zeigen aber meistens relativ wenig Abweichung von der Theorie.\\

\noindent Aus den Messwerten des Reflexionsgesetzes wurde durch einen Fit ein Faktor für die Abhängigkeit des Ausfallswinkels vom Einfallswinkel bestimmt.
Dieser beträgt $\SI{1.0094(80)}{}$  und hat damit von der Theorie eine relative Abweichung von $\SI{-0.9(8)}{\percent}$.\\
Dies ist ein sehr gutes Ergebnis. Allerdings lässt sich hier schon erkennen, was bei allen weiteren Ergebnissen auch zu sehen ist, nämlich dass fast alle bestimmten Werte größer sind als die Theoriewerte.\\
Dies lässt sich vermutlich darauf zurückführen, dass die Messungen, wie zuvor gesagt, nicht sehr genau waren und dabei vermutlich öfter überschätzt als unterschätzt wurde.
Außerdem waren die Messreihen alle recht klein, was für die Größe der Fehler und für die allgemeine Ungenauigkeit bei der Auswertung ins Gewicht schlägt.\\\\
Bei der Untersuchung der Brechung von Licht in Plexiglas wurde für den Brechungsindex der Wert $n= \SI{1.4578(0105)}{}$ und für die \\Lichtgeschwindigkeit $\SI{205641264.8144 (14778084936)}{\metre\per\second}$ bestimmt.\\
Die relative Abweichung von der Theorie beträgt dabei $\SI{2.2(7)}{\percent}$ und $\SI{-2.2(7)}{\percent}$. Dies sind wieder sehr gute Werte.\\\\
Der Strahlversatz mit den zuvor gemessenen Winkeln berechnete sich zu \\$s_1=\SI{0.0018(5)}{\metre}$. 
Die Rechnung, in der der Austrittswinkel mit dem zuvor bestimmten Brechungsindex bestimmt wurde, ergab $s_2=\SI{0.0035(6)}{\metre}$.\\
Diese Werte bewegen sich beide in der zu erwartenden Größenordnung. Der zweite zeigt eine Relative Abweichung vom ersten von fast $\SI{100}{\percent}$. \\
Dies lässt sich vermutlich darauf zurückführen, dass dieses Ergebnis über einen eigens berechneten Wert bestimmt wurde. 
Der berechnete Brechungsindex hatte zwar, wie zuvor erwähnt, nur eine geringe relative Abweichung, sein Fehler pflanzt sich aber trotzdem weiter fort.\\
Des Weiteren ist der erste Wert auch kein perfekter Wert, da er nur von Messwerten abhängt und sich zum Beispiel auch durch Verunreinigungen auf dem Plexiglasquader sehr stark beeinflussen lassen würde.\\
Welcher der beiden Werte der Bessere ist ist also schwer zu beurteilen. Allerdings bewegen sich beide in der richtigen Größenordnung.\\\\
Der Versuch, in welchem die optischen Eigenschaften eines Prismas untersucht wurden, hat Ergebnisse geliefert, die sich in der Tabelle \ref{tab:prism2} wiederfinden.\\
Auch diese bewegen sich in der erwarteten Größenordnung.\\\\
Für die Untersuchung der Beugungsmaxima sind dies die Ergebnisse und relativen Abweichungen davon:
\begin{align*}
    \intertext{Für $d=\SI{1/600}{\milli\metre}$:}\\
    \lambda_1&=\SI{664.5818}{\pico\metre}\\
    \symup{Rel_{Abw}}&=\SI{-4.6585}{\percent}\\\\
    \intertext{Für $d=\SI{1/300}{\milli\metre}$:}\\
    \lambda_2&=\SI{885.0468(51437)}{\pico\metre}\\
    \symup{Rel_{Abw}}&=\SI{39.3774(810030)}{\percent}\\\\
    \intertext{Für $d=\SI{1/100}{\milli\metre}$:}\\
    \lambda_3&=\SI{109.2370(6015887)}{\pico\metre}\\
    \symup{Rel_{Abw}}&=\SI{-72.0268(947384)}{\percent}
\end{align*}
\noindent Für die Gitterkonstante $d=\SI{1/600}{\milli\metre}$ wurden dabei die besten Ergebnisse erhalten mit einer realtiven Abweichung von nur $\SI{-4.6585}{\percent}$.
Die anderen haben zwar wesentlich größere Abweichungen sind aber trotzdem noch gute Ergebnisse.\\
Das bei den anderen die Werte weiter von der Theorie abweichen kann sich unter anderem davon ableiten, dass dort Maxima mit geringeren Gitterdichten genutzt wurden.
Dies führt dazu, dass die Maxima dort weniger scharf und weiter gestreut sind, wobei sich Abweichungen vom mathematischen Ort des Maximas einschleichen.\\\\
Alles in allem lief der Versuch gut und alle Ergebnisse bewegen sich in der zu erwartenden Größenordnung. 

