\newpage
\section{Diskussion}

Alles in allem lief die Versuchsdurchführung sehr gut.
Die Winkel konnten zwar nur bis auf einen halben Grad genau gemessen werden und Winkel mit nicht fixierten Unterlagen zu messen war nicht optimal, die Ergebnisse zeige aber meistens relativ wenig Abweichung von der Theorie.\\

Aus den Messwerten des Reflexionsgesetzes wurde durch einen Fit ein Faktor für die Abhängigkeit des Ausfallswinkels vom Einfallswinkel bestimmt.
Dieser beträgt $\SI{1.0094(80)}{}$  und hat damit von der Theorie eine relative Abweichung von $\SI{-0.9(8)}{\percent}$.\\
Dies ist ein sehr gutes Ergebnis. Allerdings lässt sich hier schon erkennen, was bei allen weiteren Ergebnissen auch zu sehen ist, nämlich dass alle bestimmten Werte größer sind als die Theoriewerte.\\
Dies lässt sich vermutlich darauf zurückführen, dass die Messungen, wie zuvor gesagt, nicht sehr genau waren und dabei vermutlich öfter überschätzt als unterschätzt wurde.
Außerdem waren die Messreihen alle recht klein, was für die Größe der Fehler und für die allgemeine Ungenauigkeit bei der Auswertung ins Gewicht schlägt.\\\\
Bei der Untersuchung der Brechung von Licht in Plexiglas wurde für den Brechungsindex der Wert $n= \SI{1.4578(0105)}{}$
