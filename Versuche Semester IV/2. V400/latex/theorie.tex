\section{Zielsetzung}

    \noindent In diesem Versuch sollen glundlegende Gesetzmäßigkeiten der Strahlenoptik untersucht werden.
    Es wird genauer die Reflixion, Brechung und Beugung angeschaut.

\sextion{Theoretische Grundlagen}

    Das für das menschliche Auge wahrnehmbare optische Licht ist ein teil des elektromagnetischen Spektrums, nur der Bereich von 380 nm bis 780 
    nm ist vom Auge zu erkennen. Generlell lässt sich die Ausbreitung von elektromagnischen Wellen von den Maxwellschen Gleichungen beschreiben, 
    für die in diesem Versuch untersuchten Effekte reichen aber die Regeln der $Strahlenoptik$. Wellen und ihre Ausbreitung werden in der 
    $Strahlenoptik$ durch die Normalen die Senkrecht auf den Wellen stehen beschreiben, sie werden als Lichtstrahl bezeichnet. Da die 
    Ausbreitungsgescwindigkeit und somit auch die lokale Lichtgeschwindigkeit in unterschiedlichen Materiealien unterschiedlich ist, entsteht beim 
    Übergang von einem Material in ein anderes der Effekt der Brechung. Der Lichtstrahl wird gebrochen und verändert seine Richtung.