\newpage
\section{Auswertung}

    \subsection{Aufnahme eines Emissionsspektrum einer Kupfer-Röntgenröhre}

        \noindent Werden die Daten \ref{tab:Cu} der Impulse pro Sekunde und der Winkel wie in Abbildung \ref{img:Spekt} geplottet, lassen sich die zwei 
        charakteristischen Winkel ablesen und nach der Beziehung \ref{eqn:bragg} auch die Wellenlänge bzw. Energien berechnen.

            \begin{figure}
                \centering
                \includegraphics[width=0.65\textwidth]{build/plots/Emissionsspektrum.pdf}
                \caption{Ein Plot in dem die Impulsrate gegen den Winkel aufgetragen ist.}
                \label{img:Spekt}
            \end{figure}

        \noindent Aus dem Plot lässt sich ein Winkel von $\SI{22.7 \pm 0.2}{\degree}$ für die $\alpha$-Linie und ein Winkel von $\SI{20.4 \pm 0.2}{\degree}$
        ablesen. Daraus berechnet sich nach Formel \ref{eqn:bragg} eine Energie von $\SI{7979 \pm 66}{\electronvolt}$ für die $\text{K}_{\alpha}$ und
        $\SI{8830 \pm 82}{\electronvolt}$ für die $\text{K}_{\beta}$ Linie. 

    \subsection{Bestimmung der Transmisson als Funktion der Wellenlänge}

        \noindent Die Daten der Impulsrate \ref{tab:al}, \ref{tab:ohne} als Funktion des Winkel werden zunächst mittels $\text{N}_{\text{err}}=\sqrt{N}$ mit einem Fehler versehen.
        Zusätzlich wird die Impulsrate über die Totzeit des Geiger-Müller-Zählrohrs mittels der Formel 
        
            \begin{equation}
                I = \frac{N}{1 - \tau \cdot N}
            \end{equation}
        
        \noindent angepasst. Hier ist $\tau$ = $\SI{90}{\micro\second}$ die Totzeit des Zählrohrs. Werden diese Daten geplottet und durch eine 
        Ausgleichsgerade approximiert ergibt sich Abbildung \ref{img:Trans}.\\ Die Steigung der Geraden berechnet sich zu 
        a = $\SI{-15194  (239)e6}{\per\meter}$ und der y-Achsenabschnitt zu b = $\num{1.225 \pm 0.014}$.\\
        Grafisch dargestellt findet man diesen Zusammenhang in Abbildung \ref{img:Trans}.

            \begin{figure}[h]
                \centering
                \includegraphics[width=0.8\textwidth]{build/plots/transmission.pdf}
                \caption{Ein Plot des Transmissionsverhältnises in Abhängigkeit der Wellenlänge.}
                \label{img:Trans}
            \end{figure}

    \subsection{Bestimmung der Compton-Wellenlänge \texorpdfstring{$\lambda_{\text{c}}$}{TEXT}}

        \noindent Über einen Zeitraum von t = 300s wurden die Impuszahlen $I_0$ = 2731, $I_1$ = 1180 und $I_2$ = 1024 gemessen.\\
        $I_0$ entstand bei der Messung ohne Absorber, für die Messung von $I_1$ wurde ein Al-Absorber zwischen Röntgenröhre und 
        Streuer platziert. Dieser wurde dann für die Messung von $I_2$ zwischen den Streuer und das Geiger-Müller-Zählrohr verschoben.
        Die Transmissionen lassen aus diesen Werten wie folgend berechnen:

            \begin{align}
                T_1 = \frac{I_1}{I_0} = 0.432   \nonumber\\
                T_2 = \frac{I_2}{I_0} = 0.375   \nonumber
            \end{align}

        \noindent Mittels der Ausgleichsgerade aus Abbildung \ref{img:Trans} lässt sich aus der Tansmissionszahl nun über 
            
            \begin{equation*}
                \lambda = \frac{T - b}{a}
            \end{equation*}

        \noindent die Wellenlänge berechnen.\\
        Aus $T_1$ ergibt sich $\lambda_1 = \SI{522 (12)e-13}{\meter}$ und $\lambda_2 = 
        \SI{559  (13)e-13}{\meter}$ für $T_2$. Die Compton-Wellenlänge wird nun über $\lambda_{\text{c}} = \lambda_2 - \lambda_1$ zu 
        $\lambda_{\text{c}} = \SI{376 (6) e-14}{\meter}$ berechnet.

            

    


    