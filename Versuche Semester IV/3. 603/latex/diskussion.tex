\section{Diskussion}
    \begin{enumerate}
        \item Emissionsspektrum Kupfer, theorie werte,      k alpha : 8048.11,      k beta : 8906.9\\
                                        berechnete werte:   k alpha : 7976.2 +- 66, k beta : 8830.4 +- 82\\
                                        Abweichung          k alpha : 0.9 +- 0.8 \% k beta : 0.9 +- 0.9 \%
        \item Totzeitkorrektur: abweichung der werte durch Korrektur von 2-4\% 

        \item Totzeitkorrektur: abweichung der Werte aufgrund er geringen Impulszahl über langen Zeitraum nur gering, kleiner einem Prozent


    \end{enumerate}

Der Versuch zur Messung der Compton-Wellenlänge lief im Allgemeinen sehr gut. Die Auswertung basiert dabei zwar nur auf zugeteilten Daten, diese liefern aber gute Ergebnisse.\\
Da der Versuch nicht sleber ausgeführt werden ist es nicht möglich Missstände bei der Durchführung zu diskutieren.\\\\
\noindent
Bei der Bestimmung
