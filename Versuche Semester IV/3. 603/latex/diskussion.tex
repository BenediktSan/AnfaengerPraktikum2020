\section{Diskussion}
    \begin{enumerate}
        \item Emissionsspektrum Kupfer, theorie werte,      k alpha : 8048.11,      k beta : 8906.9\\
                                        berechnete werte:   k alpha : 7976.2 +- 66, k beta : 8830.4 +- 82\\
                                        Abweichung          k alpha : 0.9 +- 0.8 \% k beta : 0.9 +- 0.9 \%

        Nur sehr geringe Abwewichung, entsteht wahrscheinlich durch einen Ablesefehler da Wertezahl der Peaks auch nur sehr gering.

        \item Totzeitkorrektur: abweichung der werte durch Korrektur von 2-4\% 
                                Hohe Impulsrate die in der Größenordnung der Totzeit des Zählrohrs liegt.

        \item Totzeitkorrektur: abweichung der Werte aufgrund der geringen Impulszahl über langen Zeitraum nur gering, kleiner ein Prozent
                Comptonwellenlänge theorie Wert : $\SI{2.425e-12}{\meter}$ berechnet : $\SI{375.93(5.9)e-14}{\meter}$

                                 Pronzentuale Abweichung: -$-0.549 \pm 0.024 \%$
                                 Da lambda c sich aus der Differenz von lambda 1 und lambda 2 berechnet die selber in der gleichen Größenordnung 
                                 sind,  
        
    \end{enumerate}