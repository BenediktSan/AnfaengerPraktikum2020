\section{Diskussion}
    \begin{enumerate}
        \item Emissionsspektrum Kupfer, theorie werte,      k alpha : 8048.11,      k beta : 8906.9\\
                                        berechnete werte:   k alpha : 7976.2 +- 66, k beta : 8830.4 +- 82\\
                                        Abweichung          k alpha : 0.9 +- 0.8 \% k beta : 0.9 +- 0.9 \%
        \item Totzeitkorrektur: abweichung der werte durch Korrektur von 2-4\% 

        \item Totzeitkorrektur: abweichung der Werte aufgrund er geringen Impulszahl über langen Zeitraum nur gering, kleiner einem Prozent
                Comptonwellenlänge theorie Wert : $\SI{2.425e-12}{\meter}$ berechnet : $\SI{375.93(5.9)e-14}{\meter}$

                                 Pronzentuale Abweichung: -$-0.549 \pm 0.024 \%$
                                 Da lambda c sich aus der Differenz von lambda 1 und lambda 2 berechnet die selber in der gleichen Größenordnung 
                                 sind, führen bereits kleinere statistische oder methodische Abweichungen zu großen abweichungen in lamda c
        

    \end{enumerate}

Der Versuch zur Messung der Compton-Wellenlänge lief im Allgemeinen sehr gut. Die Auswertung basiert dabei zwar nur auf zugeteilten Daten, diese liefern aber gute Ergebnisse.\\
Da der Versuch nicht selber ausgeführt werden ist es nicht möglich Missstände bei der Durchführung zu diskutieren.\\\\
\noindent
Bei der Bestimmung
