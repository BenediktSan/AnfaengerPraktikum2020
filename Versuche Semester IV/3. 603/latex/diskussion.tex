\newpage
\section{Diskussion}
    \begin{enumerate}
        \item Emissionsspektrum Kupfer, theorie werte,      k alpha : 8048.11,      k beta : 8906.9\\
                                        berechnete werte:   k alpha : 7976.2 +- 66, k beta : 8830.4 +- 82\\
                                        Abweichung          k alpha : 0.9 +- 0.8 \% k beta : 0.9 +- 0.9 \%
        \item Totzeitkorrektur: abweichung der werte durch Korrektur von 2-4\% 

        \item Totzeitkorrektur: abweichung der Werte aufgrund er geringen Impulszahl über langen Zeitraum nur gering, kleiner einem Prozent
                Comptonwellenlänge theorie Wert : $\SI{2.425e-12}{\meter}$ berechnet : $\SI{375.93(590)e-14}{\meter}$

                                % Pronzentuale Abweichung: -$-0.549 \pm 0.024 \%$
                                % Da lambda c sich aus der Differenz von lambda 1 und lambda 2 berechnet die selber in der gleichen Größenordnung 
                                 %sind, führen bereits kleinere statistische oder methodische Abweichungen zu großen abweichungen in lamda c
        

    \end{enumerate}


\noindent
Der Versuch zur Messung der Compton-Wellenlänge lief im Allgemeinen sehr gut. Die Auswertung basiert dabei zwar nur auf zugeteilten Daten, diese liefern aber gute Ergebnisse.\\
Da der Versuch nicht selber ausgeführt werden ist es nicht möglich Missstände bei der Durchführung zu diskutieren.\\\\
\noindent
EINHEITEN EINFÜGEN PLS!!!\\\\
Bei der Untersuchung des Emissionsspektrum lieferte die Auswertung für die $k_{alpha}$-Linie einen Wert von $\SI{7972.2(660)}{}$, wobei der Theoriewert\cite{theo} bei bei $\SI{8048.9}{}$ liegt.\\
Daraus lässt sich eine relative Abweichung vom Theoriewert von $\SI{0.9(08)}{\percent}$ bestimmen.\\
Für die $k_{beta}$-Linie gilt dasselbe, nur mit einem errechneten Wert von $\SI{8830.4(820)}{}$, dem theoretischen Wert $\SI{8906.9}{}$\cite{theo} und damit auch mit der relativen Abweichung $\SI{0.9(09)}{\percent}$.\\
Dabei sollte für diese Messreihe noch eine zusätzliche Abweichung durch die Korrektur der Totzeit des Geiger-Müller-Zählrohrs von $2$-$4\si{\percent}$ beachtet werden.\\
Trotzdem zeigen diese Messwerte nur eine sehr kleine Abweichung von den Theoriewerten. Diese finden sich auch in den Fehlerintervallen der Werte wieder.\\\\

\noindent Bei der Messung der Impulszahlen hat die Totzeitkorrektur nur einen Einfluss von weniger als einem Prozent auf die gemessenen Impulszahlen. \\
Dies liegt daran, dass über einen sehr langen Zeitraum gemessen wurde und nur wenige Impulse stattfanden. 
Allerdings führen die wenigen Impulse auch zu einer größeren Abweichung, da so statistische Zerfälle in der Umgebung einen größeren Einfluss auf die Messdaten haben.\\
Aus diesen Messdaten lies sich dann die Compton-Wellenlänge bestimmen. Dies ergab $\SI{3.7593(0590)e-12}{\meter}$.
Im Vergleich mit der theoretischen Wellenlänge von $\SI{2.425e-12}{\meter}$ ergibt sich die relative Abweichung zu $\SI{0.549(0024)}{\percent}$. \\
SOMETHING WRONG WITH THE RELATIVE DEVIATION!!!\\
Dies sind wieder sehr gute Messwerte. Der Theoriewert liegt hier nur leider nicht im Fehlerintervall.\\
Da die Compton-Wellenlänge als Differenz von $\lambda_1$ und $\lambda_2$ bestimmt wird und diese sich in der selben Größenordnung befinden, führt schon eine kleine statistische oder methodische Abweichung zu großen Abweichungen.\\\\
\noindent
Alles in allem hat die Auswertung aber sehr gute Ergebnisse geliefert und Messwerte scheinen damit auch wenige methodische und statistische Fehler zu beeinhalten.