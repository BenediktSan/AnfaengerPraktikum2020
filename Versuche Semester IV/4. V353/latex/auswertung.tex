\newpage
\subsection{Auswertung}
Für den ersten Teil der Auswertung wurde Entladevorgang eines Kondensators untersucht.\\
Dabei wurde die Kondensatorspannung und die Zeit seit Beginn des Entladevorgangs gemessen.
Diese Messwerte finden sich in der Tabelle \ref{tab:1} und grafisch dargestellt in Abbildung \ref{img:1}.\\
In der Abbildung wurde der Logarithmus der Kondensatorspannung geteilt durch die angelegte Spannung gegen die Zeit aufgetragen.\\
Zusätlich ist dort noch ein linearer Fit auf die halblogarithmischen Messwerten abgebildet. Dieser hat die Form:
\begin{equation*}
    \log{\frac{U_C}{U_0}= -\frac{1}{RC}\cdot t +n
\end{equation*}
Wobei $RC= \SI{0.0082(4)}{\per\second}$ und $\SI{-016694(14582)}{}$ gilt.

\begin{figure}[h]
    \centering
    \includegraphics[width=0.6\textwidth]{build/plots/plot0.pdf}
    \caption{Ein linearer Fit auf die halblogarithmischen dargestellten Messwerte von Spannung und Zeit.}
    \label{img:1}
\end{figure}
\begin{table}[H]
    \centering
    \begin{tabular}{S [table-format=2.3] S [table-format=2.3]}
        \toprule
        {$\text{Einfallswinkel in Grad}$} & {$\text{Ausfallswinkel in Grad} $}\\
        \midrule
        0.620 & -0.478 & 0.000\\
        0.520 & -0.654 & 0.002\\
        0.420 & -0.868 & 0.004\\
        0.340 & -1.079 & 0.006\\
        0.290 & -1.238 & 0.008\\
        0.240 & -1.427 & 0.010\\
        0.200 & -1.609 & 0.012\\
        0.170 & -1.772 & 0.014\\
        0.125 & -2.079 & 0.016\\
        0.100 & -2.303 & 0.018\\
        0.085 & -2.465 & 0.020\\
        0.070 & -2.659 & 0.022\\
        0.060 & -2.813 & 0.024\\
        0.050 & -2.996 & 0.026\\
        0.040 & -3.219 & 0.028\\
        0.030 & -3.507 & 0.030\\
        0.020 & -3.912 & 0.032\\
        0.010 & -4.605 & 0.036\\
        0.001 & -6.908 & 0.046\\
        \bottomrule
    \end{tabular}
\caption{Die Messwerte der abfallenden Kondensatorspannung in Abhängigkeit von der Zeit. Zusätzlich noch die für das plotten verwendeten logarithmierten Spannungswerte.}
\label{tab:einf}
\end{table}

