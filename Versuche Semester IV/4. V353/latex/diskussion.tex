\newpage
\section{Diskussion}

Die verschiedenen Methoden zur Bestimmung der Zeitkonstante haben zu folgenden Ergebnissen geführt:
\begin{align*}
    RC_1&= \SI{0.0082(4)}{\second} \\
    RC_2&=\SI{0.0067(2)}{\second}\\
    RC_3&=\SI{0.00704(61)}{\second}
\end{align*}
Dabei ist zu erkennen, dass sich alle drei Werte in der selben Größenordnung bewegen.
Die geringste Abweichung voneinander ist dabei zwischen den Werte zu erkennen, welche über die Frequenzabhängigkeit der Kondensatorspannung und über die Phasenverschiebung bestimmt wurden.\\
Ein Grund warum der Wert, welcher über die Entladungskurve des Kondensators bestimmt wurde, etwas stärker davon abweicht, ist, dass es hier nicht möglich war sehr genaue Messwerte vom Oszilloskop abzulesen.\\
Das Oszilloskop ermöglichte nämlich keine Feinjustierungen die ein genaueres ablesen ermöglicht hätten. 
Die Entladekurve konnte nämlich bei feineren Einstellung der Zeit und Spannug nicht komplett abgebildet werden, wordurch gröbere Einstellungen gewählt werden mussten.\\
Abgesehen davon verliefen die Messreihen aber ohne Probleme. Das ablesen der Werte vom Oszilloskop führt natürlich immer zu gewissen Ungenauigkeiten, diese konnten aber durch Feinjustierungen der Einstellungen minimiert werden.\\
Die Untersuchung des Relaxationsverhalten hat zu Werten für die Zeitkonstante $RC$, welche alle nur maximal eine relative Abweichung von maximal $\approx \SI{19}{\percent}$ voneinander haben, geführt.\\
Alles in allem ist dies ein zufriedenstellendes Ergebnis.