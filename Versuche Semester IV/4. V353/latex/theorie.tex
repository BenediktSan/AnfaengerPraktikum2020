\section{Theoretische Grundlagen}

    \subsection{Zielsetzung}
        In diesem Versuch 

    \begin{equation}
        \frac{\symup{d}A}{\symup{d}t}=c[A(t)-A(\infty)]
    \end{equation}

    \begin{equation}
        \int_{A(0)}^{A(t)} \frac{\symup{d}A'}{A' - A(\infty)} = \int_0^t \symup{d}t'
    \end{equation}

    \begin{equation}
        \text{ln} \frac{A(t) - A(\infty)}{A(0) - A(\infty)} = ct
    \end{equation}

    \begin{equation}
        A(t) = A(\infty) + [A(0) - A(\infty)] \text{e}^{ct}
    \end{equation}

    \begin{equation}
        U_{\text{C}} = \frac{Q}{C}
    \end{equation}

    \begin{equation}
        I = \frac{U_{\text{C}}}{R}
    \end{equation}

    \begin{equation}
        \text{d}Q = -I \text{d}t
    \end{equation}

    \begin{equation}
        \frac{\text{d}Q}{\text{d}t} = - \frac{1}{RC} Q(t)
    \end{equation}

    \begin{equation}
        Q(\infty) = 0
    \end{equation}

    \begin{equation}
        Q(t) = Q(0) \text{exp}(-t/RC)
    \end{equation}

    \begin{equation}
        Q(0) = 0 \quad \text{und} \quad Q(\infty) = CU_0 
    \end{equation}

    \begin{equation}
        Q(t) = CU_0(1- \text{exp}(-t/RC))
    \end{equation}

    \begin{equation}
        \frac{Q(t = RC)}{Q(0)} = \frac{1}{e} \approx 0,368
    \end{equation}

    \begin{equation}
        U(t) = U_0 \text{cos}( \omega t)
    \end{equation}

    \begin{equation}
        U_{\text{C}}(t) = A(\omega) \text{cos}(\omega t + \varphi \{ \omega \} )
    \end{equation}

    \begin{equation}
        U(t) = U_R(t) + U_C(t)
    \end{equation}

    \begin{equation}
        U_0 \text{cos}(\omega t) = I(t)R + A(\omega) \text{cos}(\omega t + \varphi)
    \end{equation}

    \begin{equation}
        I(t) = \frac{\text{d}Q}{\text{d}t} = C \frac{\text{d}U_\text{C}}{\text{d}t}
    \end{equation}

    \begin{equation}
        U_0 \text{cos}(\omega t) = -A\omega R C \text{sin}(\omega t + \varphi) + A(\omega) \text{cos}(\omega t + \varphi)
    \end{equation}

    \begin{equation}
        0 = -\omega R C \text{sin} \left( \frac{\pi}{2} + \varphi \right) + \text{cos} \left( \frac{\pi}{2} + \varphi \right)
    \end{equation}

    \begin{equation}
        \frac{\text{sin} \varphi}{\text{cos} \varphi} = \text{tan} \varphi (\omega) = -\omega RC 
    \end{equation}

    \begin{equation}
        \varphi (\omega) = \text{arctan} ( - \omega R C)
    \end{equation}

    \begin{equation}
        \omega t + \varphi = \frac{\pi}{2}
    \end{equation}

    \begin{equation}
        U_0 \text{cos}(\frac{\pi}{2} - \varphi) = -A \omega R C 
    \end{equation}

    \begin{equation}
        A(\omega) = - \frac{\text{sin} \varphi}{\omega R C} U_0
    \end{equation}

    \begin{equation}
        \text{sin}^2 \varphi + \text{cos}^2 \varphi = 1
    \end{equation}

    \begin{equation}
        \text{sin} \varphi = \frac{\omega R C}{ \sqrt{1 + \omega^2 R^2 C^2} }
    \end{equation}

    \begin{equation}
        A(\omega) = \frac{U_0}{\sqrt{1 + \omega^2 R^2 C^2}}
    \end{equation}

    \begin{equation}
        U(t) = U_{\text{R}}(t) + U_{\text{C}}(t) = I(t) \cdot R + U_{\text{C}}(t)
    \end{equation}

    \begin{equation}
        U(t) = RC \frac{\text{d}U_{\text{C}}}{\text{d}t} + U_{\text{C}}(t)
    \end{equation}

    \begin{equation}
        U(t) = RC \frac{\text{d}U_{\text{C}}}{\text{d}t}
    \end{equation}

    \begin{equation}
        U_{\text{C}}(t) = \frac{1}{RC} \int_0^t U(t')\text{d}t'
    \end{equation}