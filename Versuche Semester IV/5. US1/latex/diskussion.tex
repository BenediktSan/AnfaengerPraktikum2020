\section{Diskussion}

    Die Bestimmung der Schallgeschwindigkeit $c_{\text{Schall}} = \SI{2741(7)}{\metre\per\second}$ hat sehr gut funktioniert, die Abweichung zum 
    Theoriewert von $c_{\text{Theo}} = \SI{2730}{\metre\per\second}$ \cite{cA} berechnet sich zu jediglich 0,4\%, jedoch ist davon auszugehen, 
    dass sich ein paar Systematische Fehler durch alte Geräte oder ungenaues Ablesen eingeschlichen haben.\\
    Zu dem Absorptionskoeffizient wurden keine Theoriewerte gefunden, es lässt sich jedoch sagen, dass auch trotz einer relativ großen Streuung 
    der Messwerte ein sinnvoller exponential Fit berechnet werden konnte.\\
    Das Abmessen des Augenmodells war eher schwierig, da bereits sehr kleine Änderungen der Sonde zu großen Änderungen der Ergebnisse geführt 
    hatte. Es ist also nicht möglich zu sagen ob die richtigen Messdaten getroffen wurden. Außerdem wurden die genauen Werte nicht mit dem 
    Cursor-Tool abgelesen, sondern mussten jetzt von dem Photo geschätzt werden, dies führt zu relativ großen Ungenauigkeiten.
    Des weiteren ist es relativ wahrscheinlich, dass sich aufgrund des schlechten Zustands des Modells eine Luftschicht zwischen Sonde und Modell 
    gebildet hat. Dies würde alle Impulszeiten additiv verlängern und erklären weshalb die Iris und Linse schinbar in der Mitte des Modells sind.
    
