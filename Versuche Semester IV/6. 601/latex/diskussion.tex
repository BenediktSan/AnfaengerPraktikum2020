\section{Diskussion}

Die Bestimmung der mittleren freien Weglänge hat genau die Ergebnisse geliefert, die zu ertwarten waren. 
Für kleine Temperaturen ist die freie Wegglänge im Vergleich mit dem Abstand der Anode und der Kathode fast identisch, so das kaum Stoßprozesse stattfinden.\\
Bei hohen Temperaturen ist die Weglänge aber wesentlich kleiner, weswegen der Franck-Hertz-Versuch dort durchgeführt werden kann.\\\\

\noindent
Die Messreihen, welche die differentielle Energieverteilung untersuchten lieferten auch gute Ergebnisse. Die Zeichnungen hatten die Form die zu erwarten waren.\\
Der Graph, der bei einer geringeren Temperatur aufgenommen wurde, fällt langsam und immer stärker werdend ab. 
Wenn dies ausgewertet wird, wie in Abbildung \ref{img:1} lässt sich erkennen, dass fast alle gemessenen Elektronen eine Energie von $\approx \SI{6.6}{\eV}$ besitzen. 
Dies deckt sich mit den Erwartungen, dass die Elektronen eine Energie besitzen, welche der Beschleunigungsspannung abzüglich des Kontaktpotentials entspricht.\\
Aus diesem Zusammenhang wurde dieses Potential zu $k=\SI{5.4}{\volt}$ bestimmt. Dieser Wert liegt in der zu erwartenden Größenordnung.\\
Für den Graphen bei höheren Energien ist ein direktes Abfallen zu erwarten, was auch hier qualitativ zu sehen ist. \\
Die einzelnen Steigungen ausgewertet, wie in Abbildung \ref{img:2}, führen zu dem Ergebnis, dass die Anregungsenergie bei $\SI{-0.5}{\eV}$ vor dem Beginn der Zeichnungen erreicht wurde.\\
An sich ist dies erstmal nicht so schlimm, allerdings wurde dieser Wert über das Kontaktpotential berechnet, welches durch ablesen und abschätzen bestimmt wurde.
Damit hat dieser Wert einen sehr großen Fehler, was zu einem Abfall der Kurve bei $\approx 0$ führen könnte, was sich mit der bei der Auswertung gewählten Einstellung des Schreibers decken würde.\\\\

\noindent
Die Zeichung der Franck-Hertz-Kurve liefert die zu erwartende Form und besitzt die ersten sieben Maxima.\\
Die aus dem gemittelten Abstand der Maxima berechneten Werte finden sich mit den Theoriewerten und der realtiven Abweichung zu ihnen, in der folgenden Tabelle \ref{tab:4}.

\begin{table}[h]
    \centering
    \small
    \begin{tabular}{S [table-format=5.5] S [table-format=5.5] S [table-format=5.5] S [table-format=5.5]}
        \toprule
        {Wert} & {$\text{Messwert}  $} & {$\text{Theoriewert }$}& {$\text{relative Abweichung }$}\\
        \midrule
        \text{$\increment E$} & \text{$\SI{5.45(40)}{\eV}$}& \text{$\SI{4.9}{\eV}$} & \text{$\SI{-11(8)}{\percent}$} \\
        \text{$\lambda$} & \text{$\SI{227.385(17000)}{\nano\metre}$} & \text{$\SI{253.029}{\nano\metre}$} &\text{$\SI{10(7)}{\percent}$} \\
        \bottomrule
    \end{tabular}
\caption{Die Ergebnisse der Auswertung der Franck-Hertz-Kurve und ihre Abweichung von der Theorie dargestellt.  }
\label{tab:4}
\end{table}

\noindent
Eine relative Abweichung von $\approx \SI{10}{\percent}$ von den Theoriewerten ist ein sehr gutes Ergebnis, vor allem da analog gemessen wurde und keine digitalen Werte aufgenommen wurden, was die Präzision erhöht hätte.\\\\

\noindent Bei allen Messreihen ist aber zu beachten, dass der Versuch sehr leicht zu beeinflussen ist, da er sehr stark von sehr vielen Faktoren abhängt.\\
Minimale Temperaturänderungen können zum Beispiel schon zu starken Schwankungen im Ergebniss führen. Selbes gilt für die angelegten Brems- oder BEschleunigungsspannungen.\\
Nichtsdestotrotz liefert die Auswertung, wie zum Beispiel beim Franck-Hertz-Versuch, sehr gute Ergebnisse. 
Das größte Problem stellt hier das überschreiten der Anregungsenergie bei der zweiten Zeichnung, vor dem Beginn der Aufzeichnungen dar. Dies lässt sich aber bis zu einem gewissen Punkt wegargumentieren.\\
Die Ergebnisse sind also alles in allem recht gut.