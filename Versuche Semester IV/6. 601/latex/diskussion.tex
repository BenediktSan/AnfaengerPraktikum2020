\section{Diskussion}

Die Bestimmung der mittleren freien Weglänge hat genau die Ergebnisse geliefert, die zu ertwarten waren. 
Für kleine Temperaturen ist die freie Wegglänge im Vergleich mit dem Abstand der Anode und der Kathode fast identisch, so das kaum Stoßprozesse stattfinden.\\
Bei hohen Temperaturen ist die Weglänge aber wesenlich leiner, weswegen der Franck-Hertz-Versuch dort durchgeführt werden kann.\\\\

\noindent


\begin{table}[h]
    \centering
    \small
    \begin{tabular}{S [table-format=5.5] S [table-format=5.5] S [table-format=5.5] S [table-format=5.5]}
        \toprule
        {Wert} & {$\text{Messwert}  $} & {$\text{Theoriewert }$}& {$\text{relative Abweichung }$}\\
        \midrule
        \text{$\increment E$} & \text{$\SI{5.45(40)}{\eV}$}& \text{$\SI{10.438}{\eV}$} & \text{$\SI{48(4)}{\percent}$} \\
        \text{$\lambda$} & \text{$\SI{227.385}{\nano\metre}$} & \text{$\SI{118.782}{\nano\metre}$} &\text{$\SI{-91(14)}{\percent}$} \\
        \bottomrule
    \end{tabular}
\caption{Die Ergebnisse der Auswertung der Franck-Hertz-Kurve und ihre Abweichung von der Theorie dargestellt.  }
\label{tab:4}
\end{table}