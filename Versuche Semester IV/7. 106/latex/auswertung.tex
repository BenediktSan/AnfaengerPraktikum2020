\newpage
\section{Auswertung}

    \subsection{Pendellänge 1}

        \noindent Die ersten Messreihen wurden mit einem Pendel der Länge $l = 0.25 \si{\m}$ durchgeführt.

        \subsubsection{Gleichsinnige Schwingung}

            \noindent Zur Bestimmung der Schwingungsdauern werden die Zeiten gemessen, nach dem das jeweilige Pendel 10 vollständige Schwingungen 
            durchlaufen hat. Somit müssen diese Messdaten(\ref{tab:frei1}) noch einmal durch 10 dividiert werden.

            \begin{table}[ht]
                \centering
                \caption{Messdaten für frei schwingende Pendel der ersten Messreihe.}
                \label{tab:frei1}
                \begin{tabular}{c c}
                 \toprule
                 $T_1 / \si{\s}$ & $T_2 / \si{\s}$\\
                 \midrule
                 5.78  &  5.66 \\
                 5.69  &  5.84 \\
                 5.85  &  5.85 \\
                 5.76  &  5.67 \\
                 5.70  &  5.43 \\
                 5.85  &  5.92 \\
                 5.84  &  5.86 \\
                 5.81  &  5.95 \\
                 5.71  &  6.01 \\
                 5.66  &  5.69 \\
                 \bottomrule
                \end{tabular}
            \end{table}

            \noindent Mit der Formel für den Mittelwert 

            \begin{equation}
                x = \frac{1}{N} \sum_{k=1}^{N} x_k
                \label{eqn:mittel}
            \end{equation}

            \noindent und der Formel für den Fehler des Mittelwertes

            \begin{equation}
                \sigma_{x} = \frac{\sigma}{\sqrt{N}}
                \label{eqn:mif}
            \end{equation}

            \noindent mit $N$, der Anzahl der Messungen und $\sigma$ die Standardabweichung, lassen sich die Messwerte mitteln.\\
            Hieraus ergibt sich für Pendel 1 eine 
            Schwingungsdauer $T_{+,1} = (1.153 \pm 0.014) \si{\second}$ und eine Schwingungsdauer von 
            $T_{+,2} = (1.158 \pm 0.033) \si{\second}$ für das zweite Pendel. Im Mittel hat dieses System eine Schwingungsdauer von 
            $T_+ = (1.15 \pm 0.014) \si{\second}$.

        \subsubsection{Gegensinnige Schwingung}
            
            \noindent Nun wird die Schwingungsdauer $T_{-}$ bei gegensinniger Schwingung untersucht. Auch hier wurden 10 Schwingungen gemessen, 
            weshalb die Daten(\ref{tab:geg1}) noch einmal durch 10 geteilt wurden.

            \begin{table}[ht]
                \centering
                \caption{Messdaten für frei gegensinniges Pendel der ersten Messreihe.}
                \label{tab:geg1}
                \begin{tabular}{c }
                 \toprule
                 $T_- / \si{\s}$ \\
                 \midrule
                 4.61  \\ 
                 4.68  \\ 
                 4.83  \\ 
                 5.03  \\ 
                 4.67  \\ 
                 4.72  \\ 
                 4.86  \\ 
                 4.55  \\ 
                 5.02  \\ 
                 4.87  \\ 
                \end{tabular}
            \end{table}

            \noindent Nach den bereits benutzten Fehlerformeln ergibt sich die Schwingungsdauer $T_- = (0.957 \pm 0.031) \si{\second}$.
            Mit diesen Messwerten lässt sich nach der Formel 

            \begin{equation*}
            K = \frac{T_{+}^2 - T_{-}^2}{T_{+}^2 + T_{-}^2} 
            \end{equation*}

            \noindent mit der zugehörigen Gaußchen Fehlerfortpflanzung
            
            \begin{equation}
            \sigma_f = \sqrt{\sum_{i=1}^{n} \left( \frac{\partial f}{\partial x_i} \right)^2 \sigma_{x_i}^2}
            \label{eqn:gauss}
            \end{equation}
            
            \noindent über

            \begin{equation*}
                \sigma_{\kappa} = \sqrt{\sigma_{T_{+}}^{2} \left(- \frac{2 T_{+} \left(T_{+}^{2} - T_{-}^{2}\right)}{\left(T_{+}^{2}
                 + T_{-}^{2}\right)^{2}} + \frac{2 T_{+}}{T_{+}^{2} + T_{-}^{2}}\right)^{2} + \sigma_{T_{-}}^{2} \left(- \frac{2 T_{-}
                  \left(T_{+}^{2} - T_{-}^{2}\right)}{\left(T_{+}^{2} + T_{-}^{2}\right)^{2}} - \frac{2 T_{-}}{T_{+}^{2} + T_{-}^{2}}\right)^{2}}
            \end{equation*}

            \noindent der Kopplungsgrad zu $K = 0.93 \pm 0.05$ bestimmen

        \subsubsection{Gekoppelte Schwingung}

            \noindent Zuletzt wird noch das System gekoppelter Pendel untersucht und die Schwingungsdauer sowie die Schwebungsdauer gemessen. 
            Hier wurde die Schwebungsdauer für eine einfache und die Schwingungsdauer für eine doppelte Periode gemessen.\\
            Somit muss diese auch noch einmal halbiert werden. 
            Die Daten zu diesen Messungen sind in Tabelle(\ref{tab:gek1}) zu finden.

            \begin{table}[ht]
                \centering
                \caption{Messdaten für das gekoppelte Pendel der ersten Messreihe.}
                \label{tab:gek1}
                \begin{tabular}{c c}
                 \toprule
                 $T / \si{\s}$ & $T_{\text{S}} / \si{\s}$\\
                 \midrule
                 2.30   & 5.85 \\
                 2.24   & 6.01 \\
                 2.02   & 6.13 \\
                 2.34   & 6.12 \\
                 2.25   & 5.77 \\
                 \bottomrule
                \end{tabular}
            \end{table}

            \noindent Analog zu den bisher ausgerechnetten Schwingungsdauern wird nun wieder der Durchschnittswert für diese Werte ausgerechnet.
            Es ergibt sich für die Schwingungsdauer $T = (1.11 \pm 0.06) $ und $T_{\text{S}} = (5.98 \pm 0.14) \si{\second}$ für die Schwebungsdauer.
            Diese gemessenen Werte können aber auch nach der Formel 

            \begin{align*}
                T_\text{S, theoretisch} = \frac{T_{+} \cdot T_{-}}{T_{+} - T_{-}}
            \end{align*}

            \noindent mit den vorher gemessenen Werten berechnet werden. Die dazu genutzten Werte $T_+$ und $T_-$ sind jedoch auch fehlerbehaftet, 
            somit wird auch hier nach der Gaußchen Fehlerfortpflanzung mittels der Formel 

            \begin{equation*}
                \sigma_{T_S} = \sqrt{\sigma_{T_{+}}^{2} \left(- \frac{T_{+} T_{-}}{\left(T_{+} - T_{-}\right)^{2}} + \frac{T_{-}}{T_{+} - T_{-}}\right)^{2} + \sigma_{T_{-}}^{2} \left(\frac{T_{+} T_{-}}{\left(T_{+} - T_{-}\right)^{2}} + \frac{T_{+}}{T_{+} - T_{-}}\right)^{2}}
            \end{equation*}
            die theoretische Schwebungsdauer zu $T_{\text{S,theo}} = (5.6 \pm 1.1) \si{\second}$ berechnet
            \noindent 

            \begin{table}[H]
                \centering
                \caption{Messdaten für das gekoppelte Pendel der ersten Messreihe.}
                \label{tab:gek1}
                \begin{tabular}{c c}
                 \toprule
                 $T / \si{\s}$ & $T_{\text{S}} / \si{\s}$\\
                 \midrule
                 2.30   & 5.85 \\
                 2.24   & 6.01 \\
                 2.02   & 6.13 \\
                 2.34   & 6.12 \\
                 2.25   & 5.77 \\
                 \bottomrule
                \end{tabular}
            \end{table}

        \subsubsection{Frequenzen}

            \noindent Die Frequenzen $\omega_{+}$, $\omega_{-}$ und $\omega_{\text{S}}$ lassen sich aus den gemessenen Werten nach der Formel 

            \begin{equation*}
                \omega = \frac{2 \pi}{T}
            \end{equation*}

            \noindent zu $\omega_{+} =( 5.44 \pm 0.08) \si{\hertz}$, $\omega_{-} = (6.57 \pm 0.21) \si{\hertz}$ und 
            $ \omega_{\text{S}}= (1.051 \pm 0.025) \si{\hertz}$ berechnen.

            \noindent Diese Frequenzen könen aber auch mit den aus der Theorie hergeleiteten Formeln 

            \begin{align*}
            \omega_\text{+, theo} &= \sqrt{\frac{g}{l}}\\
            \omega_\text{-, theo} &= \sqrt{\frac{g}{l} + \frac{2 K}{l}}\\
            \omega_\text{S, theo} &= \omega_\text{+, theo} - \omega_\text{-, theo}
            \end{align*}

            \noindent berechnet werden. Dies ergibt dann $\omega_\text{+, theo} = 6.26 \si{\hertz}$, $\omega_\text{-, theo} = (6.830 \pm 0.030) \si{\hertz}$ 
            und $\omega_\text{S, theo} = (-0.567 \pm 0.030) \si{\hertz}$.

    \subsection{Pendellänge 2}

        \noindent Die nächsten Messreihen wurden mit einem Pendel der Länge $l = 0.25 \si{\m}$ durchgeführt. Hier werden die Formeln 
        nicht noch einmal aufgelistet.

        \subsubsection{Gleichsinnige Schwingung}

            \noindent Die Messwerte zur gleichsinnigen Schwingung aus Tabelle(\ref{tab:frei2}) wurden analog zur ersten Messreihe analog auf 
            eine Schwingungsdauer gemittelt.

            \begin{table}[ht]
                \centering
                \caption{Messdaten für das frei schwingende Pendel der zweiten Messreihe.}
                \label{tab:frei2}
                \begin{tabular}{c c}
                 \toprule
                 $T_1 / \si{\s}$ & $T_2 / \si{\s}$\\
                 \midrule
                 6.44  &  6.55 \\
                 6.42  &  6.50 \\
                 6.52  &  6.31 \\
                 6.58  &  6.37 \\
                 6.26  &  6.43 \\
                 6.30  &  6.41 \\
                 6.28  &  6.50 \\
                 6.65  &  6.56 \\
                 6.48  &  6.53 \\
                 6.57  &  6.55 \\
                 \bottomrule
                \end{tabular}
            \end{table}
            
            \noindent Es berechnet sich eine Schwingungsdauer von $T_{+,1} = (1.290 \pm 0.026) \si{\second}$, $T_{+,2} = (1.294 \pm 0.016) \si{\second}$
            und somit eine gesamt Schwingungsdauer $T_+ = (1.292 \pm 0.015) \si{\second}$

        \subsubsection{Gegensinnige Schwingung}

            \noindent Auch hier werden die Werte aus Tabelle(\ref{tab:geg1}) wie in der ersten Messreihe auf eine Schwingung reduziert 
            um dann eine Schwingungsdauer von $T_- =( 0.957 \pm 0.031) \si{\second}$ zu erhalten. Mit den Messwerten $T_+$ und $T_-$ kann jetzt  
            der Kopplungsgrad zu $K = 1.25 \pm 0.05$ berechnet werden.

            \begin{table}[ht]
                \centering
                \caption{Messdaten für das gegensinnige Pendel System der zweiten Messreihe.}
                \label{tab:geg2}
                \begin{tabular}{c }
                 \toprule
                 $T_- / \si{\s}$ \\
                 \midrule
                 5.50  \\ 
                 5.58  \\ 
                 5.57  \\ 
                 5.43  \\ 
                 5.45  \\ 
                 5.52  \\ 
                 5.52  \\ 
                 5.45  \\ 
                 5.38  \\ 
                 5.50  \\ 
                \end{tabular}
            \end{table}

        \subsubsection{Gekoppelte Schwingung}

            \noindent Analog zu den bisherigen Rechnungen werden die Daten(\ref{tab:gek2}) wieder auf eine Schwingung reduziert und gemittelt um 
            dann $T = (1.088 \pm 0.031) \si{\second}$ und $T_{\text{S}} =(7.65 \pm 0.24) \si{\second}$ zu erhalten.

            \begin{table}[ht]
                \centering
                \caption{Messdaten für das gekoppelte Pendel der zweiten Messreihe.}
                \label{tab:gek2}
                \begin{tabular}{c c}
                 \toprule
                 $T / \si{\s}$ & $T_{\text{S}} / \si{\s}$\\
                 \midrule
                 2.14  &  7.22 \\
                 2.17  &  7.58 \\
                 2.17  &  7.77 \\
                 2.11  &  7.91 \\
                 2.29  &  7.77 \\
                 \bottomrule
                \end{tabular}
            \end{table}

            \noindent Die Schwebungsdauer kann aber auch wieder theoretisch zu $T_{\text{S,theo}} = (7,3 \pm 0,7) \si{\second}$ berechnet werden.

        \subsubsection{Frequenzen}

            \noindent Die Unterscheidung zwischen gemessenen und theoretisch berechneten Frequenzen ergibt hier 
            $\omega_{+} = (4.86 \pm 0.06) \si{\hertz}$, $\omega_{-} = (5.72 \pm 0.06) \si{\hertz}$ und 
            $ \omega_{\text{S}}= (0.821 \pm 0.026) \si{\hertz}$ für die gemessenen Frequenzen. \\
            Nach den bereits genutzten Formeln für die theoretischen Werte ergibt sich
            $\omega_\text{+, theo} = (5.29) \si{\hertz}$, $\omega_\text{-, theo} = (5.930 \pm 0.022) \si{\hertz}$ 
            und $\omega_\text{S, theo} = (-0.637 \pm 0.022) \si{\hertz}$.


