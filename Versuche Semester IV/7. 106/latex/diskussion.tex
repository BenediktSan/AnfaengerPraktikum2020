\section{Diskussion}

    \subsection{Vergleichen der Werte}
    
        \noindent Zunächst werden die gemessenen Werte mit den aus der Theorie berechneten Werten verglichen. \\
        Für das erste Pendel wurde eine 
        Schwingungsdauer $T_+ = (1.155 \pm 0.018) \si{\s}$ gemessen und eine Schwingungsdauer von $T_{+\text{,theo}} = 1.001 \si{\s}$ berechnet,
        daraus ergibt sich eine Abweichung von $(15.2 \pm 1.8) \%$.
        Die Abweichung von $T_+$ für das zweite Pendel berechnet sich aus $T_+ = (1.292 \pm 0.015) \si{\s}$ und 
        $T_{+\text{,theo}} = 1.187 \si{\s}$ zu $(8.9 \pm 1.3) \%$.\\\\
        Logischerweise ergeben die Abweichungen von $\omega_+$ analog zu den Schwingungsdauer $T_+$ eine Abweichung von $(15.2 \pm 1.8) \%$ 
        für die erste Pendellänge und $(8.9 \pm 1.3) \%$ für die zweite Pendellänge.\\
        Werden die Frequenzen $\omega_-$ verglichen, ergibt sich für die erste Länge aus $\omega_{-} = (6.57 \pm 0.21) \si{\hertz}$ und
        $\omega_\text{-, theo} = (6.830 \pm 0.030 )\si{\hertz}$ eine Abweichung von $ (4.0 \pm 3.3) \%$. \\
        Übertragen auf die zweite Pendellänge berechnet sich aus  $\omega_{-} = (5.72 \pm 0.06) \si{\hertz}$ und 
        $\omega_\text{-, theo} = (5.930 \pm 0.022) \si{\hertz}$ eine Differenz von $ (3.6 \pm 1.1) \%$.\\
        Beim Vergleichen von $\omega_{\text{S}}$ ergibt sich das Problem, dass die theoretischen Werte ein anderes Vorzeichen als die gemessenen 
        Werte haben. Wird dieses nicht beachtet ergibt sich aus $\omega_{\text{S}}= (1.051 \pm 0.025 )\si{\hertz}$ und 
        $\omega_\text{S, theo} = -(0.567 \pm 0.030) \si{\hertz}$ eine Abweichung von $(46.1 \pm 3.1) \%$ für das erste Pendel.
        Die selbe Rechnung ergibt zuletzt für das zweite Pendel aus $ \omega_{\text{S}}= (0.821 \pm 0.026) \si{\hertz}$ und 
        $\omega_\text{S, theo} = (-0.637 \pm 0.022 )\si{\hertz}$ eine Abweichung von $ (22.0 \pm 4.0) \%$.
       
    \subsection{Diskussion der Werte}

        \noindent Insgesamt entstehen in den Rechnungen relativ kleine Messungenauigkeiten innerhalb der Messreihen.\\ Somit wurden die 
        Messungen insich relativ gut und identisch aufgenommen, jedoch ergibt der Vergleich zwischen theoretischen und praktisch 
        gemessenen Werten eine relativ große Abweichung. Diese Abweichung ist bei der ersten Pendellänge von $l = 0.25 \si{\meter}$ 
        konstant größer. Über beide Versuch hin weg ist die Abweichung bei der Schwebungs-Frequenz mit 46.1\% und 22\% deutlich größer als bei 
        den anderen Messwerten. Dies lässt sich dadurch erklären, dass in der Rechnung zu diesem Wert, sich die Abweichungen der vorheringen 
        Messgrößen kummulieren.\\

        \noindent Mögliche Erklärungen für diese Abweichung sind einige systematischen Fehler die in diesem Versuch auftreten. Dazu gehören 
        Reibung an den aufhänge Punkten, eventuell bei leichten Berührungen an der Wand oder innererhalb der Kopplungsfeder. Dies führt 
        alles zu Energieverlusten und somit zu Abweichungen der gemessenen Schwingungsdauer von der theoretischen Vorhergesage.\\
        Die Rechnungen für die Theoriewerte basieren auch auf dem mathematischem Pendel, was nicht der Realität entspricht.\\
        Auch menschliche Fehler wie ungenaues Ablesen der Zeit und eine ungleichmäßige Auslenkung der Pendel sind nicht auszuschließen.
        Weiterhin ist die in der Theorie benutzte Näherung sin($\alpha$) $\approx \alpha$ nur für kleine Winkel gültig, es entstehen 
        relativ schnell Fehler im Bereich 0.5\% entstehen.\\\\

        \noindent Insgesamt sind die Ergebnisse aber recht gut, wenn in Betracht gezogen wird, dass sehr schnell systematische Fehler auftreten können.
\newpage
