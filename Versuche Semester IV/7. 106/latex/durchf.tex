\newpage
\section{Durchführung}
Für die Durchführung dieses Versuches werden zwei nebeneinander an der Wand angebrachte und frei gelagerte Metallpendel genutzt. 
Die Schwungmasse ist dabei höhenverstellbar, so dass unterschiedliche Pendellängen untersucht werden können.\\\\
Zuerst werden beide Pendel auf die selbe Pendellänge eingestellt. Anschließend werden von beiden Pendeln die Periodendauer der Schwingung gemessen.\\
Dabei werden sie um einen kleinen Winkel ausgelenkt und dann für eine Messung fünf Perioden gemessen. 
Mit diesem vorgehen wird der Fehler verkleinert, da diese Messung einfacher zu realisieren ist als eine Periode zu messen.\\
Dies wird für beide Pendel jeweils zehn mal wiederholt.\\\\
Nun wird mit dem selben Vorgehen die Periode der gegensinnigen gekoppelten Schwingung gemessen. 
Dafür wird zwischen den beiden Pendeln eine Feder angebracht, die zur Kopplung der Bewegung dient.
Nun werden beide Pendel um den gleichen Winkel in entgegengesetzte Winkel ausgelenkt. Dies wird am besten realisiert in dem die Pendel nach innen hin auslenkt werden. 
Die Messung dieser Schwingungsperiode soll auch mehrfach wiederholt werden.\\\\
Zuletzt soll die Periodendauer der gekoppelten Schwingung und auch die Schwebungsdauer gemessen werden. 
Eins der beiden gekoppelten Pendeln muss sich dabei in Ruhe befinden, während das andere ausgelenkt werden. \\
Für das ausgelenkte sollen dann für ein bis zwei Perioden die Dauer gemessen werden.
Die Schwebungsdauer wird gemessen in dem der Zeitraum zwischen den Ruhezuständen des unausgelenkten Pendels aufgenommen wird.\\
Dies soll mehrfach wiederholt werden.\\\\

\noindent
Die zuvor beschriebenen Messungen sollen nun mit einer anderen Pendellänge wiederholt werden.




