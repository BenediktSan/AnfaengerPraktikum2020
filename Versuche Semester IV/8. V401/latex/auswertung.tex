\newpage
\section{Auswertung}

\subsection{Bestimmung der Wellenlänge des Laserlichts}

\noindent Im ersten Teil des Versuches wurden Messwerte aufgenommen mit denen die Wellenlänge des vom Laser abgestrahlten Lichts bestimmt werden soll.\\\\
Bei der Messung wurde dabei einer der Spiegel der Arme des Interferometers mit Hilfe eines feinen Motors um die Länge $l=\SI{5(00002)}{\milli\metre}$ verschoben.\\
Dies gilt für alle Messungen außer die vierte. Dort wurde um $\SI{5.05(00002)}{\milli\metre}$ verschoben, was auch bei der daraus berechneten wahren Verschiebung $d$ in den in Tabelle \ref{tab:lam}, zu sehen ist.\\
Die wahre Verschiebung $d$ wird benötigt, da das Verstellrad des Spiegels keine 1:1 Übersetzung besitzt, sondern eine 1:5,056 Übersetzung.\\
Daraus berechnet sich $d$ mit folgender Gleichung:
\begin{equation*}
    d=l \cdot 5.056
\end{equation*}
\noindent
Die daraus berechneten Werte finden sich, inklusive der gemessenen Interferenzmaximaanzahl und den berechneten Wellenlängen in Tabelle \ref{tab:lam}.\\
Diese Wellenlängen lassen sich mit der folgenen Gleichung bestimmen:
\begin{equation*}
    \lambda=\frac{2d}{z}
\end{equation*}


\begin{table}[ht]
    \centering
    \caption {Die Werte für die Verschiebung $d$ des Spiegels, mit den dazu korrespondierenden gezählten Interferenzmaxima und daraus berechneten Wellenlängen inklusive Abweichung.\\
    Die Verschiebung ist dabei auch mit einer Abweichung von $\SI{0.39557}{\nano\metre}$ und die Anzahl der Maxima mit einer Abweichung von $\SI{40}{}$  behaftet.}
    \sisetup{table-format=1.4}
    \begin{tabular}{ S [table-format=1.2] | S [table-format=5.0] S [table-format=3.3] @{$\; \pm{} \;$}  S [table-format=2.3]  }
        \toprule
        {$d \mathbin{\scalebox{1.5} / }\si{\milli\metre}$} & {$ \text{Interferenzmaximaanzahl} z $} &\multicolumn{2}{c}{ $\lambda \mathbin{\scalebox{1.5} / }\si{\nano\metre}$}\\
        \midrule
        0.99 &2957& 668.870 & 9.052 \\
        0.99 &2950& 670.457 & 9.095 \\
        0.99 &2966& 666.840 & 8.997 \\
        1.00 &2943& 678.772 & 9.229 \\
        0.99 &2634& 750.891 & 11.407 \\
        0.99 &2936& 673.654 & 9.182 \\
        0.99 &2964& 667.290 & 9.009 \\
        0.99 &2961& 667.966 & 9.027 \\
        \bottomrule
    \end{tabular}
\label{tab:lam}
\end{table}

\noindent Gemittelt ergibts sich hieraus eine Wert für die Wellenlänge von $\lambda=\SI{680.593(26831)}{\nano\metre}$.

\subsection{Bestimmung des Brechungsindexes von Luft}

\noindent Im Folgenden soll nun  der Brechungsindex von Luft untersucht werden.\\
Die beim herauspumpen und beim wieder hereinlassen der Luft der Messzelle gemessenen Anzahlen der Intensitätsmaxima sind in Tabelle \ref{tab:updown} zu finden.\\
Der abgeschätzte Fehler dieser Werte ist wesentlich kleiner als $1$, weswegen er für die weiteren Rechungen nicht beachtet wird.\\





\begin{table}[ht]
    \centering
    \caption {Die Messwerte für die Anzahl der Interferenzmaxima. Der abgeschätzte Fehler dabei ist kleiner als $1$ und wird deswegen nicht weiter beachtet.}
    \begin{tabular}{ S [table-format=2.0]  S [table-format=2.0] }
        \toprule
        {$ \text{Interferenzmaximaanzahl für das Abpumpen}  $} &{$ \text{Interferenzmaximaanzahl für das Vakuum ablassen}  $}\\
        \midrule
        32 &50 \\
        31 &20 \\
        31 &39 \\
        30 &30 \\
        30 &26 \\
        \bottomrule
    \end{tabular}
\label{tab:updown}
\end{table}

\noindent Die Messzelle hat dabei die Länge $b=\SI{50}{\milli\metre}$. Anschließend wurde der Druck innerhalb der Zelle um $\Delta p = \SI{600}{\mmHg} = \SI{7999.32}{Pa}$ verringert.\\
Mit diesen Werten und mit $T_0= \SI{273.15}{\kelvin}$ und dem vom Laser abgelesenen Wert für die Wellenlängen $\lambda=\SI{635}{\nano\metre}$, lässt sich dann $\increment n$ bestimmen.\\
Zusätzlich wird aber noch $z$ benötigt, welches durch das Bestimmen des Mittelwertes aus den in Tabelle \ref{tab:updown} aufgetragenen Messwerten zu $z= \SI{32(8)}{}$ führt.\\
Mit diesen Werten lässt sich dann im Folgenden weiter rechnen.
\begin{equation*}
   \increment n(p_0,T_0)=\frac{z \lambda}{2 b}=\SI{0.00020(5)}{}
\end{equation*}
Aus diesem Ergebnis lässt sich dann der gesuchte Brechungsindex bestimmen.
\begin{align*}
    n(p_0,T_0)&=1+ \increment n(p,p')\frac{T}{T_0}\frac{p_0}{\increment p}\\
    n(p_0,T_0)&=\SI{1.00028}{7}
\end{align*}
Dabei gilt noch zusätzlich für $p_0=\SI{101320}{\pascal}$.