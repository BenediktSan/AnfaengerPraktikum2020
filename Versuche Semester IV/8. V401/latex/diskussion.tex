\newpage
\section{Diskussion}

\noindent Die Versuchsdurchführung lief ohne allzu große Probleme ab. 
Die Messwerte \ref{tab:lam} der ersten Messreihe, die zur Bestimmung der Wellenlänge genutzt wurden, zeigen fast alle nur kleine Abweichung voneinander.\\
Allerdings zeigt ein Wert eine große Abweichung und die später gemessenen Werte sind allesamt ungenauer. Dies liegt daran, dass die erzeugten Interferenzlinien nicht sehr deutlich waren, wodurch der Zähler sie nicht regristriert hat.
Bei den späteren Messungen musste deswegen nachjustiert werden, so dass überhaubt Interferenzlinien gemessen werden konnten.\\
Zusätzlich wurde einmal für eine etwas längere Spiegelstrecke die Interferenzmaxima gezählt. Da der untersuchte Vorgang allerdings linear ist sollte dies zu vernachlässigen sein.\\
Die obigen Gründe führen dazu, dass die Abweichung der Messwerte mit einem Wert von $40$ Zählern recht hoch abgeschätzt wurde.\\
Das Ergebnis dieser Messreihe ist $\lambda= \SI{680.593(26831)}{\nano\metre}$.
Im Vergleich mit dem vom Laser abgelesenen Theoriewert von $\lambda_{theo}=\SI{635}{\nano\metre}$ ergibt sich die relative Abweichung vom Theoriewert zu:
\begin{equation*}
    \frac{\lambda_{theo}-\lambda}{\lambda_{theo}}=\SI{-7(4)}{\percent}
\end{equation*}
Dieses Ergebnis ist wenn die nicht optimale Durchführung in Betracht gezogen wird trotzdem recht gut. Dies liegt vermutlich an der langen Durchführungsdauer.\\\\

\noindent Für die zweite Messreihe, die zur Bestimmung des Brechungsindexex von Luft unternommen wurde, sind die Messreihe in Tabelle \ref{tab:updown} zu finden. 
Obwohl diese Werte eine halbwegs große Streuung besitzen und die Messungen die beim Pumpen aufgenommen wurden schwierig fein zu justieren waren, liefert die Auswertung ein sehr gutes Ergebnis.\\
Für den Brechungsindex ergibt sich dabei $n=\SI{1.00028}{7}$, wobei der Theoriewert $n_{theo}=\SI{1.00027316}{}$\cite{n} der Brechungsindex von Luft für $\lambda_{theo}$ ist.\\
Für die relative Abweichung von dem Theoriewert ergibt sich damit nach der obigen Formel  $\SI{-0.000(0007)}{\percent}$.\\
Trotz nicht optimaler Durchführung ist diese Ergebnis also sehr gut.\\\\
Alles in allem sind die gemessenen Werte also sehr zufriedenstellend. Sie führen alle zu Ergebnissen, die nur eine recht kleine relative Abweichung von der Theorie zeigen. 
\newpage
