\section{Theorie}

    \subsection{Inter}

        \noindent Bums


    3. 
        keine Interferenz bei 2 normalen Quellen 
            unterschiedlich zeitlich abhängige Phasen
                Im mittel keine Interferenz

            Es gibt Emissionsakte innerhalb des Atoms
                Die Wellengruppen einer dieser Akte ist kohärent
                jedoch licht der gleichen Quelle aber von unterschiedlichen AKten ist inkohärent

            kohärentes Licht 
                einheitliche Gleichung mit festem k, \omega , \delta

            erzeugung nur durch selbe Quelle möglich 
                Strahlteiler 
                    Phasenverschiebung durch Wegunterschied 

            emissionsakt nur edlichiche Länge \tau 
                ist ser Wegunterschied zu groß verwschinden die Interferenzerscheinungen
                    das Licht des selbes Aktes kommt zu unteschiedlichen Zeiden delta t größer \tau an und ist inkohärent
                es lässt sich eine Kohärenzlänge $l$ bestimmen 

            Aus dem Fourierschen Thoerem folgt, dass auch ein Wellenzug in sich nicht monochromatisch sein kann 
                daraus folgt, dass eigentlich keine Intefernz möglioch ist 
                    Das Frequenzspektrom oder der Wegunterschied muss klein genug sein um dies dennoch zu ermöglichen 

            Fourier Rechnungen 
            maxima beo $\omega = \omega_0$

            Minima bei 

            \begin{equation}
                \omega_{\text{N}} = \omega_0 \pm 2\frac{\pi}{\tau}              
            \end{equation}

            größter Teil im Bereich 

            \begin{equation}
                \omega_0 - 2\frac{\pi}{\tau} < \omega < \omega_0 + 2\frac{\pi}{\tau}
            \end{equation}

            Somit ist die "Breite" der Verteilungsfunktion durch 

            \begin{equation}
                \Delta \omega = 2 \frac{\pi}{\tau}
            \end{equation}

            charakterisiert. Die breite der Wellenlängenverteilung kann durch die Formel 

            \begin{equation}
                \lambda_0 := \frac{2 \pi \text{c}}{\omega_0}
            \end{equation}

            Differentieren und einstzten von Formel (10) ergibt dann 

            \begin{equation}
                \Delta \lambda = \frac{\lambda_0^2}{\text{c}\tau}
            \end{equation}

            über die Kohärenzzeit $\tau = \frac{l}{c}$ ergibt sich 

            \begin{equation}
                \Delta \lambda = \frac{\lambda_0^2}{l}
            \end{equation}

            Die benutze Lichtquelle ist auch keine Punktförmige Quelle, sonder hat eine endliche Ausdehnung. Dies schließt kohärenz nicht aus, 
            lässt den Effekt aber ein wenig verschleiern. Dies ist in Skizze (ABbildung 3) zu sehen.

        Bums zu Skizze 3, im Altprotokoll nachschauen 

            \begin{equation}
                \Delta \varphi = \frac{2 \pi}{\lambda} a \text{sin} \psi 
            \end{equation}

            Kohärenzbedingung für ausgedehnte Lichtquellen

            \begin{equation}
                a \text{sin} << \frac{\lambda}{2}
            \end{equation}

            Für Interferenz bei ausgedehnten Quellen müssen also entweder die Abmessungen oder der Oeffnungswinkel klein gehalten werden 

            Polarisation
            Sind die Lichtquellen polarisiert entsteht nur Interferenz wenn die Lichtquellen nicht senkrecht zueinander polarisiert 
            sind da sonst nach dem Skalarprodukt 

            \begin{equation}
                \vec{\text{E}}_1 \cdot \vec{\text{E}}_2 = 0
            \end{equation}

            ist.

        4. Prinzipieller Aufbau des Michelson Interferometers-Interferometers

            Generells sind Interferometer Geräte die mittels interferenzeffekten optische Größen messen. Im Michelson Interferometer wird 
            interferenz dadurch ermöglicht, dass ein Strahl in mindestens Teilbündel aufgeteilt wird und wieder zusammen geführt wird nach  
            dem einer der Teilstrahlen verädert wird. Die Aufteilung geschieht hier wie in Abbildung(\ref{}) zu sehen ist, 
            durch eine semipermeable Platte.





