\section{Diskussion}

    \noindent Die Bragg-Bedingung konnte gut bestätigt werden. Der gemessene Maximalwert bei $\alpha = \SI{28.2}{\degree}$ liegt sehr nah an dem 
    theoretisch berechnetem Sollwinkel bei $\SI{28}{\degree}$, zwischen den Werten liegt nur eine Abweichung von $\SI{0.7}{\percent}$. 
    Würde diese Bedingung nicht so gut bestätigt werden gäb es einen systematischen Fehler in der Apperatur und die darauf gemessenen Spektren 
    könnten nicht den jeweilien Peaks zugeordnet werden.\\

    \noindent Auch die Analyse des Emissionsspektrums von Kupfer hat gute Werte ergeben.
    \begin{table}
        \centering
        \caption{Der Vergleich der $K_{\alpha}$ und $K_{\beta}$ Linie mit den Literatur Werten.}
        \begin{tabular}{c c c c}
            \toprule
            Linie & $E_{K , \text{Vers}} \mathbin{/} \si{\electronvolt}$ & $ E_{K , \text{Theo}} \mathbin{/} \si{\electronvolt}$ & Abweichung $ \, \mathbin{/} \si{\percent}$\\
            \midrule
            $\alpha$ & 8043 & 8109 & 0.40\\
            $\beta$  & 8872 & 8914 & 0.12\\
            \bottomrule
        \end{tabular}
    \end{table}\\

    \noindent Nun werden die Abschirmkonstanten der anderen Metalle mit den Literaturwerten verglichen, dies ist in Tabelle(\ref{tab:sigmadisk}) 
    zu finden. Die Abweichungen sind auch hier sehr gering, die bestehenden Abweichungen könnten daher entstanden sein, dass das 
    Röntgengerät ur in Diskreten Werten misst und die exakten Stellen vielleicht nicht getroffen wurden.

    \begin{table}
        \centering
        \caption{Vergleich der Abschirmkonstanten}
        \label{tab:sigmadisk}
        \begin{tabular}{c c c c}
            \toprule
            Metall & $\sigma_{\text{Vers}}$ & $ \sigma_{\text{Theo}}$ \cite{sigma} & Abweichung $ \, \mathbin{/} \si{\percent}$\\
            \text{Zn}   & 3.61  & 3.56  & 1.46\\
            \text{Ga}   & 3.67  & 3.62  & 1.34\\
            \text{Bro}  & 3.87  & 3.84  & 0.67\\
            \text{Rub}  & 4.07  & 3.95  & 2.92\\
            \text{Str}  & 4.11  & 3.99  & 2.93\\
            \text{Zir}  & 4.30  & 4.39  & 2.13\\
            \bottomrule
        \end{tabular}
    \end{table}
        
    \noindent Zuletzt wird noch die Berechnung der Rydbergenergie diskutiert. Diese wurde aus der Linearen Regression zu 

    \begin{align*}
        R_{\infty, \text{Vers}} &= \SI{12.53}{\electronvolt}\\
        R_{\infty, \text{Theo}} &= \SI{13.6}{\electronvolt}
    \end{align*}

    \noindent berechnet. Daraus ergibt sich eine Abweichung von $\SI{8.52}{\percent}$, diese ist insgesamt noch relativ genau, jedoch relativ groß 
    im Vergleich zu den bisher berechneten Werten. Der Grund für diese Abweichung könnte sein, dass die Rechnung auf bereits Fehlerbehafteten 
    Werten aufbaut, oder es könnte auch sein, dass in der Rechnung noch kleinere Quantenmechanische Effekte nicht betrachtet worden sind.
    
    

    
    