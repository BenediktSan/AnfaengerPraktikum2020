\newpage
\section{Zielsetzung}
    Das Ziel des Versuches ist die Untersuchung des Röntgenemissionsspektrums  von Kupfer der
    Röntgenabsorbtionsspektren anderer Materialien. Des Weiteren sollen die maximale Intensität der Röntgenröhre und das energetische Auflösevermögen der Apparatur bestimmt werden.
    Zusätzlich wird auch das Moseleysche-Gesetz überprüft.

\section{Theoretische Grundlagen }

\subsection{Erzeugung von Röntgenstrahlung}

Röntgenstrahlung ist jene Strahlung, welche Energien von $\SI{10}{\eV}$ bis $\SI{200}{\kilo\eV}$.\\
Eine der vielen Methoden um Röntgenstrahlung zu Erzeugen ist die Beschleunigung von durch den glühelekktrischen Effekt ausgelösten Elektronen auf eine Bremsanode.\\


